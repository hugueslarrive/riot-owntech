\subsection{Prérequis}
(depuis \url{https://github.com/RIOT-OS/RIOT/wiki/Creating-your-first-RIOT-project}) :
\begin{itemize}
	\item Install and set up git
	\item Install the build-essential packet (make, gcc etc.). This varies based on
			the operating system in use.
	\item Install Native dependencies
	\item Install OpenOCD
	\item Install GCC Arm Embedded Toolchain
	\item On OS X: install Tuntap for OS X
	\item additional tweaks necessary to work with the targeted hardware (ATSAMR21)
	\item Install netcat with IPv6 support (if necessary)
			\begin{verbatim}sudo apt-get install netcat-openbsd}\end{verbatim}
	\item \begin{verbatim}git clone --recursive https://github.com/RIOT-OS/Tutorials\end{verbatim}
	\item Go to the Tutorials directory: cd Tutorials\\
\end{itemize}

Les 3 premiers points sont déjà OK sur ma machine.
\subsection{OpenOCD}

\url{https://github.com/RIOT-OS/RIOT/wiki/OpenOCD} :
\begin{verbatim}
OpenOCD (the open on-chip debugger) is an open source tool for debugging and
flashing microcontrollers. In RIOT we try to use this tool for as many
platforms as possible to reduce the overhead of having to keep track of many
different (and sometimes proprietary) tools.
\end{verbatim}

C'est donc l'outil qui devrait nous permettre de téléverser notre firmware. La page
explique son installation à partir des sources car les versions empaquetés dans les
distributions ne supporte pas les cartes les plus récentes supportées par RIOT. Cette
page datant de 2018 elle parle de la version 0.7.0 sur Mint 17 et de la version 0.9.0
compilée depuis les sources.\\

La version distribuée sur Debian 10 étant la 0.10.0 je vais m'en contenter dans un
premier temps et j'y reviendrai si nécessaire.

\subsection{GCC Arm Embedded Toolchain}

Sur Debian c'est le package gcc-arm-none-eabi déjà présent sur ma machine en versions
7-2018-q2, la version actuelle étant la 9-2019-q4. Comme pour OpenOCD je vais essayer
de commencer avec ça.

\subsection{En résumé}

Dans un shell root :
%{\scriptsize
\begin{verbatim}
dpkg --add-architecture i386
apt-get update
apt-get install libc6-dev-i386 libc6-dbg:i386 build-essential pkg-config \
uml-utilities bridge-utils git unzip
apt-get install openocd gcc-arm-none-eabi python3-serial
\end{verbatim}
%}
Dans un shell utilisateur :
%{\scriptsize
\begin{verbatim}
git clone --recursive https://github.com/RIOT-OS/Tutorials
cd Tutorials
\end{verbatim}
%}

Facile !

