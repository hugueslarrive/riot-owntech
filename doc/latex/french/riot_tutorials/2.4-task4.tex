Utiliser \texttt{xtimer} pour créer un thread qui affiche le temps
système courant toutes les 2 secondes.\\

\subsubsection{Thread handler}
{\small
\begin{lstlisting}
void *timer2s_handler(void *arg)
{
    /* Thread that print the current system time every 2 seconds... */
    (void)arg;
    
    xtimer_ticks32_t last_wakeup = xtimer_now();

    while (1) {
        printf("Current system time is %lu \(\mu\)s\n", xtimer_now_usec());
        xtimer_periodic_wakeup(&last_wakeup, PERIOD_US);
    }

    return NULL;
}
\end{lstlisting}
}
\subsubsection{nucleo}
{\scriptsize
\begin{verbatim}
2020-06-17 15:39:54,773 # reboot
2020-06-17 15:39:54,779 # main(): This is RIOT! (Version: 2020.04)
2020-06-17 15:39:54,781 # This is Task-04
2020-06-17 15:39:54,784 # Current system time is 5036 \(\mu\)s
> 2020-06-17 15:39:56,784 #  Current system time is 2005053 \(\mu\)s
2020-06-17 15:39:58,784 # Current system time is 4005052 \(\mu\)s
2020-06-17 15:40:00,784 # Current system time is 6005052 \(\mu\)s
ps
2020-06-17 15:40:02,222 # ps
2020-06-17 15:40:02,230 # 	pid | name                 | state    Q | pri | stack  ( used) | base addr  | current     
2020-06-17 15:40:02,238 # 	  - | isr_stack            | -        - |   - |    512 (  140) | 0x20000000 | 0x200001c8
2020-06-17 15:40:02,246 # 	  1 | idle                 | pending  Q |  15 |    256 (  132) | 0x200003e8 | 0x20000464 
2020-06-17 15:40:02,254 # 	  2 | main                 | running  Q |   7 |   1536 (  660) | 0x200004e8 | 0x2000093c 
2020-06-17 15:40:02,262 # 	  3 | timer2s              | bl mutex _ |   6 |   1024 (  356) | 0x20000aec | 0x20000e04 
2020-06-17 15:40:02,268 # 	    | SUM                  |            |     |   3328 ( 1288)
> 2020-06-17 15:40:02,784 #  Current system time is 8005052 \(\mu\)s
2020-06-17 15:40:04,784 # Current system time is 10005052 \(\mu\)s
2020-06-17 15:40:06,784 # Current system time is 12005053 \(\mu\)s
\end{verbatim}
}

Le printf imprime sur la console à 115200 bauds ce qui prend environ
87 $\mu$s par caractère. La fonction \texttt{xtimer\_periodic\_wakeup()}
se débrouille pour compenser ça. Le chiffre des unités oscille entre 2
et 3, il y a donc une légère imprécision.\\

\subsubsection{native}
Il y a une petite erreur à la compilation au niveau du printf, il faut
remplacer \%lu par \%u pour que ça fonctionne.
{\scriptsize
\begin{verbatim}
main(): This is RIOT! (Version: 2020.04)
This is Task-04
Current system time is 66 \(\mu\)s
> Current system time is 2000220 \(\mu\)s
Current system time is 4000204 \(\mu\)s
Current system time is 6000216 \(\mu\)s
ps
ps
	pid | name                 | state    Q | pri | stack  ( used) | base addr  | current     
	  - | isr_stack            | -        - |   - |   8192 (   -1) | 0x565f4380 | 0x565f4380
	  1 | idle                 | pending  Q |  15 |   8192 (  436) | 0x565f20a0 | 0x565f3f00 
	  2 | main                 | running  Q |   7 |  12288 ( 3212) | 0x565ef0a0 | 0x565f1f00 
	  3 | timer2s              | bl mutex _ |   6 |   8192 ( 2452) | 0x565ed0a0 | 0x565eef00 
	    | SUM                  |            |     |  36864 ( 6100)
> Current system time is 8000134 \(\mu\)s
Current system time is 10000222 \(\mu\)s
Current system time is 12000209 \(\mu\)s
Current system time is 14000232 \(\mu\)s
Current system time is 16000342 \(\mu\)s
Current system time is 18000165 \(\mu\)s
\end{verbatim}
}

Là c'est beaucoup moins stable ce qui est normal étant donné que linux
n'est pas un système temps réel. Un \texttt{renice} du processus linux
avec un priorité haute (-20) ne change pas grand chose. La gestion d'énergie
change sans cesse la fréquence du processeur sur ma machine, peut être
qu'une fréquence fixe améliorerait les chose mais ce n'est pas faisable
sans passer par le BIOS avec ce processeur...

\subsubsection{arduino-nano}
{\scriptsize
\begin{verbatim}
2020-06-17 16:35:00,748 # main(): This is RIOT! (Version: 2020.04)
2020-06-17 16:35:00,749 # This is Task-04
2020-06-17 16:35:00,781 # Current system time is 59124 \(\mu\)s
> 2020-06-17 16:35:02,781 #  Current system time is 2059280 \(\mu\)s
2020-06-17 16:35:04,784 # Current system time is 4059280 \(\mu\)s
2020-06-17 16:35:06,786 # Current system time is 6059280 \(\mu\)s
2020-06-17 16:35:08,788 # Current system time is 8059280 \(\mu\)s
2020-06-17 16:35:10,792 # Current system time is 10059280 \(\mu\)s
ps
2020-06-17 16:35:11,199 # ps
2020-06-17 16:35:11,266 # 	pid | name                 | state    Q | pri | stack  ( used) | base addr  | current     
2020-06-17 16:35:11,366 # 	  1 | idle                 | pending  Q |  15 |    128 (   94) |      0x562 |      0x58f 
2020-06-17 16:35:11,466 # 	  2 | main                 | running  Q |   7 |    384 (  364) |      0x5e2 |      0x659 
2020-06-17 16:35:11,566 # 	  3 | timer2s              | bl mutex _ |   6 |    256 (  142) |      0x462 |      0x4d5 
2020-06-17 16:35:11,621 # 	    | SUM                  |            |     |    768 (  600)
> 2020-06-17 16:35:12,794 #  Current system time is 12059280 \(\mu\)s
2020-06-17 16:35:14,797 # Current system time is 14059280 \(\mu\)s
2020-06-17 16:35:16,799 # Current system time is 16059280 \(\mu\)s
2020-06-17 16:35:18,802 # Current system time is 18059280 \(\mu\)s
\end{verbatim}
}
Ça fonctionne avec une stabilité surprenante (supérieure au stm32 !).
