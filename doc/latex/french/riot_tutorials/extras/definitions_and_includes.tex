\usepackage[T1]{fontenc}
\usepackage[utf8]{inputenc}
%~ \usepackage{lmodern}
\usepackage{helvet}
\renewcommand{\familydefault}{\sfdefault}
\usepackage[francais]{babel}
\usepackage{verbatim}
\usepackage{tikz}
\usepackage{eurosym} % pour € \euro{}
\usepackage{csquotes} % pour les guillemets

%\usepackage{newtxmath}

\usepackage[
  % Some remarks:
  % * drivers like 'pdftex' that can be detected automatically
  %   are not necessary
  % * breaklinks is rather an internal option.
  %   If a driver does not support it, then forcing the option
  %   let the text break across lines, but also the link
  %   areas are "broken". If the driver supports the option,
  %   then the option is enabled anyway.
  % * Information entries should be set outside,
  %   because LaTeX expands the package options,
  %   hyperref does not like them, if they are
  %   prematurely expanded.
  % * Hyperref has a new option for hiding links: hidelinks
  hidelinks,
  pagebackref,
  bookmarksopen,
  bookmarksnumbered,
  a4paper,
]{hyperref}
\hypersetup{
  pdfauthor={Hugues Larrive},
  % ...
}
% Adding package bookmark improves bookmarks handling.
% More features and faster updated bookmarks.
\usepackage{bookmark}

\usepackage{layout}
\oddsidemargin=-1cm
\usepackage[top=2cm, bottom=2cm, left=2cm, right=2cm]{geometry}

\usepackage{wrapfig}
\usepackage{graphicx}


\usepackage{listings}
\lstset{
     literate=%
		{à}{{\`a}}1
		{é}{{\'e}}1
		{ê}{{\^e}}1
		{è}{{\`e}}1
		{É}{{\'E}}1
		{ç}{{\c{c}}}1
}
\usepackage{color}

\definecolor{mygreen}{rgb}{0,0.6,0}
\definecolor{mygray}{rgb}{0.5,0.5,0.5}
\definecolor{mymauve}{rgb}{0.58,0,0.82}

\lstset{ 
  backgroundcolor=\color{white},   % choose the background color; you must add \usepackage{color} or \usepackage{xcolor}; should come as last argument
  basicstyle=\footnotesize,        % the size of the fonts that are used for the code
  breakatwhitespace=false,         % sets if automatic breaks should only happen at whitespace
  breaklines=true,                 % sets automatic line breaking
  captionpos=b,                    % sets the caption-position to bottom
  commentstyle=\color{mygreen},    % comment style
  deletekeywords={...},            % if you want to delete keywords from the given language
  escapeinside={\%*}{*)},          % if you want to add LaTeX within your code
  extendedchars=true,              % lets you use non-ASCII characters; for 8-bits encodings only, does not work with UTF-8
  frame=single,	                   % adds a frame around the code
  keepspaces=true,                 % keeps spaces in text, useful for keeping indentation of code (possibly needs columns=flexible)
  keywordstyle=\color{blue},       % keyword style
  language=Octave,                 % the language of the code
  morekeywords={*,...},            % if you want to add more keywords to the set
  numbers=left,                    % where to put the line-numbers; possible values are (none, left, right)
  numbersep=5pt,                   % how far the line-numbers are from the code
  numberstyle=\tiny\color{mygray}, % the style that is used for the line-numbers
  rulecolor=\color{black},         % if not set, the frame-color may be changed on line-breaks within not-black text (e.g. comments (green here))
  showspaces=false,                % show spaces everywhere adding particular underscores; it overrides 'showstringspaces'
  showstringspaces=false,          % underline spaces within strings only
  showtabs=false,                  % show tabs within strings adding particular underscores
  stepnumber=1,                    % the step between two line-numbers. If it's 1, each line will be numbered
  stringstyle=\color{mymauve},     % string literal style
  tabsize=2%,	                   % sets default tabsize to 2 spaces
  %title=\lstname                   % show the filename of files included with \lstinputlisting; also try caption instead of title
}

\lstdefinestyle{customc}{
  belowcaptionskip=1\baselineskip,
  breaklines=true,
  frame=L,
  xleftmargin=\parindent,
  language=C,
  showstringspaces=false,
  basicstyle=\footnotesize\ttfamily,
  keywordstyle=\bfseries\color{green!40!black},
  commentstyle=\itshape\color{purple!40!black},
  identifierstyle=\color{blue},
  stringstyle=\color{orange},
}

\lstset{escapechar=£,style=customc}

\usepackage{algorithm2e}
\usepackage{pdfpages}
\usepackage{gensymb} % for \degree
%\usepackage{setspace}

\usepackage{caption}
\usepackage{booktabs}
\usepackage{multirow}
\newcommand{\MR}[2]{\multirow{#1}{*}{#2}}			%simplifies multirow command - must use the multirow package
\newcommand{\MC}[2]{\multicolumn{#1}{c}{#2}}			%simplifies the multicollumn command - must use the multicollumn package




