\subsubsection{native}
{\footnotesize
\begin{verbatim}
hugues@W520:~$ git clone --recursive https://github.com/RIOT-OS/Tutorials
Clonage dans 'Tutorials'...
remote: Enumerating objects: 8, done.
remote: Counting objects: 100% (8/8), done.
remote: Compressing objects: 100% (8/8), done.
remote: Total 637 (delta 0), reused 2 (delta 0), pack-reused 629
Réception d'objets: 100% (637/637), 2.37 MiB | 1.87 MiB/s, fait.
Résolution des deltas: 100% (357/357), fait.
Sous-module 'RIOT' (https://github.com/RIOT-OS/RIOT.git) enregistré pour le chemin 'RIOT'
Clonage dans '/home/hugues/Tutorials/RIOT'...
remote: Enumerating objects: 1, done.        
remote: Counting objects: 100% (1/1), done.        
remote: Total 222536 (delta 0), reused 0 (delta 0), pack-reused 222535        
Réception d'objets: 100% (222536/222536), 80.89 MiB | 2.07 MiB/s, fait.
Résolution des deltas: 100% (145265/145265), fait.
Chemin de sous-module 'RIOT' : '673187e29c06cdba74f0b82c7b30b9de2538531f' extrait
hugues@W520:~$ cd Tutorials/
hugues@W520:~/Tutorials$ ls
overview-net.png  phytec.png  README.md  RIOT  SAM-R21.jpg  slides  task-01  task-02
task-03  task-04  task-05  task-06  task-07  task-08  task-09  Vagrantfile
hugues@W520:~/Tutorials$ cd task-01/
hugues@W520:~/Tutorials/task-01$ make all term
Building application "Task01" for "native" with MCU "native".

"make" -C /home/hugues/Tutorials/RIOT/boards/native
"make" -C /home/hugues/Tutorials/RIOT/boards/native/drivers
"make" -C /home/hugues/Tutorials/RIOT/core
"make" -C /home/hugues/Tutorials/RIOT/cpu/native
"make" -C /home/hugues/Tutorials/RIOT/cpu/native/periph
"make" -C /home/hugues/Tutorials/RIOT/cpu/native/stdio_native
"make" -C /home/hugues/Tutorials/RIOT/cpu/native/vfs
"make" -C /home/hugues/Tutorials/RIOT/drivers
"make" -C /home/hugues/Tutorials/RIOT/drivers/periph_common
"make" -C /home/hugues/Tutorials/RIOT/sys
"make" -C /home/hugues/Tutorials/RIOT/sys/auto_init
"make" -C /home/hugues/Tutorials/RIOT/sys/ps
"make" -C /home/hugues/Tutorials/RIOT/sys/shell
"make" -C /home/hugues/Tutorials/RIOT/sys/shell/commands
   text	   data	    bss	    dec	    hex	filename
  28944	    728	  47728	  77400	  12e58	/home/hugues/Tutorials/task-01/bin/native/Task01.elf
/home/hugues/Tutorials/task-01/bin/native/Task01.elf  
RIOT native interrupts/signals initialized.
LED_RED_OFF
LED_GREEN_ON
RIOT native board initialized.
RIOT native hardware initialization complete.

main(): This is RIOT! (Version: 2020.04)
This is Task-01
> ps
ps
	pid | name                 | state    Q | pri | stack  ( used) | base addr  | current     
	  - | isr_stack            | -        - |   - |   8192 (   -1) | 0x56648400 | 0x56648400
	  1 | idle                 | pending  Q |  15 |   8192 (  436) | 0x56646120 | 0x56647f80 
	  2 | main                 | running  Q |   7 |  12288 ( 3148) | 0x56643120 | 0x56645f80 
	    | SUM                  |            |     |  28672 ( 3584)
> ^C
native: exiting
\end{verbatim}
}

Jusque-là rien de bien nouveau.

\subsubsection{Nucleo-144}

Il s'agit maintenant de déterminer la variable BOARD
pour notre matériel :
{\footnotesize
\begin{verbatim}
hugues@W520:~/Tutorials/task-01$ ls ../RIOT/boards/
acd52832                   iotlab-m3            openlabs-kw41z-mini-256kib
adafruit-clue              Kconfig              openmote-b
airfy-beacon               limifrog-v1          openmote-cc2538
arduino-due                lobaro-lorabox       particle-argon
arduino-duemilanove        lsn50                particle-boron
arduino-leonardo           maple-mini           particle-xenon
arduino-mega2560           mbed_lpc1768         pba-d-01-kw2x
arduino-mkr1000            mcb2388              phynode-kw41z
arduino-mkrfox1200         mega-xplained        pic32-clicker
arduino-mkrwan1300         microbit             pic32-wifire
arduino-mkrzero            microduino-corerf    pinetime
arduino-nano               msb-430              p-l496g-cell02
arduino-nano-33-ble        msb-430h             p-nucleo-wb55
arduino-uno                msba2                pyboard
arduino-zero               msbiot               README.md
atmega1284p                mulle                reel
atmega256rfr2-xpro         native               remote-pa
atmega328p                 nrf51dk              remote-reva
avr-rss2                   nrf51dongle          remote-revb
avsextrem                  nrf52832-mdk         ruuvitag
b-l072z-lrwan1             nrf52840dk           samd21-xpro
b-l475e-iot01a             nrf52840-mdk         same54-xpro
blackpill                  nrf52dk              saml10-xpro
blackpill-128kib           nrf6310              saml11-xpro
bluepill                   nucleo-f030r8        saml21-xpro
bluepill-128kib            nucleo-f031k6        samr21-xpro
calliope-mini              nucleo-f042k6        samr30-xpro
cc1312-launchpad           nucleo-f070rb        samr34-xpro
cc1352-launchpad           nucleo-f072rb        seeeduino_arch-pro
cc2538dk                   nucleo-f091rc        sensebox_samd21
cc2650-launchpad           nucleo-f103rb        slstk3401a
cc2650stk                  nucleo-f207zg        slstk3402a
chronos                    nucleo-f302r8        sltb001a
common                     nucleo-f303k8        slwstk6000b-slwrb4150a
derfmega128                nucleo-f303re        slwstk6000b-slwrb4162a
derfmega256                nucleo-f303ze        slwstk6220a
doc.txt                    nucleo-f334r8        sodaq-autonomo
ek-lm4f120xl               nucleo-f401re        sodaq-explorer
esp32-heltec-lora32-v2     nucleo-f410rb        sodaq-one
esp32-mh-et-live-minikit   nucleo-f411re        sodaq-sara-aff
esp32-olimex-evb           nucleo-f412zg        spark-core
esp32-ttgo-t-beam          nucleo-f413zh        stk3600
esp32-wemos-lolin-d32-pro  nucleo-f429zi        stk3700
esp32-wroom-32             nucleo-f446re        stm32f030f4-demo
esp32-wrover-kit           nucleo-f446ze        stm32f0discovery
esp8266-esp-12x            nucleo-f722ze        stm32f3discovery
esp8266-olimex-mod         nucleo-f746zg        stm32f429i-disc1
esp8266-sparkfun-thing     nucleo-f767zi        stm32f4discovery
f4vi1                      nucleo-l031k6        stm32f723e-disco
feather-m0                 nucleo-l053r8        stm32f769i-disco
feather-nrf52840           nucleo-l073rz        stm32l0538-disco
firefly                    nucleo-l152re        stm32l476g-disco
fox                        nucleo-l412kb        teensy31
frdm-k22f                  nucleo-l432kc        telosb
frdm-k64f                  nucleo-l433rc        thingy52
frdm-kw41z                 nucleo-l452re        ublox-c030-u201
hamilton                   nucleo-l476rg        udoo
hifive1                    nucleo-l496zg        usb-kw41z
hifive1b                   nucleo-l4r5zi        waspmote-pro
ikea-tradfri               nz32-sc151           wsn430-v1_3b
im880b                     olimexino-stm32      wsn430-v1_4
i-nucleo-lrwan1            opencm904            yunjia-nrf51822
iotlab-a8-m3               openlabs-kw41z-mini  z1
\end{verbatim}
}
Pour nous ce sera donc \enquote{nucleo-f746zg}.\\

Je connecte la carte à mon PC (coté st-link) et c'est parti :
{\scriptsize
\begin{verbatim}
hugues@W520:~/Tutorials/task-01$ BOARD=nucleo-f746zg make all flash term
Building application "Task01" for "nucleo-f746zg" with MCU "stm32f7".

"make" -C /home/hugues/Tutorials/RIOT/boards/nucleo-f746zg
"make" -C /home/hugues/Tutorials/RIOT/boards/common/nucleo
"make" -C /home/hugues/Tutorials/RIOT/core
"make" -C /home/hugues/Tutorials/RIOT/cpu/stm32f7
"make" -C /home/hugues/Tutorials/RIOT/cpu/cortexm_common
"make" -C /home/hugues/Tutorials/RIOT/cpu/cortexm_common/periph
"make" -C /home/hugues/Tutorials/RIOT/cpu/stm32_common
"make" -C /home/hugues/Tutorials/RIOT/cpu/stm32_common/periph
"make" -C /home/hugues/Tutorials/RIOT/cpu/stm32f7/periph
"make" -C /home/hugues/Tutorials/RIOT/drivers
"make" -C /home/hugues/Tutorials/RIOT/drivers/periph_common
"make" -C /home/hugues/Tutorials/RIOT/sys
"make" -C /home/hugues/Tutorials/RIOT/sys/auto_init
"make" -C /home/hugues/Tutorials/RIOT/sys/isrpipe
"make" -C /home/hugues/Tutorials/RIOT/sys/newlib_syscalls_default
"make" -C /home/hugues/Tutorials/RIOT/sys/pm_layered
"make" -C /home/hugues/Tutorials/RIOT/sys/ps
"make" -C /home/hugues/Tutorials/RIOT/sys/shell
"make" -C /home/hugues/Tutorials/RIOT/sys/shell/commands
"make" -C /home/hugues/Tutorials/RIOT/sys/stdio_uart
"make" -C /home/hugues/Tutorials/RIOT/sys/tsrb
   text	   data	    bss	    dec	    hex	filename
  14940	    132	   2632	  17704	   4528	/home/hugues/Tutorials/task-01/bin/nucleo-f746zg/Task01.elf
/home/hugues/Tutorials/RIOT/dist/tools/openocd/openocd.sh flash /home/hugues/Tutorials/task-01/bin/nucleo-f746zg/Task01.elf
### Flashing Target ###
Open On-Chip Debugger 0.10.0
Licensed under GNU GPL v2
For bug reports, read
	http://openocd.org/doc/doxygen/bugs.html
hla_swd
Info : The selected transport took over low-level target control. The results might differ compared to plain JTAG/SWD
adapter speed: 2000 kHz
adapter_nsrst_delay: 100
srst_only separate srst_nogate srst_open_drain connect_deassert_srst
srst_only separate srst_nogate srst_open_drain connect_deassert_srst
srst_only separate srst_nogate srst_open_drain connect_assert_srst
Info : Unable to match requested speed 2000 kHz, using 1800 kHz
Info : Unable to match requested speed 2000 kHz, using 1800 kHz
Info : clock speed 1800 kHz
Info : STLINK v2 JTAG v29 API v2 SWIM v18 VID 0x0483 PID 0x374B
Info : using stlink api v2
Info : Target voltage: 3.247431
Warn : Silicon bug: single stepping will enter pending exception handler!
Info : stm32f7x.cpu: hardware has 8 breakpoints, 4 watchpoints
    TargetName         Type       Endian TapName            State       
--  ------------------ ---------- ------ ------------------ ------------
 0* stm32f7x.cpu       hla_target little stm32f7x.cpu       reset
target halted due to debug-request, current mode: Thread 
xPSR: 0x01000000 pc: 0x0800a83c msp: 0x20003410
auto erase enabled
Info : device id = 0x10016449
Info : flash size = 1024kbytes
target halted due to breakpoint, current mode: Thread 
xPSR: 0x61000000 pc: 0x20000046 msp: 0x20003410
wrote 32768 bytes from file /home/hugues/Tutorials/task-01/bin/nucleo-f746zg/Task01.elf in 0.811100s (39.453 KiB/s)
target halted due to breakpoint, current mode: Thread 
xPSR: 0x61000000 pc: 0x2000002e msp: 0x20003410
target halted due to breakpoint, current mode: Thread 
xPSR: 0x61000000 pc: 0x2000002e msp: 0x20003410
verified 15072 bytes in 0.128518s (114.527 KiB/s)
shutdown command invoked
Done flashing
/home/hugues/Tutorials/RIOT/dist/tools/pyterm/pyterm -p "/dev/ttyACM0" -b "115200" 
Twisted not available, please install it if you want to use pyterm's JSON capabilities
2020-05-04 17:36:35,989 # Connect to serial port /dev/ttyACM0
Welcome to pyterm!
Type '/exit' to exit.
ps
2020-05-04 17:36:50,692 # ps
2020-05-04 17:36:50,700 # 	pid | name                 | state    Q | pri | stack  ( used) | base addr  | current     
2020-05-04 17:36:50,708 # 	  - | isr_stack            | -        - |   - |    512 (  116) | 0x20000000 | 0x200001c8
2020-05-04 17:36:50,716 # 	  1 | idle                 | pending  Q |  15 |    256 (  132) | 0x200003c0 | 0x2000043c 
2020-05-04 17:36:50,724 # 	  2 | main                 | running  Q |   7 |   1536 (  652) | 0x200004c0 | 0x20000924 
2020-05-04 17:36:50,729 # 	    | SUM                  |            |     |   2304 (  900)
> reboot
2020-05-04 17:41:10,933 #  reboot
2020-05-04 17:41:10,940 # main(): This is RIOT! (Version: 2020.04)
2020-05-04 17:41:10,941 # This is Task-01
> /exit
2020-05-04 17:46:17,710 # Exiting Pyterm
hugues@W520:~/Tutorials/task-01$ /home/hugues/Tutorials/RIOT/dist/tools/pyterm/pyterm -p "/dev/ttyACM0" -b "115200"
Twisted not available, please install it if you want to use pyterm's JSON capabilities
2020-05-04 17:48:12,032 # Connect to serial port /dev/ttyACM0
Welcome to pyterm!
Type '/exit' to exit.
> /exit
2020-05-04 17:48:33,838 # Exiting Pyterm
hugues@W520:~/Tutorials/task-01$
\end{verbatim}
}
Que dire ? Ça fonctionne tout seul !\\

\subsubsection{bluepill (STM32F103C8T6)}

Cette carte très bon marché (encore moins cher qu'un nano atmega328p) pourrait être
utilisée dans une version mono-bras du convertisseur. J'envisage de démarrer le
développement du nouveau firmware par le remplacement du nano par cette carte sur
un prototype OwnWall dont je dispose en attendant le premier prototype OwnTech, le
CAN (MCP3208) étant identique à ceux qui seront utilisé.

{\scriptsize
\begin{verbatim}
### Flashing Target ###
Open On-Chip Debugger 0.10.0
Licensed under GNU GPL v2
For bug reports, read
	http://openocd.org/doc/doxygen/bugs.html
hla_swd
srst_only separate srst_nogate srst_open_drain connect_assert_srst
Info : The selected transport took over low-level target control. The results might differ compared to plain JTAG/SWD
adapter speed: 1000 kHz
adapter_nsrst_delay: 100
srst_only separate srst_nogate srst_open_drain connect_assert_srst
srst_only separate srst_nogate srst_open_drain connect_assert_srst
Info : Unable to match requested speed 1000 kHz, using 950 kHz
Info : Unable to match requested speed 1000 kHz, using 950 kHz
Info : clock speed 950 kHz
Error: open failed
in procedure 'init' 
in procedure 'ocd_bouncer'

make: *** [/home/hugues/Tutorials/task-01/../RIOT/Makefile.include:667: flash] Error 1
\end{verbatim}
}
Ça n'a pas fonctionné, peut-être à cause du clône de st-link v2 Chinois à 2\euro...\\

Chaque dossier dans RIOT/boards contient un fichier doc.txt, pour la bluepill j'y
trouve l'info suivante :
\begin{verbatim}
## Flashing

To program and debug the board you need a SWD capable debugger. The
easiest way is using [OpenOCD][OpenOCD]. By default RIOT uses the hardware
reset signal and connects to the chip under reset for flashing. This is
required to reliably connect to the device even when the MCU is in a low power
mode. Therefore not only SWDIO and SWCLK, but also the RST pin of your
debugger need to be connected to the board.
\end{verbatim}

J'ai donc commencé par ajouter le fil de reset manquant dans mon câblage, mais ça n'a
pas résolu le problème.\\

Je me suis donc penché sur OpenOCD en essayant de le lancer directement :
{\scriptsize
\begin{verbatim}
hugues@W520:~/Tutorials/task-01$ openocd -f /usr/share/openocd/scripts/interface/stlink-v2.cfg \
-f /usr/share/openocd/scripts/target/stm32f1x.cfg 
Open On-Chip Debugger 0.10.0
Licensed under GNU GPL v2
For bug reports, read
	http://openocd.org/doc/doxygen/bugs.html
Info : auto-selecting first available session transport "hla_swd". To override use 'transport select <transport>'.
Info : The selected transport took over low-level target control. The results might differ compared to plain JTAG/SWD
adapter speed: 1000 kHz
adapter_nsrst_delay: 100
none separate
Info : Unable to match requested speed 1000 kHz, using 950 kHz
Info : Unable to match requested speed 1000 kHz, using 950 kHz
Info : clock speed 950 kHz
Info : STLINK v2 JTAG v29 API v2 SWIM v7 VID 0x0483 PID 0x3748
Info : using stlink api v2
Info : Target voltage: 3.166154
Warn : UNEXPECTED idcode: 0x2ba01477
Error: expected 1 of 1: 0x1ba01477
in procedure 'init' 
in procedure 'ocd_bouncer'
\end{verbatim}
}
Ça n'a pas fonctionné non plus mais l'erreur est différente, le voyant du stlink est
passé du rouge au bleue et il semble que ma bluepill n'a pas le bon idcode. J'ai donc
remplacé 0x1ba01477 par 0x2ba01477 dans
\texttt{/usr/share/openocd/scripts/target/stm32f1x.cfg}, après quoi openocd semble se
lancer correctement.\\

J'obtiens malgré tout toujours la même erreur avec la commande make flash.\\

Dans le fichier \texttt{/RIOT/dist/tools/openocd/openocd.sh} j'observe que la
commande utilise la constante OPENOCD\_ADAPTER\_INIT, j'effectue donc une recherche
dans le code source :
{\scriptsize
\begin{verbatim}
hugues@W520:~/Tutorials/task-01$ grep -r OPENOCD_ADAPTER_INIT ../RIOT/*
../RIOT/dist/tools/openocd/openocd.sh:: ${OPENOCD_ADAPTER_INIT:=}
../RIOT/dist/tools/openocd/openocd.sh:            ${OPENOCD_ADAPTER_INIT} \
../RIOT/dist/tools/openocd/openocd.sh:            ${OPENOCD_ADAPTER_INIT} \
../RIOT/dist/tools/openocd/openocd.sh:            ${OPENOCD_ADAPTER_INIT} \
../RIOT/dist/tools/openocd/openocd.sh:            ${OPENOCD_ADAPTER_INIT} \
../RIOT/dist/tools/openocd/openocd.sh:            ${OPENOCD_ADAPTER_INIT} \
../RIOT/dist/tools/openocd/openocd.sh:            ${OPENOCD_ADAPTER_INIT} \
../RIOT/dist/tools/buildsystem_sanity_check/check.sh:UNEXPORTED_VARIABLES+=('OPENOCD_ADAPTER_INIT')
../RIOT/makefiles/tools/openocd.inc.mk:# Export OPENOCD_ADAPTER_INIT to required targets
../RIOT/makefiles/tools/openocd.inc.mk:$(call target-export-variables,$(OPENOCD_TARGETS),OPENOCD_ADAPTER_INIT)
../RIOT/makefiles/tools/openocd-adapters/stlink.inc.mk:OPENOCD_ADAPTER_INIT ?= \
../RIOT/makefiles/tools/openocd-adapters/stlink.inc.mk:  OPENOCD_ADAPTER_INIT += -c 'hla_serial $(DEBUG_ADAPTER_ID)'
../RIOT/makefiles/tools/openocd-adapters/iotlab.inc.mk:OPENOCD_ADAPTER_INIT ?= -c 'source [find interface/ftdi/iotlab-usb.cfg]'
../RIOT/makefiles/tools/openocd-adapters/iotlab.inc.mk:  OPENOCD_ADAPTER_INIT += -c 'ftdi_serial $(DEBUG_ADAPTER_ID)'
../RIOT/makefiles/tools/openocd-adapters/jlink.inc.mk:OPENOCD_ADAPTER_INIT ?= -c 'source [find interface/jlink.cfg]'
../RIOT/makefiles/tools/openocd-adapters/jlink.inc.mk:  OPENOCD_ADAPTER_INIT += -c 'jlink serial $(DEBUG_ADAPTER_ID)'
../RIOT/makefiles/tools/openocd-adapters/dap.inc.mk:OPENOCD_ADAPTER_INIT ?= -c 'source [find interface/cmsis-dap.cfg]'
../RIOT/makefiles/tools/openocd-adapters/dap.inc.mk:  OPENOCD_ADAPTER_INIT += -c 'cmsis_dap_serial $(DEBUG_ADAPTER_ID)'
../RIOT/makefiles/tools/openocd-adapters/sysfs_gpio.inc.mk:OPENOCD_ADAPTER_INIT ?= \
../RIOT/makefiles/tools/openocd-adapters/mulle.inc.mk:OPENOCD_ADAPTER_INIT ?= -f '$(RIOTBASE)/boards/mulle/dist/openocd/mulle-programmer-$(PROGRAMMER_VERSION).cfg'
../RIOT/makefiles/tools/openocd-adapters/mulle.inc.mk:  OPENOCD_ADAPTER_INIT += -c 'ftdi_serial $(DEBUG_ADAPTER_ID)'
../RIOT/makefiles/tools/openocd-adapters/raspi.inc.mk:OPENOCD_ADAPTER_INIT ?= \
hugues@W520:~/Tutorials/task-01$ cat ../RIOT/makefiles/tools/openocd-adapters/stlink.inc.mk
# ST-Link debug adapter
# Use st-link v2-1 by default
STLINK_VERSION ?= 2-1

# Use STLINK_VERSION to select which stlink version is used
OPENOCD_ADAPTER_INIT ?= \
  -c 'source [find interface/stlink-v$(STLINK_VERSION).cfg]' \
  -c 'transport select hla_swd'
# Add serial matching command, only if DEBUG_ADAPTER_ID was specified
ifneq (,$(DEBUG_ADAPTER_ID))
  OPENOCD_ADAPTER_INIT += -c 'hla_serial $(DEBUG_ADAPTER_ID)'
endif

# if no openocd specific configuration file, check for default locations:
# 1. Using the default dist/openocd.cfg (automatically set by openocd.sh)
# 2. Using the common cpu specific config file
ifeq (,$(OPENOCD_CONFIG))
  # if no openocd default configuration is provided by the board,
  # use the STM32 common one
  ifeq (0,$(words $(wildcard $(BOARDSDIR)/$(BOARD)/dist/openocd.cfg)))
    OPENOCD_CONFIG = $(RIOTBASE)/boards/common/stm32/dist/$(CPU).cfg
  endif
endif
\end{verbatim}
}
Je constate qu'il cherche à utilise la version 2-1 de l'adaptateur stlink qui est
adapté pour l'adaptateur intégré à la nucleo mais pas pour mon adaptateur à 2\euro
(v2 tout court).\\

Je retente donc en fixant cette constante dans ma commande make :
{\scriptsize
\begin{verbatim}
### Flashing Target ###
Open On-Chip Debugger 0.10.0
Licensed under GNU GPL v2
For bug reports, read
	http://openocd.org/doc/doxygen/bugs.html
hla_swd
srst_only separate srst_nogate srst_open_drain connect_assert_srst
Info : The selected transport took over low-level target control. The results might differ compared to plain JTAG/SWD
adapter speed: 1000 kHz
adapter_nsrst_delay: 100
srst_only separate srst_nogate srst_open_drain connect_assert_srst
srst_only separate srst_nogate srst_open_drain connect_assert_srst
Info : Unable to match requested speed 1000 kHz, using 950 kHz
Info : Unable to match requested speed 1000 kHz, using 950 kHz
Info : clock speed 950 kHz
Info : STLINK v2 JTAG v29 API v2 SWIM v7 VID 0x0483 PID 0x3748
Info : using stlink api v2
Info : Target voltage: 3.165663
Error: init mode failed (unable to connect to the target)
in procedure 'init' 
in procedure 'ocd_bouncer'

make: *** [/home/hugues/Tutorials/task-01/../RIOT/Makefile.include:667: flash] Error 1
\end{verbatim}
}
Cette fois il semble reconnaître convenablement mon stlink mais n'arrive pas à se
connecter au stm32, je fais un essai avec un reset manuel :
{\scriptsize
\begin{verbatim}
### Flashing Target ###
Open On-Chip Debugger 0.10.0
Licensed under GNU GPL v2
For bug reports, read
	http://openocd.org/doc/doxygen/bugs.html
hla_swd
srst_only separate srst_nogate srst_open_drain connect_assert_srst
Info : The selected transport took over low-level target control. The results might differ compared to plain JTAG/SWD
adapter speed: 1000 kHz
adapter_nsrst_delay: 100
srst_only separate srst_nogate srst_open_drain connect_assert_srst
srst_only separate srst_nogate srst_open_drain connect_assert_srst
Info : Unable to match requested speed 1000 kHz, using 950 kHz
Info : Unable to match requested speed 1000 kHz, using 950 kHz
Info : clock speed 950 kHz
Info : STLINK v2 JTAG v29 API v2 SWIM v7 VID 0x0483 PID 0x3748
Info : using stlink api v2
Info : Target voltage: 3.159513
Info : STM32F103C8Tx.cpu: hardware has 6 breakpoints, 4 watchpoints
    TargetName         Type       Endian TapName            State       
--  ------------------ ---------- ------ ------------------ ------------
 0* STM32F103C8Tx.cpu  hla_target little STM32F103C8Tx.cpu  reset
target halted due to debug-request, current mode: Thread 
xPSR: 0x01000000 pc: 0x08000450 msp: 0x20000200
auto erase enabled
Info : device id = 0x20036410
Info : flash size = 64kbytes
target halted due to breakpoint, current mode: Thread 
xPSR: 0x61000000 pc: 0x2000003a msp: 0x20000200
wrote 14336 bytes from file /home/hugues/Tutorials/task-01/bin/bluepill/Task01.elf in 0.483442s (28.959 KiB/s)
target halted due to breakpoint, current mode: Thread 
xPSR: 0x61000000 pc: 0x2000002e msp: 0x20000200
target halted due to breakpoint, current mode: Thread 
xPSR: 0x61000000 pc: 0x2000002e msp: 0x20000200
verified 14336 bytes in 0.249575s (56.095 KiB/s)
target halted due to breakpoint, current mode: Thread 
xPSR: 0x61000000 pc: 0x2000002e msp: 0x20000200
shutdown command invoked
Done flashing
\end{verbatim}
}
Après plusieurs essais je parviens à relâcher le reset au bon moment et ça flash. Le
pin reset de ce st-link ne semble donc pas fonctionner. Après démontage il est
seulement connecté à une résistance de pull-up.\\

On peut trouver le schéma du st-link/v2-1 dans le manuel de la nucleo-144, et on y
voit que le reset est sensé être connecté à la pin 18 du microcontrôleur interne ce
qui n'est pas le cas. Deux coups de fer à souder plus tard (j'ai vite regretté de
m'être lancé dans une soudure sur un pas de 0,5mm en télétravail, avec une panne de
0,5 et de la soudure de 0,7 sans flux et sans loupe) j'ai pu de nouveau
tester : cette fois c'est passé tout seul.\\

Nouveau problème avec pyterm cette fois : \texttt{No such file or directory:
'/dev/ttyACM0'}. Ce stlink n'intègre pas d'adaptateur série mais j'ai un dongle
ch340 que je peux utiliser pour ça. Une petite recherche m'indique : \enquote{The
default UART port used is UART2, which uses pins A2 (TX) and A3 (RX).}.\\

Le CH340 n'est pas attaché à /dev/ttyACM0 comme le st-link v2-1 de la nucleo mais à
/dev/ttyUSB0. Il faut donc ajouter PORT\_LINUX=/dev/ttyUSB0 pour la commande make ce
qui nous donne :
{\scriptsize
\begin{verbatim}
hugues@W520:~/Tutorials/task-01$ BOARD=bluepill STLINK_VERSION=2 PORT_LINUX=/dev/ttyUSB0 make all flash term
\end{verbatim}
}
Bien sûr on peut mettre tout ça dans le Makefile :
\begin{lstlisting}[language=make]
# If no BOARD is found in the environment, use this default:
BOARD ?= bluepill

# To use chinese st-link v2 and ch340 dongle with bluepill
ifeq ($(BOARD),bluepill)
	STLINK_VERSION=2
	PORT_LINUX=/dev/ttyUSB0
endif
\end{lstlisting}

Ainsi je n'aurai plus qu'à tapper la commande \texttt{make all flash term} pour
utiliser la bluepill par défaut (je me suis fais gronder d'avoir emprunté le câble
de l'appareil photo de ma femme pour relier la nucleo) en attendant de m'en
procurer un.\\

Voilà ce que ça donne au niveau de la pile :
{\scriptsize
\begin{verbatim}
2020-05-06 12:20:49,693 # 	pid | name                 | state    Q | pri | stack  ( used) | base addr  | current     
2020-05-06 12:20:49,701 # 	  - | isr_stack            | -        - |   - |    512 (  116) | 0x20000000 | 0x200001c8
2020-05-06 12:20:49,709 # 	  1 | idle                 | pending  Q |  15 |    256 (  132) | 0x200003c0 | 0x2000043c 
2020-05-06 12:20:49,718 # 	  2 | main                 | running  Q |   7 |   1536 (  660) | 0x200004c0 | 0x20000924 
2020-05-06 12:20:49,724 # 	    | SUM                  |            |     |   2304 (  908)
\end{verbatim}
}

\subsubsection{arduino-nano (ATmega328p)}

Disons \enquote{pour le fun}, et pour voir ce que ça donne sur un $\mu$C
8 bits avec seulement 2Ko de RAM.\\

La commande toujours aussi simple :
{\scriptsize
\begin{verbatim}
hugues@W520:~/Tutorials/task-01$ BOARD=arduino-nano make all flash term
\end{verbatim}
}
Et le résultat :
{\scriptsize
\begin{verbatim}
   text	   data	    bss	    dec	    hex	filename
   7094	    852	    961	   8907	   22cb	/home/hugues/Tutorials/task-01/bin/arduino-nano/Task01.elf
...
avrdude: writing flash (7946 bytes)
...
2020-05-06 12:35:58,933 # 	pid | name                 | state    Q | pri | stack  ( used) | base addr  | current     
2020-05-06 12:35:59,032 # 	  1 | idle                 | pending  Q |  15 |    128 (   86) |      0x454 |      0x481 
2020-05-06 12:35:59,132 # 	  2 | main                 | running  Q |   7 |    640 (  306) |      0x4d4 |      0x64b 
2020-05-06 12:35:59,193 # 	    | SUM                  |            |     |    768 (  392)
\end{verbatim}
}
L'utilisation mémoire (data+bss) est importante 1813 sur 2048 mais ça
passe...
