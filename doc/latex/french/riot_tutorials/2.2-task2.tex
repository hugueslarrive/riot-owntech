Il s'agit d'écrire 2 gestionnaires de commande shell :
\begin{itemize}
	\item une commande echo ;
	\item une commande toggle pour une led.\\
\end{itemize}

\subsubsection{stm32}
Lors de la compilation j'ai recontré le problème suivant :
{\scriptsize
\begin{verbatim}
hugues@W520:~/Tutorials/task-02$ make all flash term
Building application "Task02" for "bluepill" with MCU "stm32f1".

In file included from /home/hugues/Tutorials/RIOT/boards/bluepill/include/board.h:32:0,
                 from /home/hugues/Tutorials/RIOT/drivers/include/led.h:35,
                 from /home/hugues/Tutorials/task-02/main.c:5:
/home/hugues/Tutorials/task-02/main.c: In function 'toggle':
/home/hugues/Tutorials/RIOT/boards/common/blxxxpill/include/board_common.h:36:29: error: 'GPIOC' undeclared (first u
se in this function)
 #define LED0_PORT           GPIOC                                   /**< GPIO-Port the LED is connected to */
                             ^
/home/hugues/Tutorials/RIOT/boards/common/blxxxpill/include/board_common.h:49:30: note: in expansion of macro 'LED0_
PORT'
 #define LED0_TOGGLE         (LED0_PORT->ODR  ^= LED0_MASK)          /**< Toggle LED0 */
                              ^~~~~~~~~
/home/hugues/Tutorials/task-02/main.c:33:2: note: in expansion of macro 'LED0_TOGGLE'
  LED0_TOGGLE;
  ^~~~~~~~~~~
/home/hugues/Tutorials/RIOT/boards/common/blxxxpill/include/board_common.h:36:29: note: each undeclared identifier i
s reported only once for each function it appears in
 #define LED0_PORT           GPIOC                                   /**< GPIO-Port the LED is connected to */
                             ^
/home/hugues/Tutorials/RIOT/boards/common/blxxxpill/include/board_common.h:49:30: note: in expansion of macro 'LED0_
PORT'
 #define LED0_TOGGLE         (LED0_PORT->ODR  ^= LED0_MASK)          /**< Toggle LED0 */
                              ^~~~~~~~~
/home/hugues/Tutorials/task-02/main.c:33:2: note: in expansion of macro 'LED0_TOGGLE'
  LED0_TOGGLE;
  ^~~~~~~~~~~
make[1]: *** [/home/hugues/Tutorials/RIOT/Makefile.base:110: /home/hugues/Tutorials/task-02/bin/bluepill/application
_Task01/main.o] Error 1
make: *** [/home/hugues/Tutorials/task-02/../RIOT/Makefile.include:538: /home/hugues/Tutorials/task-02/bin/bluepill/
application_Task01.a] Error 2
\end{verbatim}
}
Le problème ne se produit qu'avec BOARD=bluepill, \texttt{make all}
fonctionne bien avec \enquote{native}, \enquote{nucleo-f746zg}, ou même
\enquote{nucleo-f103rb}.\\

Le problème est peut-être lié à l'ordre de traitement des dépendances
 : \url{https://github.com/RIOT-OS/RIOT/issues/9913}.\\
 
J'ai pu le contourner par une inclusion conditionnelle :
\begin{lstlisting}
#ifdef BOARD_BLUEPILL
#include "vendor/stm32f103xe.h"
#endif
\end{lstlisting}

\subsubsection{main.c}
Voilà ce que donne le fichier main.c :
\begin{lstlisting}
#include <stdio.h>
#include <string.h>

#include "shell.h"
#include "led.h"

#ifdef BOARD_BLUEPILL
#include "vendor/stm32f103xe.h"
#endif

int echo(int argc, char **argv)
{
    /* Print a line of text */
    (void)argc;
    (void)argv;
    
    if (argc > 1) {
    	for (int i = 1; i < argc-1; i++) {
	        printf("%s ", argv[i]);
	    }
		printf("%s\n", argv[argc-1]);
    }
    else {
		printf("\n");
	}
	
    return 0;
}

int toggle(int argc, char **argv)
{
	/** Toggles the primary LED on the board **/
    (void)argc;
    (void)argv;

	LED0_TOGGLE;
	
	return 0;
}

static const shell_command_t commands[] = {
    { "echo", "Print a line of text", echo },
    { "toggle", "Toggles the primary LED on the board", toggle },
    { NULL, NULL, NULL }
};

int main(void)
{
    puts("This is Task-02");

    char line_buf[SHELL_DEFAULT_BUFSIZE];
    shell_run(commands, line_buf, SHELL_DEFAULT_BUFSIZE);

    return 0;
}
\end{lstlisting}

\subsubsection{arduino-nano}
Sur l'arduino ça ne passe plus :
{\scriptsize
\begin{verbatim}
/usr/lib/gcc/avr/5.4.0/../../../avr/bin/ld : l'adresse 0x800871 de /home/hugues/Tutorials/task-02/bin/arduino-nano
/Task02.elf de la section «.bss» n'est pas dans la région «data»
\end{verbatim}
}
Pour que ça passe on peut réduire la valeur de
\texttt{THREAD\_STACKSIZE\_DEFAULT} dans le fichier
\texttt{RIOT/cpu/atmega\_common/include/cpu\_conf.h} de 512 à 256 :
{\scriptsize
\begin{verbatim}
2020-05-06 15:24:43,923 # 	pid | name                 | state    Q | pri | stack  ( used) | base addr  | current     
2020-05-06 15:24:44,023 # 	  1 | idle                 | pending  Q |  15 |    128 (   84) |      0x4b2 |      0x4df 
2020-05-06 15:24:44,123 # 	  2 | main                 | running  Q |   7 |    384 (  336) |      0x532 |      0x5a9 
2020-05-06 15:24:44,166 # 	    | SUM                  |            |     |    512 (  420)
\end{verbatim}
}
La commande \texttt{toggle} ne fonctionne pas.\\

Il y avait une erreur dans le fichier :\\
\texttt{RIOT/boards/common/arduino-atmega/include/board\_common.h}.\\

Le patch pour que ça fonctionne sur le nano :
\begin{lstlisting}
--- ../RIOT/boards/common/arduino-atmega/include/board_common.h	2020-04-27 21:01:08.376660474 +0200
+++ RIOT/boards/common/arduino-atmega/include/board_common.h	2020-05-06 16:14:12.768798005 +0200
@@ -76,9 +76,9 @@
 #define LED2_ON             (PORTD &= ~LED2_MASK)
 #define LED2_TOGGLE         (PORTD ^=  LED2_MASK)
 #else
-#define LED0_ON             (PORTD |=  LED0_MASK)
-#define LED0_OFF            (PORTD &= ~LED0_MASK)
-#define LED0_TOGGLE         (PORTD ^=  LED0_MASK)
+#define LED0_ON             (PORTB |=  LED0_MASK)
+#define LED0_OFF            (PORTB &= ~LED0_MASK)
+#define LED0_TOGGLE         (PORTB ^=  LED0_MASK)
 #endif
 /** @} */
\end{lstlisting}
