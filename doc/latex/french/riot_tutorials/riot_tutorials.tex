\documentclass[a4paper,12pt, twoside]{article}

\usepackage[T1]{fontenc}
\usepackage[utf8]{inputenc}
%~ \usepackage{lmodern}
\usepackage{helvet}
\renewcommand{\familydefault}{\sfdefault}
\usepackage[francais]{babel}
\usepackage{verbatim}
\usepackage{tikz}
\usepackage{eurosym} % pour € \euro{}
\usepackage{csquotes} % pour les guillemets

%\usepackage{newtxmath}

\usepackage[
  % Some remarks:
  % * drivers like 'pdftex' that can be detected automatically
  %   are not necessary
  % * breaklinks is rather an internal option.
  %   If a driver does not support it, then forcing the option
  %   let the text break across lines, but also the link
  %   areas are "broken". If the driver supports the option,
  %   then the option is enabled anyway.
  % * Information entries should be set outside,
  %   because LaTeX expands the package options,
  %   hyperref does not like them, if they are
  %   prematurely expanded.
  % * Hyperref has a new option for hiding links: hidelinks
  hidelinks,
  pagebackref,
  bookmarksopen,
  bookmarksnumbered,
  a4paper,
]{hyperref}
\hypersetup{
  pdfauthor={Hugues Larrive},
  % ...
}
% Adding package bookmark improves bookmarks handling.
% More features and faster updated bookmarks.
\usepackage{bookmark}

\usepackage{layout}
\oddsidemargin=-1cm
\usepackage[top=2cm, bottom=2cm, left=2cm, right=2cm]{geometry}

\usepackage{wrapfig}
\usepackage{graphicx}


\usepackage{listings}
\lstset{
     literate=%
		{à}{{\`a}}1
		{é}{{\'e}}1
		{ê}{{\^e}}1
		{è}{{\`e}}1
		{É}{{\'E}}1
		{ç}{{\c{c}}}1
}
\usepackage{color}

\definecolor{mygreen}{rgb}{0,0.6,0}
\definecolor{mygray}{rgb}{0.5,0.5,0.5}
\definecolor{mymauve}{rgb}{0.58,0,0.82}

\lstset{ 
  backgroundcolor=\color{white},   % choose the background color; you must add \usepackage{color} or \usepackage{xcolor}; should come as last argument
  basicstyle=\footnotesize,        % the size of the fonts that are used for the code
  breakatwhitespace=false,         % sets if automatic breaks should only happen at whitespace
  breaklines=true,                 % sets automatic line breaking
  captionpos=b,                    % sets the caption-position to bottom
  commentstyle=\color{mygreen},    % comment style
  deletekeywords={...},            % if you want to delete keywords from the given language
  escapeinside={\%*}{*)},          % if you want to add LaTeX within your code
  extendedchars=true,              % lets you use non-ASCII characters; for 8-bits encodings only, does not work with UTF-8
  frame=single,	                   % adds a frame around the code
  keepspaces=true,                 % keeps spaces in text, useful for keeping indentation of code (possibly needs columns=flexible)
  keywordstyle=\color{blue},       % keyword style
  language=Octave,                 % the language of the code
  morekeywords={*,...},            % if you want to add more keywords to the set
  numbers=left,                    % where to put the line-numbers; possible values are (none, left, right)
  numbersep=5pt,                   % how far the line-numbers are from the code
  numberstyle=\tiny\color{mygray}, % the style that is used for the line-numbers
  rulecolor=\color{black},         % if not set, the frame-color may be changed on line-breaks within not-black text (e.g. comments (green here))
  showspaces=false,                % show spaces everywhere adding particular underscores; it overrides 'showstringspaces'
  showstringspaces=false,          % underline spaces within strings only
  showtabs=false,                  % show tabs within strings adding particular underscores
  stepnumber=1,                    % the step between two line-numbers. If it's 1, each line will be numbered
  stringstyle=\color{mymauve},     % string literal style
  tabsize=2%,	                   % sets default tabsize to 2 spaces
  %title=\lstname                   % show the filename of files included with \lstinputlisting; also try caption instead of title
}

\lstdefinestyle{customc}{
  belowcaptionskip=1\baselineskip,
  breaklines=true,
  frame=L,
  xleftmargin=\parindent,
  language=C,
  showstringspaces=false,
  basicstyle=\footnotesize\ttfamily,
  keywordstyle=\bfseries\color{green!40!black},
  commentstyle=\itshape\color{purple!40!black},
  identifierstyle=\color{blue},
  stringstyle=\color{orange},
}

\lstset{escapechar=£,style=customc}

\usepackage{algorithm2e}
\usepackage{pdfpages}
\usepackage{gensymb} % for \degree
%\usepackage{setspace}

\usepackage{caption}
\usepackage{booktabs}
\usepackage{multirow}
\newcommand{\MR}[2]{\multirow{#1}{*}{#2}}			%simplifies multirow command - must use the multirow package
\newcommand{\MC}[2]{\multicolumn{#1}{c}{#2}}			%simplifies the multicollumn command - must use the multicollumn package





\usepackage{amssymb}

%\usepackage[firstpage]{draftwatermark}

\usepackage{alltt}

\usepackage{pifont}% http://ctan.org/pkg/pifont
\newcommand{\cm}{\textcolor{green}{\ding{51}}}%
\newcommand{\xm}{\textcolor{red}{\ding{55}}}%


\title{Notes sur les tutoriels de RIOT}
\author{Hugues Larrive <hlarrive@laas.fr>}
\date{\today}	%defines the date of the document - leave empty for no date


\begin{document}

\maketitle{}

\vspace*{\stretch{1}}

\begin{abstract}
    Voilà mes notes personnelles sur les tutos de riot os :
    \url{https://github.com/RIOT-OS/Tutorials}. Seuls les six
    premiers sont traités.\\
    
    J'ai testé les quatre premiers sur plusieurs cartes, ce qui m'a
    permis d'approfondir certains mécanismes du système comme la gestion
    de la mémoire et le système de compilation.
\end{abstract}

\vspace*{\stretch{1}}

{\footnotesize
\begin{verbatim}
    Author: Hugues Larrive <hugues.larrive@laas.fr>
    
	Copyright (C) 2020 LAAS-CNRS.
	Permission is granted to copy, distribute and/or modify this document
	under the terms of the GNU Free Documentation License, Version 1.3
	or any later version published by the Free Software Foundation;
	with no Invariant Sections, no Front-Cover Texts, and no Back-Cover Texts.
	A copy of the license is included in the section entitled "GNU
	Free Documentation License".
\end{verbatim}
}

\newpage
%\cleardoublepage

\pdfbookmark[section]{\contentsname}{toc}
\renewcommand{\contentsname}{Sommaire}
\tableofcontents{}

\section{Préparations}
\subsection{Prérequis}
(depuis \url{https://github.com/RIOT-OS/RIOT/wiki/Creating-your-first-RIOT-project}) :
\begin{itemize}
	\item Install and set up git
	\item Install the build-essential packet (make, gcc etc.). This varies based on
			the operating system in use.
	\item Install Native dependencies
	\item Install OpenOCD
	\item Install GCC Arm Embedded Toolchain
	\item On OS X: install Tuntap for OS X
	\item additional tweaks necessary to work with the targeted hardware (ATSAMR21)
	\item Install netcat with IPv6 support (if necessary)
			\begin{verbatim}sudo apt-get install netcat-openbsd}\end{verbatim}
	\item \begin{verbatim}git clone --recursive https://github.com/RIOT-OS/Tutorials\end{verbatim}
	\item Go to the Tutorials directory: cd Tutorials\\
\end{itemize}

Les 3 premiers points sont déjà OK sur ma machine.
\subsection{OpenOCD}

\url{https://github.com/RIOT-OS/RIOT/wiki/OpenOCD} :
\begin{verbatim}
OpenOCD (the open on-chip debugger) is an open source tool for debugging and
flashing microcontrollers. In RIOT we try to use this tool for as many
platforms as possible to reduce the overhead of having to keep track of many
different (and sometimes proprietary) tools.
\end{verbatim}

C'est donc l'outil qui devrait nous permettre de téléverser notre firmware. La page
explique son installation à partir des sources car les versions empaquetés dans les
distributions ne supporte pas les cartes les plus récentes supportées par RIOT. Cette
page datant de 2018 elle parle de la version 0.7.0 sur Mint 17 et de la version 0.9.0
compilée depuis les sources.\\

La version distribuée sur Debian 10 étant la 0.10.0 je vais m'en contenter dans un
premier temps et j'y reviendrai si nécessaire.

\subsection{GCC Arm Embedded Toolchain}

Sur Debian c'est le package gcc-arm-none-eabi déjà présent sur ma machine en versions
7-2018-q2, la version actuelle étant la 9-2019-q4. Comme pour OpenOCD je vais essayer
de commencer avec ça.

\subsection{En résumé}

Dans un shell root :
%{\scriptsize
\begin{verbatim}
dpkg --add-architecture i386
apt-get update
apt-get install libc6-dev-i386 libc6-dbg:i386 build-essential pkg-config \
uml-utilities bridge-utils git unzip
apt-get install openocd gcc-arm-none-eabi python3-serial
\end{verbatim}
%}
Dans un shell utilisateur :
%{\scriptsize
\begin{verbatim}
git clone --recursive https://github.com/RIOT-OS/Tutorials
cd Tutorials
\end{verbatim}
%}

Facile !



\section{Tutorials}
\subsection{Task 1: Starting the RIOT}
\subsubsection{native}
{\footnotesize
\begin{verbatim}
hugues@W520:~$ git clone --recursive https://github.com/RIOT-OS/Tutorials
Clonage dans 'Tutorials'...
remote: Enumerating objects: 8, done.
remote: Counting objects: 100% (8/8), done.
remote: Compressing objects: 100% (8/8), done.
remote: Total 637 (delta 0), reused 2 (delta 0), pack-reused 629
Réception d'objets: 100% (637/637), 2.37 MiB | 1.87 MiB/s, fait.
Résolution des deltas: 100% (357/357), fait.
Sous-module 'RIOT' (https://github.com/RIOT-OS/RIOT.git) enregistré pour le chemin 'RIOT'
Clonage dans '/home/hugues/Tutorials/RIOT'...
remote: Enumerating objects: 1, done.        
remote: Counting objects: 100% (1/1), done.        
remote: Total 222536 (delta 0), reused 0 (delta 0), pack-reused 222535        
Réception d'objets: 100% (222536/222536), 80.89 MiB | 2.07 MiB/s, fait.
Résolution des deltas: 100% (145265/145265), fait.
Chemin de sous-module 'RIOT' : '673187e29c06cdba74f0b82c7b30b9de2538531f' extrait
hugues@W520:~$ cd Tutorials/
hugues@W520:~/Tutorials$ ls
overview-net.png  phytec.png  README.md  RIOT  SAM-R21.jpg  slides  task-01  task-02
task-03  task-04  task-05  task-06  task-07  task-08  task-09  Vagrantfile
hugues@W520:~/Tutorials$ cd task-01/
hugues@W520:~/Tutorials/task-01$ make all term
Building application "Task01" for "native" with MCU "native".

"make" -C /home/hugues/Tutorials/RIOT/boards/native
"make" -C /home/hugues/Tutorials/RIOT/boards/native/drivers
"make" -C /home/hugues/Tutorials/RIOT/core
"make" -C /home/hugues/Tutorials/RIOT/cpu/native
"make" -C /home/hugues/Tutorials/RIOT/cpu/native/periph
"make" -C /home/hugues/Tutorials/RIOT/cpu/native/stdio_native
"make" -C /home/hugues/Tutorials/RIOT/cpu/native/vfs
"make" -C /home/hugues/Tutorials/RIOT/drivers
"make" -C /home/hugues/Tutorials/RIOT/drivers/periph_common
"make" -C /home/hugues/Tutorials/RIOT/sys
"make" -C /home/hugues/Tutorials/RIOT/sys/auto_init
"make" -C /home/hugues/Tutorials/RIOT/sys/ps
"make" -C /home/hugues/Tutorials/RIOT/sys/shell
"make" -C /home/hugues/Tutorials/RIOT/sys/shell/commands
   text	   data	    bss	    dec	    hex	filename
  28944	    728	  47728	  77400	  12e58	/home/hugues/Tutorials/task-01/bin/native/Task01.elf
/home/hugues/Tutorials/task-01/bin/native/Task01.elf  
RIOT native interrupts/signals initialized.
LED_RED_OFF
LED_GREEN_ON
RIOT native board initialized.
RIOT native hardware initialization complete.

main(): This is RIOT! (Version: 2020.04)
This is Task-01
> ps
ps
	pid | name                 | state    Q | pri | stack  ( used) | base addr  | current     
	  - | isr_stack            | -        - |   - |   8192 (   -1) | 0x56648400 | 0x56648400
	  1 | idle                 | pending  Q |  15 |   8192 (  436) | 0x56646120 | 0x56647f80 
	  2 | main                 | running  Q |   7 |  12288 ( 3148) | 0x56643120 | 0x56645f80 
	    | SUM                  |            |     |  28672 ( 3584)
> ^C
native: exiting
\end{verbatim}
}

Jusque-là rien de bien nouveau.

\subsubsection{Nucleo-144}

Il s'agit maintenant de déterminer la variable BOARD
pour notre matériel :
{\footnotesize
\begin{verbatim}
hugues@W520:~/Tutorials/task-01$ ls ../RIOT/boards/
acd52832                   iotlab-m3            openlabs-kw41z-mini-256kib
adafruit-clue              Kconfig              openmote-b
airfy-beacon               limifrog-v1          openmote-cc2538
arduino-due                lobaro-lorabox       particle-argon
arduino-duemilanove        lsn50                particle-boron
arduino-leonardo           maple-mini           particle-xenon
arduino-mega2560           mbed_lpc1768         pba-d-01-kw2x
arduino-mkr1000            mcb2388              phynode-kw41z
arduino-mkrfox1200         mega-xplained        pic32-clicker
arduino-mkrwan1300         microbit             pic32-wifire
arduino-mkrzero            microduino-corerf    pinetime
arduino-nano               msb-430              p-l496g-cell02
arduino-nano-33-ble        msb-430h             p-nucleo-wb55
arduino-uno                msba2                pyboard
arduino-zero               msbiot               README.md
atmega1284p                mulle                reel
atmega256rfr2-xpro         native               remote-pa
atmega328p                 nrf51dk              remote-reva
avr-rss2                   nrf51dongle          remote-revb
avsextrem                  nrf52832-mdk         ruuvitag
b-l072z-lrwan1             nrf52840dk           samd21-xpro
b-l475e-iot01a             nrf52840-mdk         same54-xpro
blackpill                  nrf52dk              saml10-xpro
blackpill-128kib           nrf6310              saml11-xpro
bluepill                   nucleo-f030r8        saml21-xpro
bluepill-128kib            nucleo-f031k6        samr21-xpro
calliope-mini              nucleo-f042k6        samr30-xpro
cc1312-launchpad           nucleo-f070rb        samr34-xpro
cc1352-launchpad           nucleo-f072rb        seeeduino_arch-pro
cc2538dk                   nucleo-f091rc        sensebox_samd21
cc2650-launchpad           nucleo-f103rb        slstk3401a
cc2650stk                  nucleo-f207zg        slstk3402a
chronos                    nucleo-f302r8        sltb001a
common                     nucleo-f303k8        slwstk6000b-slwrb4150a
derfmega128                nucleo-f303re        slwstk6000b-slwrb4162a
derfmega256                nucleo-f303ze        slwstk6220a
doc.txt                    nucleo-f334r8        sodaq-autonomo
ek-lm4f120xl               nucleo-f401re        sodaq-explorer
esp32-heltec-lora32-v2     nucleo-f410rb        sodaq-one
esp32-mh-et-live-minikit   nucleo-f411re        sodaq-sara-aff
esp32-olimex-evb           nucleo-f412zg        spark-core
esp32-ttgo-t-beam          nucleo-f413zh        stk3600
esp32-wemos-lolin-d32-pro  nucleo-f429zi        stk3700
esp32-wroom-32             nucleo-f446re        stm32f030f4-demo
esp32-wrover-kit           nucleo-f446ze        stm32f0discovery
esp8266-esp-12x            nucleo-f722ze        stm32f3discovery
esp8266-olimex-mod         nucleo-f746zg        stm32f429i-disc1
esp8266-sparkfun-thing     nucleo-f767zi        stm32f4discovery
f4vi1                      nucleo-l031k6        stm32f723e-disco
feather-m0                 nucleo-l053r8        stm32f769i-disco
feather-nrf52840           nucleo-l073rz        stm32l0538-disco
firefly                    nucleo-l152re        stm32l476g-disco
fox                        nucleo-l412kb        teensy31
frdm-k22f                  nucleo-l432kc        telosb
frdm-k64f                  nucleo-l433rc        thingy52
frdm-kw41z                 nucleo-l452re        ublox-c030-u201
hamilton                   nucleo-l476rg        udoo
hifive1                    nucleo-l496zg        usb-kw41z
hifive1b                   nucleo-l4r5zi        waspmote-pro
ikea-tradfri               nz32-sc151           wsn430-v1_3b
im880b                     olimexino-stm32      wsn430-v1_4
i-nucleo-lrwan1            opencm904            yunjia-nrf51822
iotlab-a8-m3               openlabs-kw41z-mini  z1
\end{verbatim}
}
Pour nous ce sera donc \enquote{nucleo-f746zg}.\\

Je connecte la carte à mon PC (coté st-link) et c'est parti :
{\scriptsize
\begin{verbatim}
hugues@W520:~/Tutorials/task-01$ BOARD=nucleo-f746zg make all flash term
Building application "Task01" for "nucleo-f746zg" with MCU "stm32f7".

"make" -C /home/hugues/Tutorials/RIOT/boards/nucleo-f746zg
"make" -C /home/hugues/Tutorials/RIOT/boards/common/nucleo
"make" -C /home/hugues/Tutorials/RIOT/core
"make" -C /home/hugues/Tutorials/RIOT/cpu/stm32f7
"make" -C /home/hugues/Tutorials/RIOT/cpu/cortexm_common
"make" -C /home/hugues/Tutorials/RIOT/cpu/cortexm_common/periph
"make" -C /home/hugues/Tutorials/RIOT/cpu/stm32_common
"make" -C /home/hugues/Tutorials/RIOT/cpu/stm32_common/periph
"make" -C /home/hugues/Tutorials/RIOT/cpu/stm32f7/periph
"make" -C /home/hugues/Tutorials/RIOT/drivers
"make" -C /home/hugues/Tutorials/RIOT/drivers/periph_common
"make" -C /home/hugues/Tutorials/RIOT/sys
"make" -C /home/hugues/Tutorials/RIOT/sys/auto_init
"make" -C /home/hugues/Tutorials/RIOT/sys/isrpipe
"make" -C /home/hugues/Tutorials/RIOT/sys/newlib_syscalls_default
"make" -C /home/hugues/Tutorials/RIOT/sys/pm_layered
"make" -C /home/hugues/Tutorials/RIOT/sys/ps
"make" -C /home/hugues/Tutorials/RIOT/sys/shell
"make" -C /home/hugues/Tutorials/RIOT/sys/shell/commands
"make" -C /home/hugues/Tutorials/RIOT/sys/stdio_uart
"make" -C /home/hugues/Tutorials/RIOT/sys/tsrb
   text	   data	    bss	    dec	    hex	filename
  14940	    132	   2632	  17704	   4528	/home/hugues/Tutorials/task-01/bin/nucleo-f746zg/Task01.elf
/home/hugues/Tutorials/RIOT/dist/tools/openocd/openocd.sh flash /home/hugues/Tutorials/task-01/bin/nucleo-f746zg/Task01.elf
### Flashing Target ###
Open On-Chip Debugger 0.10.0
Licensed under GNU GPL v2
For bug reports, read
	http://openocd.org/doc/doxygen/bugs.html
hla_swd
Info : The selected transport took over low-level target control. The results might differ compared to plain JTAG/SWD
adapter speed: 2000 kHz
adapter_nsrst_delay: 100
srst_only separate srst_nogate srst_open_drain connect_deassert_srst
srst_only separate srst_nogate srst_open_drain connect_deassert_srst
srst_only separate srst_nogate srst_open_drain connect_assert_srst
Info : Unable to match requested speed 2000 kHz, using 1800 kHz
Info : Unable to match requested speed 2000 kHz, using 1800 kHz
Info : clock speed 1800 kHz
Info : STLINK v2 JTAG v29 API v2 SWIM v18 VID 0x0483 PID 0x374B
Info : using stlink api v2
Info : Target voltage: 3.247431
Warn : Silicon bug: single stepping will enter pending exception handler!
Info : stm32f7x.cpu: hardware has 8 breakpoints, 4 watchpoints
    TargetName         Type       Endian TapName            State       
--  ------------------ ---------- ------ ------------------ ------------
 0* stm32f7x.cpu       hla_target little stm32f7x.cpu       reset
target halted due to debug-request, current mode: Thread 
xPSR: 0x01000000 pc: 0x0800a83c msp: 0x20003410
auto erase enabled
Info : device id = 0x10016449
Info : flash size = 1024kbytes
target halted due to breakpoint, current mode: Thread 
xPSR: 0x61000000 pc: 0x20000046 msp: 0x20003410
wrote 32768 bytes from file /home/hugues/Tutorials/task-01/bin/nucleo-f746zg/Task01.elf in 0.811100s (39.453 KiB/s)
target halted due to breakpoint, current mode: Thread 
xPSR: 0x61000000 pc: 0x2000002e msp: 0x20003410
target halted due to breakpoint, current mode: Thread 
xPSR: 0x61000000 pc: 0x2000002e msp: 0x20003410
verified 15072 bytes in 0.128518s (114.527 KiB/s)
shutdown command invoked
Done flashing
/home/hugues/Tutorials/RIOT/dist/tools/pyterm/pyterm -p "/dev/ttyACM0" -b "115200" 
Twisted not available, please install it if you want to use pyterm's JSON capabilities
2020-05-04 17:36:35,989 # Connect to serial port /dev/ttyACM0
Welcome to pyterm!
Type '/exit' to exit.
ps
2020-05-04 17:36:50,692 # ps
2020-05-04 17:36:50,700 # 	pid | name                 | state    Q | pri | stack  ( used) | base addr  | current     
2020-05-04 17:36:50,708 # 	  - | isr_stack            | -        - |   - |    512 (  116) | 0x20000000 | 0x200001c8
2020-05-04 17:36:50,716 # 	  1 | idle                 | pending  Q |  15 |    256 (  132) | 0x200003c0 | 0x2000043c 
2020-05-04 17:36:50,724 # 	  2 | main                 | running  Q |   7 |   1536 (  652) | 0x200004c0 | 0x20000924 
2020-05-04 17:36:50,729 # 	    | SUM                  |            |     |   2304 (  900)
> reboot
2020-05-04 17:41:10,933 #  reboot
2020-05-04 17:41:10,940 # main(): This is RIOT! (Version: 2020.04)
2020-05-04 17:41:10,941 # This is Task-01
> /exit
2020-05-04 17:46:17,710 # Exiting Pyterm
hugues@W520:~/Tutorials/task-01$ /home/hugues/Tutorials/RIOT/dist/tools/pyterm/pyterm -p "/dev/ttyACM0" -b "115200"
Twisted not available, please install it if you want to use pyterm's JSON capabilities
2020-05-04 17:48:12,032 # Connect to serial port /dev/ttyACM0
Welcome to pyterm!
Type '/exit' to exit.
> /exit
2020-05-04 17:48:33,838 # Exiting Pyterm
hugues@W520:~/Tutorials/task-01$
\end{verbatim}
}
Que dire ? Ça fonctionne tout seul !\\

\subsubsection{bluepill (STM32F103C8T6)}

Cette carte très bon marché (encore moins cher qu'un nano atmega328p) pourrait être
utilisée dans une version mono-bras du convertisseur. J'envisage de démarrer le
développement du nouveau firmware par le remplacement du nano par cette carte sur
un prototype OwnWall dont je dispose en attendant le premier prototype OwnTech, le
CAN (MCP3208) étant identique à ceux qui seront utilisé.

{\scriptsize
\begin{verbatim}
### Flashing Target ###
Open On-Chip Debugger 0.10.0
Licensed under GNU GPL v2
For bug reports, read
	http://openocd.org/doc/doxygen/bugs.html
hla_swd
srst_only separate srst_nogate srst_open_drain connect_assert_srst
Info : The selected transport took over low-level target control. The results might differ compared to plain JTAG/SWD
adapter speed: 1000 kHz
adapter_nsrst_delay: 100
srst_only separate srst_nogate srst_open_drain connect_assert_srst
srst_only separate srst_nogate srst_open_drain connect_assert_srst
Info : Unable to match requested speed 1000 kHz, using 950 kHz
Info : Unable to match requested speed 1000 kHz, using 950 kHz
Info : clock speed 950 kHz
Error: open failed
in procedure 'init' 
in procedure 'ocd_bouncer'

make: *** [/home/hugues/Tutorials/task-01/../RIOT/Makefile.include:667: flash] Error 1
\end{verbatim}
}
Ça n'a pas fonctionné, peut-être à cause du clône de st-link v2 Chinois à 2\euro...\\

Chaque dossier dans RIOT/boards contient un fichier doc.txt, pour la bluepill j'y
trouve l'info suivante :
\begin{verbatim}
## Flashing

To program and debug the board you need a SWD capable debugger. The
easiest way is using [OpenOCD][OpenOCD]. By default RIOT uses the hardware
reset signal and connects to the chip under reset for flashing. This is
required to reliably connect to the device even when the MCU is in a low power
mode. Therefore not only SWDIO and SWCLK, but also the RST pin of your
debugger need to be connected to the board.
\end{verbatim}

J'ai donc commencé par ajouter le fil de reset manquant dans mon câblage, mais ça n'a
pas résolu le problème.\\

Je me suis donc penché sur OpenOCD en essayant de le lancer directement :
{\scriptsize
\begin{verbatim}
hugues@W520:~/Tutorials/task-01$ openocd -f /usr/share/openocd/scripts/interface/stlink-v2.cfg \
-f /usr/share/openocd/scripts/target/stm32f1x.cfg 
Open On-Chip Debugger 0.10.0
Licensed under GNU GPL v2
For bug reports, read
	http://openocd.org/doc/doxygen/bugs.html
Info : auto-selecting first available session transport "hla_swd". To override use 'transport select <transport>'.
Info : The selected transport took over low-level target control. The results might differ compared to plain JTAG/SWD
adapter speed: 1000 kHz
adapter_nsrst_delay: 100
none separate
Info : Unable to match requested speed 1000 kHz, using 950 kHz
Info : Unable to match requested speed 1000 kHz, using 950 kHz
Info : clock speed 950 kHz
Info : STLINK v2 JTAG v29 API v2 SWIM v7 VID 0x0483 PID 0x3748
Info : using stlink api v2
Info : Target voltage: 3.166154
Warn : UNEXPECTED idcode: 0x2ba01477
Error: expected 1 of 1: 0x1ba01477
in procedure 'init' 
in procedure 'ocd_bouncer'
\end{verbatim}
}
Ça n'a pas fonctionné non plus mais l'erreur est différente, le voyant du stlink est
passé du rouge au bleue et il semble que ma bluepill n'a pas le bon idcode. J'ai donc
remplacé 0x1ba01477 par 0x2ba01477 dans
\texttt{/usr/share/openocd/scripts/target/stm32f1x.cfg}, après quoi openocd semble se
lancer correctement.\\

J'obtiens malgré tout toujours la même erreur avec la commande make flash.\\

Dans le fichier \texttt{/RIOT/dist/tools/openocd/openocd.sh} j'observe que la
commande utilise la constante OPENOCD\_ADAPTER\_INIT, j'effectue donc une recherche
dans le code source :
{\scriptsize
\begin{verbatim}
hugues@W520:~/Tutorials/task-01$ grep -r OPENOCD_ADAPTER_INIT ../RIOT/*
../RIOT/dist/tools/openocd/openocd.sh:: ${OPENOCD_ADAPTER_INIT:=}
../RIOT/dist/tools/openocd/openocd.sh:            ${OPENOCD_ADAPTER_INIT} \
../RIOT/dist/tools/openocd/openocd.sh:            ${OPENOCD_ADAPTER_INIT} \
../RIOT/dist/tools/openocd/openocd.sh:            ${OPENOCD_ADAPTER_INIT} \
../RIOT/dist/tools/openocd/openocd.sh:            ${OPENOCD_ADAPTER_INIT} \
../RIOT/dist/tools/openocd/openocd.sh:            ${OPENOCD_ADAPTER_INIT} \
../RIOT/dist/tools/openocd/openocd.sh:            ${OPENOCD_ADAPTER_INIT} \
../RIOT/dist/tools/buildsystem_sanity_check/check.sh:UNEXPORTED_VARIABLES+=('OPENOCD_ADAPTER_INIT')
../RIOT/makefiles/tools/openocd.inc.mk:# Export OPENOCD_ADAPTER_INIT to required targets
../RIOT/makefiles/tools/openocd.inc.mk:$(call target-export-variables,$(OPENOCD_TARGETS),OPENOCD_ADAPTER_INIT)
../RIOT/makefiles/tools/openocd-adapters/stlink.inc.mk:OPENOCD_ADAPTER_INIT ?= \
../RIOT/makefiles/tools/openocd-adapters/stlink.inc.mk:  OPENOCD_ADAPTER_INIT += -c 'hla_serial $(DEBUG_ADAPTER_ID)'
../RIOT/makefiles/tools/openocd-adapters/iotlab.inc.mk:OPENOCD_ADAPTER_INIT ?= -c 'source [find interface/ftdi/iotlab-usb.cfg]'
../RIOT/makefiles/tools/openocd-adapters/iotlab.inc.mk:  OPENOCD_ADAPTER_INIT += -c 'ftdi_serial $(DEBUG_ADAPTER_ID)'
../RIOT/makefiles/tools/openocd-adapters/jlink.inc.mk:OPENOCD_ADAPTER_INIT ?= -c 'source [find interface/jlink.cfg]'
../RIOT/makefiles/tools/openocd-adapters/jlink.inc.mk:  OPENOCD_ADAPTER_INIT += -c 'jlink serial $(DEBUG_ADAPTER_ID)'
../RIOT/makefiles/tools/openocd-adapters/dap.inc.mk:OPENOCD_ADAPTER_INIT ?= -c 'source [find interface/cmsis-dap.cfg]'
../RIOT/makefiles/tools/openocd-adapters/dap.inc.mk:  OPENOCD_ADAPTER_INIT += -c 'cmsis_dap_serial $(DEBUG_ADAPTER_ID)'
../RIOT/makefiles/tools/openocd-adapters/sysfs_gpio.inc.mk:OPENOCD_ADAPTER_INIT ?= \
../RIOT/makefiles/tools/openocd-adapters/mulle.inc.mk:OPENOCD_ADAPTER_INIT ?= -f '$(RIOTBASE)/boards/mulle/dist/openocd/mulle-programmer-$(PROGRAMMER_VERSION).cfg'
../RIOT/makefiles/tools/openocd-adapters/mulle.inc.mk:  OPENOCD_ADAPTER_INIT += -c 'ftdi_serial $(DEBUG_ADAPTER_ID)'
../RIOT/makefiles/tools/openocd-adapters/raspi.inc.mk:OPENOCD_ADAPTER_INIT ?= \
hugues@W520:~/Tutorials/task-01$ cat ../RIOT/makefiles/tools/openocd-adapters/stlink.inc.mk
# ST-Link debug adapter
# Use st-link v2-1 by default
STLINK_VERSION ?= 2-1

# Use STLINK_VERSION to select which stlink version is used
OPENOCD_ADAPTER_INIT ?= \
  -c 'source [find interface/stlink-v$(STLINK_VERSION).cfg]' \
  -c 'transport select hla_swd'
# Add serial matching command, only if DEBUG_ADAPTER_ID was specified
ifneq (,$(DEBUG_ADAPTER_ID))
  OPENOCD_ADAPTER_INIT += -c 'hla_serial $(DEBUG_ADAPTER_ID)'
endif

# if no openocd specific configuration file, check for default locations:
# 1. Using the default dist/openocd.cfg (automatically set by openocd.sh)
# 2. Using the common cpu specific config file
ifeq (,$(OPENOCD_CONFIG))
  # if no openocd default configuration is provided by the board,
  # use the STM32 common one
  ifeq (0,$(words $(wildcard $(BOARDSDIR)/$(BOARD)/dist/openocd.cfg)))
    OPENOCD_CONFIG = $(RIOTBASE)/boards/common/stm32/dist/$(CPU).cfg
  endif
endif
\end{verbatim}
}
Je constate qu'il cherche à utilise la version 2-1 de l'adaptateur stlink qui est
adapté pour l'adaptateur intégré à la nucleo mais pas pour mon adaptateur à 2\euro
(v2 tout court).\\

Je retente donc en fixant cette constante dans ma commande make :
{\scriptsize
\begin{verbatim}
### Flashing Target ###
Open On-Chip Debugger 0.10.0
Licensed under GNU GPL v2
For bug reports, read
	http://openocd.org/doc/doxygen/bugs.html
hla_swd
srst_only separate srst_nogate srst_open_drain connect_assert_srst
Info : The selected transport took over low-level target control. The results might differ compared to plain JTAG/SWD
adapter speed: 1000 kHz
adapter_nsrst_delay: 100
srst_only separate srst_nogate srst_open_drain connect_assert_srst
srst_only separate srst_nogate srst_open_drain connect_assert_srst
Info : Unable to match requested speed 1000 kHz, using 950 kHz
Info : Unable to match requested speed 1000 kHz, using 950 kHz
Info : clock speed 950 kHz
Info : STLINK v2 JTAG v29 API v2 SWIM v7 VID 0x0483 PID 0x3748
Info : using stlink api v2
Info : Target voltage: 3.165663
Error: init mode failed (unable to connect to the target)
in procedure 'init' 
in procedure 'ocd_bouncer'

make: *** [/home/hugues/Tutorials/task-01/../RIOT/Makefile.include:667: flash] Error 1
\end{verbatim}
}
Cette fois il semble reconnaître convenablement mon stlink mais n'arrive pas à se
connecter au stm32, je fais un essai avec un reset manuel :
{\scriptsize
\begin{verbatim}
### Flashing Target ###
Open On-Chip Debugger 0.10.0
Licensed under GNU GPL v2
For bug reports, read
	http://openocd.org/doc/doxygen/bugs.html
hla_swd
srst_only separate srst_nogate srst_open_drain connect_assert_srst
Info : The selected transport took over low-level target control. The results might differ compared to plain JTAG/SWD
adapter speed: 1000 kHz
adapter_nsrst_delay: 100
srst_only separate srst_nogate srst_open_drain connect_assert_srst
srst_only separate srst_nogate srst_open_drain connect_assert_srst
Info : Unable to match requested speed 1000 kHz, using 950 kHz
Info : Unable to match requested speed 1000 kHz, using 950 kHz
Info : clock speed 950 kHz
Info : STLINK v2 JTAG v29 API v2 SWIM v7 VID 0x0483 PID 0x3748
Info : using stlink api v2
Info : Target voltage: 3.159513
Info : STM32F103C8Tx.cpu: hardware has 6 breakpoints, 4 watchpoints
    TargetName         Type       Endian TapName            State       
--  ------------------ ---------- ------ ------------------ ------------
 0* STM32F103C8Tx.cpu  hla_target little STM32F103C8Tx.cpu  reset
target halted due to debug-request, current mode: Thread 
xPSR: 0x01000000 pc: 0x08000450 msp: 0x20000200
auto erase enabled
Info : device id = 0x20036410
Info : flash size = 64kbytes
target halted due to breakpoint, current mode: Thread 
xPSR: 0x61000000 pc: 0x2000003a msp: 0x20000200
wrote 14336 bytes from file /home/hugues/Tutorials/task-01/bin/bluepill/Task01.elf in 0.483442s (28.959 KiB/s)
target halted due to breakpoint, current mode: Thread 
xPSR: 0x61000000 pc: 0x2000002e msp: 0x20000200
target halted due to breakpoint, current mode: Thread 
xPSR: 0x61000000 pc: 0x2000002e msp: 0x20000200
verified 14336 bytes in 0.249575s (56.095 KiB/s)
target halted due to breakpoint, current mode: Thread 
xPSR: 0x61000000 pc: 0x2000002e msp: 0x20000200
shutdown command invoked
Done flashing
\end{verbatim}
}
Après plusieurs essais je parviens à relâcher le reset au bon moment et ça flash. Le
pin reset de ce st-link ne semble donc pas fonctionner. Après démontage il est
seulement connecté à une résistance de pull-up.\\

On peut trouver le schéma du st-link/v2-1 dans le manuel de la nucleo-144, et on y
voit que le reset est sensé être connecté à la pin 18 du microcontrôleur interne ce
qui n'est pas le cas. Deux coups de fer à souder plus tard (j'ai vite regretté de
m'être lancé dans une soudure sur un pas de 0,5mm en télétravail, avec une panne de
0,5 et de la soudure de 0,7 sans flux et sans loupe) j'ai pu de nouveau
tester : cette fois c'est passé tout seul.\\

Nouveau problème avec pyterm cette fois : \texttt{No such file or directory:
'/dev/ttyACM0'}. Ce stlink n'intègre pas d'adaptateur série mais j'ai un dongle
ch340 que je peux utiliser pour ça. Une petite recherche m'indique : \enquote{The
default UART port used is UART2, which uses pins A2 (TX) and A3 (RX).}.\\

Le CH340 n'est pas attaché à /dev/ttyACM0 comme le st-link v2-1 de la nucleo mais à
/dev/ttyUSB0. Il faut donc ajouter PORT\_LINUX=/dev/ttyUSB0 pour la commande make ce
qui nous donne :
{\scriptsize
\begin{verbatim}
hugues@W520:~/Tutorials/task-01$ BOARD=bluepill STLINK_VERSION=2 PORT_LINUX=/dev/ttyUSB0 make all flash term
\end{verbatim}
}
Bien sûr on peut mettre tout ça dans le Makefile :
\begin{lstlisting}[language=make]
# If no BOARD is found in the environment, use this default:
BOARD ?= bluepill

# To use chinese st-link v2 and ch340 dongle with bluepill
ifeq ($(BOARD),bluepill)
	STLINK_VERSION=2
	PORT_LINUX=/dev/ttyUSB0
endif
\end{lstlisting}

Ainsi je n'aurai plus qu'à tapper la commande \texttt{make all flash term} pour
utiliser la bluepill par défaut (je me suis fais gronder d'avoir emprunté le câble
de l'appareil photo de ma femme pour relier la nucleo) en attendant de m'en
procurer un.\\

Voilà ce que ça donne au niveau de la pile :
{\scriptsize
\begin{verbatim}
2020-05-06 12:20:49,693 # 	pid | name                 | state    Q | pri | stack  ( used) | base addr  | current     
2020-05-06 12:20:49,701 # 	  - | isr_stack            | -        - |   - |    512 (  116) | 0x20000000 | 0x200001c8
2020-05-06 12:20:49,709 # 	  1 | idle                 | pending  Q |  15 |    256 (  132) | 0x200003c0 | 0x2000043c 
2020-05-06 12:20:49,718 # 	  2 | main                 | running  Q |   7 |   1536 (  660) | 0x200004c0 | 0x20000924 
2020-05-06 12:20:49,724 # 	    | SUM                  |            |     |   2304 (  908)
\end{verbatim}
}

\subsubsection{arduino-nano (ATmega328p)}

Disons \enquote{pour le fun}, et pour voir ce que ça donne sur un $\mu$C
8 bits avec seulement 2Ko de RAM.\\

La commande toujours aussi simple :
{\scriptsize
\begin{verbatim}
hugues@W520:~/Tutorials/task-01$ BOARD=arduino-nano make all flash term
\end{verbatim}
}
Et le résultat :
{\scriptsize
\begin{verbatim}
   text	   data	    bss	    dec	    hex	filename
   7094	    852	    961	   8907	   22cb	/home/hugues/Tutorials/task-01/bin/arduino-nano/Task01.elf
...
avrdude: writing flash (7946 bytes)
...
2020-05-06 12:35:58,933 # 	pid | name                 | state    Q | pri | stack  ( used) | base addr  | current     
2020-05-06 12:35:59,032 # 	  1 | idle                 | pending  Q |  15 |    128 (   86) |      0x454 |      0x481 
2020-05-06 12:35:59,132 # 	  2 | main                 | running  Q |   7 |    640 (  306) |      0x4d4 |      0x64b 
2020-05-06 12:35:59,193 # 	    | SUM                  |            |     |    768 (  392)
\end{verbatim}
}
L'utilisation mémoire (data+bss) est importante 1813 sur 2048 mais ça
passe...


\subsection{Task 2: Custom shell handlers}
Il s'agit d'écrire 2 gestionnaires de commande shell :
\begin{itemize}
	\item une commande echo ;
	\item une commande toggle pour une led.\\
\end{itemize}

\subsubsection{stm32}
Lors de la compilation j'ai recontré le problème suivant :
{\scriptsize
\begin{verbatim}
hugues@W520:~/Tutorials/task-02$ make all flash term
Building application "Task02" for "bluepill" with MCU "stm32f1".

In file included from /home/hugues/Tutorials/RIOT/boards/bluepill/include/board.h:32:0,
                 from /home/hugues/Tutorials/RIOT/drivers/include/led.h:35,
                 from /home/hugues/Tutorials/task-02/main.c:5:
/home/hugues/Tutorials/task-02/main.c: In function 'toggle':
/home/hugues/Tutorials/RIOT/boards/common/blxxxpill/include/board_common.h:36:29: error: 'GPIOC' undeclared (first u
se in this function)
 #define LED0_PORT           GPIOC                                   /**< GPIO-Port the LED is connected to */
                             ^
/home/hugues/Tutorials/RIOT/boards/common/blxxxpill/include/board_common.h:49:30: note: in expansion of macro 'LED0_
PORT'
 #define LED0_TOGGLE         (LED0_PORT->ODR  ^= LED0_MASK)          /**< Toggle LED0 */
                              ^~~~~~~~~
/home/hugues/Tutorials/task-02/main.c:33:2: note: in expansion of macro 'LED0_TOGGLE'
  LED0_TOGGLE;
  ^~~~~~~~~~~
/home/hugues/Tutorials/RIOT/boards/common/blxxxpill/include/board_common.h:36:29: note: each undeclared identifier i
s reported only once for each function it appears in
 #define LED0_PORT           GPIOC                                   /**< GPIO-Port the LED is connected to */
                             ^
/home/hugues/Tutorials/RIOT/boards/common/blxxxpill/include/board_common.h:49:30: note: in expansion of macro 'LED0_
PORT'
 #define LED0_TOGGLE         (LED0_PORT->ODR  ^= LED0_MASK)          /**< Toggle LED0 */
                              ^~~~~~~~~
/home/hugues/Tutorials/task-02/main.c:33:2: note: in expansion of macro 'LED0_TOGGLE'
  LED0_TOGGLE;
  ^~~~~~~~~~~
make[1]: *** [/home/hugues/Tutorials/RIOT/Makefile.base:110: /home/hugues/Tutorials/task-02/bin/bluepill/application
_Task01/main.o] Error 1
make: *** [/home/hugues/Tutorials/task-02/../RIOT/Makefile.include:538: /home/hugues/Tutorials/task-02/bin/bluepill/
application_Task01.a] Error 2
\end{verbatim}
}
Le problème ne se produit qu'avec BOARD=bluepill, \texttt{make all}
fonctionne bien avec \enquote{native}, \enquote{nucleo-f746zg}, ou même
\enquote{nucleo-f103rb}.\\

Le problème est peut-être lié à l'ordre de traitement des dépendances
 : \url{https://github.com/RIOT-OS/RIOT/issues/9913}.\\
 
J'ai pu le contourner par une inclusion conditionnelle :
\begin{lstlisting}
#ifdef BOARD_BLUEPILL
#include "vendor/stm32f103xe.h"
#endif
\end{lstlisting}

\subsubsection{main.c}
Voilà ce que donne le fichier main.c :
\begin{lstlisting}
#include <stdio.h>
#include <string.h>

#include "shell.h"
#include "led.h"

#ifdef BOARD_BLUEPILL
#include "vendor/stm32f103xe.h"
#endif

int echo(int argc, char **argv)
{
    /* Print a line of text */
    (void)argc;
    (void)argv;
    
    if (argc > 1) {
    	for (int i = 1; i < argc-1; i++) {
	        printf("%s ", argv[i]);
	    }
		printf("%s\n", argv[argc-1]);
    }
    else {
		printf("\n");
	}
	
    return 0;
}

int toggle(int argc, char **argv)
{
	/** Toggles the primary LED on the board **/
    (void)argc;
    (void)argv;

	LED0_TOGGLE;
	
	return 0;
}

static const shell_command_t commands[] = {
    { "echo", "Print a line of text", echo },
    { "toggle", "Toggles the primary LED on the board", toggle },
    { NULL, NULL, NULL }
};

int main(void)
{
    puts("This is Task-02");

    char line_buf[SHELL_DEFAULT_BUFSIZE];
    shell_run(commands, line_buf, SHELL_DEFAULT_BUFSIZE);

    return 0;
}
\end{lstlisting}

\subsubsection{arduino-nano}
Sur l'arduino ça ne passe plus :
{\scriptsize
\begin{verbatim}
/usr/lib/gcc/avr/5.4.0/../../../avr/bin/ld : l'adresse 0x800871 de /home/hugues/Tutorials/task-02/bin/arduino-nano
/Task02.elf de la section «.bss» n'est pas dans la région «data»
\end{verbatim}
}
Pour que ça passe on peut réduire la valeur de
\texttt{THREAD\_STACKSIZE\_DEFAULT} dans le fichier
\texttt{RIOT/cpu/atmega\_common/include/cpu\_conf.h} de 512 à 256 :
{\scriptsize
\begin{verbatim}
2020-05-06 15:24:43,923 # 	pid | name                 | state    Q | pri | stack  ( used) | base addr  | current     
2020-05-06 15:24:44,023 # 	  1 | idle                 | pending  Q |  15 |    128 (   84) |      0x4b2 |      0x4df 
2020-05-06 15:24:44,123 # 	  2 | main                 | running  Q |   7 |    384 (  336) |      0x532 |      0x5a9 
2020-05-06 15:24:44,166 # 	    | SUM                  |            |     |    512 (  420)
\end{verbatim}
}
La commande \texttt{toggle} ne fonctionne pas.\\

Il y avait une erreur dans le fichier :\\
\texttt{RIOT/boards/common/arduino-atmega/include/board\_common.h}.\\

Le patch pour que ça fonctionne sur le nano :
\begin{lstlisting}
--- ../RIOT/boards/common/arduino-atmega/include/board_common.h	2020-04-27 21:01:08.376660474 +0200
+++ RIOT/boards/common/arduino-atmega/include/board_common.h	2020-05-06 16:14:12.768798005 +0200
@@ -76,9 +76,9 @@
 #define LED2_ON             (PORTD &= ~LED2_MASK)
 #define LED2_TOGGLE         (PORTD ^=  LED2_MASK)
 #else
-#define LED0_ON             (PORTD |=  LED0_MASK)
-#define LED0_OFF            (PORTD &= ~LED0_MASK)
-#define LED0_TOGGLE         (PORTD ^=  LED0_MASK)
+#define LED0_ON             (PORTB |=  LED0_MASK)
+#define LED0_OFF            (PORTB &= ~LED0_MASK)
+#define LED0_TOGGLE         (PORTB ^=  LED0_MASK)
 #endif
 /** @} */
\end{lstlisting}


\subsection{Task 3: Multithreading}
Il s'agit de lancer un thread.\\

On ajoute le code donné dans la fonction main() et lorsque le handler
est exécuté il affiche \enquote{I'm in "thread" now} dans le terminal.\\

J'ai trouvé ça un peu nul donc j'ai ajouté un \texttt{ps();} dans le
handler pour pouvoir mieux l'observer (il faut aussi ajouter
\texttt{\#include "ps.h"}) :
{\scriptsize
\begin{verbatim}
 2020-05-07 01:25:33,090 # reboot
2020-05-07 01:25:33,093 # main(): This is RIOT! (Version: 2020.04)
2020-05-07 01:25:33,095 # This is Task-03
2020-05-07 01:25:33,096 # I'm in "thread" now
2020-05-07 01:25:33,104 # 	pid | name                 | state    Q | pri | stack  ( used) | base addr  | current     
2020-05-07 01:25:33,113 # 	  - | isr_stack            | -        - |   - |    512 (  104) | 0x20000000 | 0x200001c8
2020-05-07 01:25:33,121 # 	  1 | idle                 | pending  Q |  15 |    256 (  124) | 0x200003c0 | 0x20000444 
2020-05-07 01:25:33,130 # 	  2 | main                 | pending  Q |   7 |   1536 (  360) | 0x200004c0 | 0x20000974 
2020-05-07 01:25:33,136 # 	  3 | thread               | running  Q |   6 |   1536 (  448) | 0x20000ac4 | 0x2000104c 
2020-05-07 01:25:33,143 # 	    | SUM                  |            |     |   3840 ( 1036)
> ps
2020-05-07 01:25:41,334 #  ps
2020-05-07 01:25:41,340 # 	pid | name                 | state    Q | pri | stack  ( used) | base addr  | current     
2020-05-07 01:25:41,348 # 	  - | isr_stack            | -        - |   - |    512 (  116) | 0x20000000 | 0x200001c8
2020-05-07 01:25:41,357 # 	  1 | idle                 | pending  Q |  15 |    256 (  132) | 0x200003c0 | 0x2000043c 
2020-05-07 01:25:41,365 # 	  2 | main                 | running  Q |   7 |   1536 (  676) | 0x200004c0 | 0x20000914 
2020-05-07 01:25:41,372 # 	    | SUM                  |            |     |   2304 (  924)
\end{verbatim}
}
J'observe une forte augmentation de la mémoire utilisée dans la pile du
thread main (360$\rightarrow$676) suite à la commande \texttt{ps}.\\

Au début du programme donné il y a :
\texttt{char stack[THREAD\_STACKSIZE\_MAIN];}.

Cette constante est définie dans le fichier :
\texttt{RIOT/blob/master/core/include/thread.h}.

On y voit que :\\
\texttt{THREAD\_STACKSIZE\_MAIN} =
\texttt{THREAD\_STACKSIZE\_DEFAULT} +
\texttt{THREAD\_EXTRA\_STACKSIZE\_PRINTF}\\

Sur le stm32 \texttt{THREAD\_STACKSIZE\_MAIN} = 1024 et 
\texttt{THREAD\_EXTRA\_STACKSIZE\_PRINTF} = 512.\\

On y voit qu'il y a aussi :\\
\begin{itemize}
\item \texttt{THREAD\_STACKSIZE\_LARGE = THREAD\_STACKSIZE\_MEDIUM x 2}
\item \texttt{THREAD\_STACKSIZE\_MEDIUM = THREAD\_STACKSIZE\_DEFAULT}
\item \texttt{THREAD\_STACKSIZE\_SMALL = THREAD\_STACKSIZE\_MEDIUM / 2}
\item \texttt{THREAD\_STACKSIZE\_TINY = THREAD\_STACKSIZE\_MEDIUM / 4}
\end{itemize}~\\

Je l'ai donc définie à \texttt{THREAD\_STACKSIZE\_SMALL} (512) ce qui
est suffisant.\\

Sur le nano c'est insuffisant : vu que j'ai réduit
\texttt{THREAD\_STACKSIZE\_MAIN} de moitié on ser retrouve avec
l'équivalent de \texttt{THREAD\_EXTRA\_STACKSIZE\_PRINTF} et il nous 
faudrait un peu plus.\\

Le bon compromis est donc \texttt{THREAD\_STACKSIZE\_TINY} +
\texttt{THREAD\_EXTRA\_STACKSIZE\_PRINTF} ce qui nous donnera 768 sur le
stm32 et 392 sur l'avr.


\subsection{Task 4: Timers}
Utiliser \texttt{xtimer} pour créer un thread qui affiche le temps
système courant toutes les 2 secondes.\\

\subsubsection{Thread handler}
{\small
\begin{lstlisting}
void *timer2s_handler(void *arg)
{
    /* Thread that print the current system time every 2 seconds... */
    (void)arg;
    
    xtimer_ticks32_t last_wakeup = xtimer_now();

    while (1) {
        printf("Current system time is %lu \(\mu\)s\n", xtimer_now_usec());
        xtimer_periodic_wakeup(&last_wakeup, PERIOD_US);
    }

    return NULL;
}
\end{lstlisting}
}
\subsubsection{nucleo}
{\scriptsize
\begin{verbatim}
2020-06-17 15:39:54,773 # reboot
2020-06-17 15:39:54,779 # main(): This is RIOT! (Version: 2020.04)
2020-06-17 15:39:54,781 # This is Task-04
2020-06-17 15:39:54,784 # Current system time is 5036 \(\mu\)s
> 2020-06-17 15:39:56,784 #  Current system time is 2005053 \(\mu\)s
2020-06-17 15:39:58,784 # Current system time is 4005052 \(\mu\)s
2020-06-17 15:40:00,784 # Current system time is 6005052 \(\mu\)s
ps
2020-06-17 15:40:02,222 # ps
2020-06-17 15:40:02,230 # 	pid | name                 | state    Q | pri | stack  ( used) | base addr  | current     
2020-06-17 15:40:02,238 # 	  - | isr_stack            | -        - |   - |    512 (  140) | 0x20000000 | 0x200001c8
2020-06-17 15:40:02,246 # 	  1 | idle                 | pending  Q |  15 |    256 (  132) | 0x200003e8 | 0x20000464 
2020-06-17 15:40:02,254 # 	  2 | main                 | running  Q |   7 |   1536 (  660) | 0x200004e8 | 0x2000093c 
2020-06-17 15:40:02,262 # 	  3 | timer2s              | bl mutex _ |   6 |   1024 (  356) | 0x20000aec | 0x20000e04 
2020-06-17 15:40:02,268 # 	    | SUM                  |            |     |   3328 ( 1288)
> 2020-06-17 15:40:02,784 #  Current system time is 8005052 \(\mu\)s
2020-06-17 15:40:04,784 # Current system time is 10005052 \(\mu\)s
2020-06-17 15:40:06,784 # Current system time is 12005053 \(\mu\)s
\end{verbatim}
}

Le printf imprime sur la console à 115200 bauds ce qui prend environ
87 $\mu$s par caractère. La fonction \texttt{xtimer\_periodic\_wakeup()}
se débrouille pour compenser ça. Le chiffre des unités oscille entre 2
et 3, il y a donc une légère imprécision.\\

\subsubsection{native}
Il y a une petite erreur à la compilation au niveau du printf, il faut
remplacer \%lu par \%u pour que ça fonctionne.
{\scriptsize
\begin{verbatim}
main(): This is RIOT! (Version: 2020.04)
This is Task-04
Current system time is 66 \(\mu\)s
> Current system time is 2000220 \(\mu\)s
Current system time is 4000204 \(\mu\)s
Current system time is 6000216 \(\mu\)s
ps
ps
	pid | name                 | state    Q | pri | stack  ( used) | base addr  | current     
	  - | isr_stack            | -        - |   - |   8192 (   -1) | 0x565f4380 | 0x565f4380
	  1 | idle                 | pending  Q |  15 |   8192 (  436) | 0x565f20a0 | 0x565f3f00 
	  2 | main                 | running  Q |   7 |  12288 ( 3212) | 0x565ef0a0 | 0x565f1f00 
	  3 | timer2s              | bl mutex _ |   6 |   8192 ( 2452) | 0x565ed0a0 | 0x565eef00 
	    | SUM                  |            |     |  36864 ( 6100)
> Current system time is 8000134 \(\mu\)s
Current system time is 10000222 \(\mu\)s
Current system time is 12000209 \(\mu\)s
Current system time is 14000232 \(\mu\)s
Current system time is 16000342 \(\mu\)s
Current system time is 18000165 \(\mu\)s
\end{verbatim}
}

Là c'est beaucoup moins stable ce qui est normal étant donné que linux
n'est pas un système temps réel. Un \texttt{renice} du processus linux
avec un priorité haute (-20) ne change pas grand chose. La gestion d'énergie
change sans cesse la fréquence du processeur sur ma machine, peut être
qu'une fréquence fixe améliorerait les chose mais ce n'est pas faisable
sans passer par le BIOS avec ce processeur...

\subsubsection{arduino-nano}
{\scriptsize
\begin{verbatim}
2020-06-17 16:35:00,748 # main(): This is RIOT! (Version: 2020.04)
2020-06-17 16:35:00,749 # This is Task-04
2020-06-17 16:35:00,781 # Current system time is 59124 \(\mu\)s
> 2020-06-17 16:35:02,781 #  Current system time is 2059280 \(\mu\)s
2020-06-17 16:35:04,784 # Current system time is 4059280 \(\mu\)s
2020-06-17 16:35:06,786 # Current system time is 6059280 \(\mu\)s
2020-06-17 16:35:08,788 # Current system time is 8059280 \(\mu\)s
2020-06-17 16:35:10,792 # Current system time is 10059280 \(\mu\)s
ps
2020-06-17 16:35:11,199 # ps
2020-06-17 16:35:11,266 # 	pid | name                 | state    Q | pri | stack  ( used) | base addr  | current     
2020-06-17 16:35:11,366 # 	  1 | idle                 | pending  Q |  15 |    128 (   94) |      0x562 |      0x58f 
2020-06-17 16:35:11,466 # 	  2 | main                 | running  Q |   7 |    384 (  364) |      0x5e2 |      0x659 
2020-06-17 16:35:11,566 # 	  3 | timer2s              | bl mutex _ |   6 |    256 (  142) |      0x462 |      0x4d5 
2020-06-17 16:35:11,621 # 	    | SUM                  |            |     |    768 (  600)
> 2020-06-17 16:35:12,794 #  Current system time is 12059280 \(\mu\)s
2020-06-17 16:35:14,797 # Current system time is 14059280 \(\mu\)s
2020-06-17 16:35:16,799 # Current system time is 16059280 \(\mu\)s
2020-06-17 16:35:18,802 # Current system time is 18059280 \(\mu\)s
\end{verbatim}
}
Ça fonctionne avec une stabilité surprenante (supérieure au stm32 !).


\subsection{Task 5: Using network devices}
\subsubsection{nucleo-f746zg ethernet}
La seule carte avec une interface réseau dont je dispose est la
nucleo-f746zg.\\

Malheureusement ça ne fonctionne pas tout seul : tout semble bien se
passer pendant la compilation mais la commande \texttt{ifconfig} ne
retourne rien.\\

Il y a une page \enquote{Driver for stm32 ethernet} dans la
documentation mais elle est vide. Dans le code source, on trouve le
dossier correspondant \texttt{drivers/stm32\_eth}.\\

Une petite recherche :
\begin{verbatim}
hugues@W520:~/Tutorials/RIOT$ grep -r stm32_eth boards/*
boards/nucleo-f207zg/Makefile.dep:  USEMODULE += stm32_eth
boards/nucleo-f767zi/Makefile.dep:  USEMODULE += stm32_eth
\end{verbatim}

La plus proche des 2 est la nucleo-f767zi.
{\scriptsize
\begin{verbatim}
hugues@W520:~/Tutorials/RIOT$ diff -u boards/nucleo-f746zg/Makefile.features boards/nucleo-f767zi/Makefile.features 
--- boards/nucleo-f746zg/Makefile.features	2020-05-04 17:04:48.253357832 +0200
+++ boards/nucleo-f767zi/Makefile.features	2020-05-04 17:04:48.253357832 +0200
@@ -1,13 +1,16 @@
 CPU = stm32f7
-CPU_MODEL = stm32f746zg
+CPU_MODEL = stm32f767zi
 
 # Put defined MCU peripherals here (in alphabetical order)
+FEATURES_PROVIDED += periph_dma
 FEATURES_PROVIDED += periph_i2c
 FEATURES_PROVIDED += periph_rtc
 FEATURES_PROVIDED += periph_rtt
+FEATURES_PROVIDED += periph_spi
 FEATURES_PROVIDED += periph_timer
 FEATURES_PROVIDED += periph_uart
 FEATURES_PROVIDED += periph_usbdev
+FEATURES_PROVIDED += periph_eth
 
 # Put other features for this board (in alphabetical order)
 FEATURES_PROVIDED += riotboot
\end{verbatim}
}
Le support de la nucleo-f746zg est donc incomplet, en plus de l'ethernet,
on aura aussi besoin du SPI et du PWM (présent pour la nucleo-f207zg).\\

Le stm32f746zg et le stm32f767zi étant très proche, je tente ma chance
en utilisant \texttt{BOARD=nucleo-f767zi make all flash term} dans un
premier temps. Là le \texttt{ifconfig} a l'air de fonctionner :
\begin{verbatim}
2020-05-14 01:36:38,623 # This is Task-05
> ifconfig
2020-05-14 01:36:53,581 #  ifconfig
2020-05-14 01:36:53,585 # Iface  4  HWaddr: 06:17:2A:15:1C:41 
2020-05-14 01:36:53,589 #           L2-PDU:1500 Source address length: 6
2020-05-14 01:36:53,590 #           
\end{verbatim}

Lorsque je connecte un câble réseau je vois immédiatement un tas de
paquets reçus par pktdump.\\


\subsubsection{Communication nucleo-f746zg $\leftrightarrow$ native}
Comme je n'ai qu'une carte, je vais lancer un riot native pour essayer
de les faire communiquer. Il faut d'abord configurer une interface tap
et un bridge :
\begin{verbatim}
root@W520:/home/hugues/Tutorials/RIOT# tunctl -u hugues
Set 'tap0' persistent and owned by uid 1000
root@W520:/home/hugues/Tutorials/RIOT# ifconfig tap0 up
root@W520:/home/hugues/Tutorials/RIOT# brctl addbr br0
root@W520:/home/hugues/Tutorials/RIOT# ifconfig br0 up
root@W520:/home/hugues/Tutorials/RIOT# brctl addif br0 tap0 wlp3s0
can't add wlp3s0 to bridge br0: Operation not supported
\end{verbatim}

J'ai dû débrancher mon portable du réseau filaire pour brancher la
nucleo, il n'est malheureusement pas possible de bridger l'interface
wifi (sauf si elle était en mode AP) pour pouvoir faire communiquer la
carte avec un riot native.\\

Le plus simple aurait été d'utiliser un soit un switch et 2 câbles
réseaux pour tout relier en RJ45, ou simplement un câble croisé plus
économique. On trouvait couramment ce genre de câble avec les premiers
modems ADSL ou les box dépourvues de switch, c'est moins courant
aujourd'hui. Ça se bricole bien aussi en coupant un câble droit en 2,
voir
\url{https://fr.wikipedia.org/wiki/RJ45#C%C3%A2blage_crois%C3%A9_complet}
;-).
\\

Mon routeur maison tourne sous
linux et je l'utilise parfois pour faire du vpn niveau 2 avec ssh.
Je vais donc créer un tunnel ethernet vers mon routeur avec un bridge à
chaque extrémité :
{\scriptsize
\begin{verbatim}
root@W520:/home/hugues/Tutorials/task-05# ssh 192.168.21.254 -o Tunnel=ethernet -w 1:1 -f sleep 1
root@W520:/home/hugues/Tutorials/task-05# ssh 192.168.21.254 ifconfig tap1 up
root@W520:/home/hugues/Tutorials/task-05# ssh 192.168.21.254 brctl addif br0 tap1
root@W520:/home/hugues/Tutorials/task-05# ifconfig tap1 up
root@W520:/home/hugues/Tutorials/task-05# brctl addif br0 tap1
root@W520:/home/hugues/Tutorials/task-05# ifconfig tap0
tap0: flags=4099<UP,BROADCAST,MULTICAST>  mtu 1500
        ether fa:1b:cf:b5:6f:ef  txqueuelen 1000  (Ethernet)
\end{verbatim}
}
J'en ai profité pour récupérer l'adresse ethernet de l'interface qui
sera utilisée par le riot native qu'il ne reste plus qu'à lancer :
\begin{verbatim}
hugues@W520:~/Tutorials/task-05$ BOARD=native make all term PORT=tap0
\end{verbatim}

Maintenant j'essaie d'envoyer un paquet avec
\texttt{txtsnd 4 bcast test}.\\

D’abord de la nucleo $\rightarrow$ native :
{\scriptsize
\begin{verbatim}
PKTDUMP: data received:
~~ SNIP  0 - size:  46 byte, type: NETTYPE_UNDEF (0)
00000000  74  65  73  74  00  00  00  00  00  00  00  00  00  00  00  00
00000010  00  00  00  00  00  00  00  00  00  00  00  00  00  00  00  00
00000020  00  00  00  00  00  00  00  00  00  00  00  00  00  00
~~ SNIP  1 - size:  20 byte, type: NETTYPE_NETIF (-1)
if_pid: 4  rssi: 0  lqi: 0
flags: 0x0
src_l2addr: 06:17:2A:15:1C:41
dst_l2addr: FF:FF:FF:FF:FF:FF
~~ PKT    -  2 snips, total size:  66 byte
\end{verbatim}
}
Puis de native $\rightarrow$ nucleo :
{\scriptsize
\begin{verbatim}
2020-05-14 02:00:16,095 # PKTDUMP: data received:
2020-05-14 02:00:16,100 # ~~ SNIP  0 - size:  46 byte, type: NETTYPE_UNDEF (0)
2020-05-14 02:00:16,106 # 00000000  74  65  73  74  00  00  00  00  00  00  00  00  00  00  00  00
2020-05-14 02:00:16,113 # 00000010  00  00  00  00  00  00  00  00  00  00  00  00  00  00  00  00
2020-05-14 02:00:16,119 # 00000020  00  00  00  00  00  00  00  00  00  00  00  00  00  00
2020-05-14 02:00:16,124 # ~~ SNIP  1 - size:  20 byte, type: NETTYPE_NETIF (-1)
2020-05-14 02:00:16,126 # if_pid: 4  rssi: 0  lqi: 0
2020-05-14 02:00:16,127 # flags: 0x0
2020-05-14 02:00:16,131 # src_l2addr: FA:1B:CF:B5:6F:F0
2020-05-14 02:00:16,133 # dst_l2addr: FF:FF:FF:FF:FF:FF
2020-05-14 02:00:16,136 # ~~ PKT    -  2 snips, total size:  66 byte
\end{verbatim}
}
Enfin un peu de nettoyage dans les config réseau :
{\tiny
\begin{verbatim}
root@W520:/home/hugues/Tutorials/task-05# ssh 192.168.21.254 brctl delif br0 tap1
root@W520:/home/hugues/Tutorials/task-05# ssh 192.168.21.254 ifconfig tap1 down
root@W520:/home/hugues/Tutorials/task-05# brctl delif br0 tap0 tap1
root@W520:/home/hugues/Tutorials/task-05# ifconfig tap1 down
root@W520:/home/hugues/Tutorials/task-05# kill `ps aux | grep 'ssh 192.168.21.254 -o Tunnel=ethernet -w 1:1 -f sleep 1' | grep -v grep | awk '{printf $2}'`
root@W520:/home/hugues/Tutorials/task-05# ifconfig tap0 down
root@W520:/home/hugues/Tutorials/task-05# ifconfig br0 down
root@W520:/home/hugues/Tutorials/task-05# brctl delbr br0
root@W520:/home/hugues/Tutorials/task-05# tunctl -d tap0
Set 'tap0' nonpersistent
\end{verbatim}
}

\subsubsection{Remarques}
Finalement je me suis un peu emballé dans la config réseau, le bridge
local était inutile (je l'avais créé dans l'optique de bridger tap0 et
mon interface wifi), j'aurai pu connecter directement le riot native au
bout du tunnel ethernet ssh (tap1).\\

Aux cours de ces expérimentations j'ai parfois rencontré un problème
après une reprogrammation, un reset, ou même un reboot : seul le voyant
orange du RJ45 clignote, le voyant vert reste éteind, le
\texttt{ifconfig} fonctionne bien mais aucun paquet n'est reçu et il est
impossible d'en émettre. J'ai effectué 10 resets et 10 reboots et 10
flashage d'affilés, dans tous les cas le problème s'est produit 4 fois
sur 10.\\

Le problème est connu chez RIOT pour la nucleo-f767zi :\\
\url{https://github.com/RIOT-OS/RIOT/issues/13490}.\\

On peut trouver le même genre de problème sur la nucleo-f746zg avec Mbed
OS décrit ici :\\
\url{https://os.mbed.com/questions/77138/EthernetInterface-unreliable-on
-NUCLEO-F/}.\\

Et là avec lwIP \enquote{Bare Metal} sauf que ça fonctionne sur la
nucléo et pas sur une autre carte de dev :\\
{\scriptsize\url{https://stackoverflow.com/questions/55260931/stm32h7-
lan8742-lwip-only-works-fine-after-power-up-not-after-reset}}.


\subsection{Task 6: UDP Client / Server}
\subsubsection{native $\leftrightarrow$ native}
Création du réseau virtuel :
\begin{verbatim}
root@W520:~# tunctl -u 1000
Set 'tap0' persistent and owned by uid 1000
root@W520:~# tunctl -u 1000
Set 'tap1' persistent and owned by uid 1000
root@W520:~# brctl addbr br0
root@W520:~# brctl addif br0 tap0 tap1
root@W520:~# ifconfig br0 up
root@W520:~# ifconfig tap0 up
root@W520:~# ifconfig tap1 up
\end{verbatim}

La communication entre les 2 riot natifs a fonctionné comme prévue.\\

\subsubsection{native $\leftrightarrow$ linux}
La commande netcat sur ma machine n'implémentait pas l'IPv6 :
{\footnotesize
\begin{verbatim}
hugues@W520:~/Tutorials/task-06$ echo "hello" | nc -6u fe80::ac51:57ff:fe42:8286%br0 8888
nc: invalid option -- '6'
nc -h for help
\end{verbatim}
}
J'y ai donc installé le package \texttt{netcat-openbsd}.\\

J'ai utilisé \texttt{br0} au lieu de \texttt{tapbr0} dans la commandes
linux.\\

Dans ma configuration br0 a la même adresse IPv6 que tap1, c'est donc
cette adresse que j'ai du utiliser pour envoyer un message de RIOT 
$\rightarrow$ l'hôte linux.\\

Avec la commande \texttt{nc -6lu 8888} je ne pouvais recevoir q'un seul
message, après quoi elle restait active mais ne recevait plus rien. On
peut recevoir plusieurs messages de suite en ajoutant l'option
\enquote{k}.\\

\subsubsection{nucleo $\leftrightarrow$ linux}
D'abord je récupère l'adresse IPv6 de l'interface wifi de mon portable 
et je lance un serveur UDP avec netcat :
{\scriptsize
\begin{verbatim}
hugues@W520:~/Tutorials/task-06$ /sbin/ifconfig wlp3s0
wlp3s0: flags=4163<UP,BROADCAST,RUNNING,MULTICAST>  mtu 1500
        inet 192.168.21.32  netmask 255.255.255.0  broadcast 192.168.21.255
        inet6 2a01:e34:ed4f:42d0:a054:e4fc:dea1:f876  prefixlen 64  scopeid 0x0<global>
        inet6 fe80::9588:ffe2:5b3e:7c2b  prefixlen 64  scopeid 0x20<link>
        ether 08:11:96:24:86:9c  txqueuelen 1000  (Ethernet)
        RX packets 62172  bytes 19221714 (18.3 MiB)
        RX errors 0  dropped 0  overruns 0  frame 0
        TX packets 56365  bytes 8802188 (8.3 MiB)
        TX errors 0  dropped 0 overruns 0  carrier 0  collisions 0
hugues@W520:~/Tutorials/task-06$ nc -6luk 8888
\end{verbatim}
}
Il faut prendre la seconde (\texttt{<link>}) : 
\texttt{fe80::9588:ffe2:5b3e:7c2b}.\\

Puis je flash la nucleo, j'essaie un ping vers mon portable et un petit
\enquote{\texttt{hello}} :
{\scriptsize
\begin{verbatim}
> ping6 fe80::9588:ffe2:5b3e:7c2b
2020-05-15 00:11:50,930 #  ping6 fe80::9588:ffe2:5b3e:7c2b
2020-05-15 00:11:50,967 # 12 bytes from fe80::9588:ffe2:5b3e:7c2b%5: icmp_seq=0 ttl=64 time=30.498 ms
2020-05-15 00:11:51,938 # 12 bytes from fe80::9588:ffe2:5b3e:7c2b%5: icmp_seq=1 ttl=64 time=1.171 ms
2020-05-15 00:11:52,938 # 12 bytes from fe80::9588:ffe2:5b3e:7c2b%5: icmp_seq=2 ttl=64 time=1.153 ms
2020-05-15 00:11:52,938 # 
2020-05-15 00:11:52,942 # --- fe80::9588:ffe2:5b3e:7c2b PING statistics ---
2020-05-15 00:11:52,947 # 3 packets transmitted, 3 packets received, 0% packet loss
2020-05-15 00:11:52,951 # round-trip min/avg/max = 1.153/10.940/30.498 ms
> udp fe80::9588:ffe2:5b3e:7c2b 8888 hello
2020-05-15 00:16:03,049 #  udp fe80::9588:ffe2:5b3e:7c2b 8888 hello
2020-05-15 00:16:03,054 # Success: send 5 byte to fe80::9588:ffe2:5b3e:7c2b
\end{verbatim}
}
Sur le portable j'obtiens bien :
\begin{verbatim}
hugues@W520:~/Tutorials/task-06$ nc -6luk 8888
hello
\end{verbatim}


\subsection{Task 7: The GNRC network stack}
C'est un peu la même chose que les précédents en utilisant les
exemples d'applications du dépôt.

\subsection{Task 8: CCN-Lite on RIOT}
CCN: Content Centric Networking.\\
\url{https://en.wikipedia.org/wiki/Content_centric_networking}

\subsection{Task 9: RIOT and RPL}
RPL: IPv6 Routing Protocol for Low-Power and Lossy Networks.\\
Ça aussi on va se le garder pour plus tard.

\end{document}
