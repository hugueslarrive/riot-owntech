\documentclass[a4paper,12pt, twoside]{article}

\usepackage[T1]{fontenc}
\usepackage[utf8]{inputenc}
%~ \usepackage{lmodern}
\usepackage{helvet}
\renewcommand{\familydefault}{\sfdefault}
\usepackage[francais]{babel}
\usepackage{verbatim}
\usepackage{tikz}
\usepackage{eurosym} % pour € \euro{}
\usepackage{csquotes} % pour les guillemets

%\usepackage{newtxmath}

\usepackage[
  % Some remarks:
  % * drivers like 'pdftex' that can be detected automatically
  %   are not necessary
  % * breaklinks is rather an internal option.
  %   If a driver does not support it, then forcing the option
  %   let the text break across lines, but also the link
  %   areas are "broken". If the driver supports the option,
  %   then the option is enabled anyway.
  % * Information entries should be set outside,
  %   because LaTeX expands the package options,
  %   hyperref does not like them, if they are
  %   prematurely expanded.
  % * Hyperref has a new option for hiding links: hidelinks
  hidelinks,
  pagebackref,
  bookmarksopen,
  bookmarksnumbered,
  a4paper,
]{hyperref}
\hypersetup{
  pdfauthor={Hugues Larrive},
  % ...
}
% Adding package bookmark improves bookmarks handling.
% More features and faster updated bookmarks.
\usepackage{bookmark}

\usepackage{layout}
\oddsidemargin=-1cm
\usepackage[top=2cm, bottom=2cm, left=2cm, right=2cm]{geometry}

\usepackage{wrapfig}
\usepackage{graphicx}


\usepackage{listings}
\lstset{
     literate=%
		{à}{{\`a}}1
		{é}{{\'e}}1
		{ê}{{\^e}}1
		{è}{{\`e}}1
		{É}{{\'E}}1
		{ç}{{\c{c}}}1
}
\usepackage{color}

\definecolor{mygreen}{rgb}{0,0.6,0}
\definecolor{mygray}{rgb}{0.5,0.5,0.5}
\definecolor{mymauve}{rgb}{0.58,0,0.82}

\lstset{ 
  backgroundcolor=\color{white},   % choose the background color; you must add \usepackage{color} or \usepackage{xcolor}; should come as last argument
  basicstyle=\footnotesize,        % the size of the fonts that are used for the code
  breakatwhitespace=false,         % sets if automatic breaks should only happen at whitespace
  breaklines=true,                 % sets automatic line breaking
  captionpos=b,                    % sets the caption-position to bottom
  commentstyle=\color{mygreen},    % comment style
  deletekeywords={...},            % if you want to delete keywords from the given language
  escapeinside={\%*}{*)},          % if you want to add LaTeX within your code
  extendedchars=true,              % lets you use non-ASCII characters; for 8-bits encodings only, does not work with UTF-8
  frame=single,	                   % adds a frame around the code
  keepspaces=true,                 % keeps spaces in text, useful for keeping indentation of code (possibly needs columns=flexible)
  keywordstyle=\color{blue},       % keyword style
  language=Octave,                 % the language of the code
  morekeywords={*,...},            % if you want to add more keywords to the set
  numbers=left,                    % where to put the line-numbers; possible values are (none, left, right)
  numbersep=5pt,                   % how far the line-numbers are from the code
  numberstyle=\tiny\color{mygray}, % the style that is used for the line-numbers
  rulecolor=\color{black},         % if not set, the frame-color may be changed on line-breaks within not-black text (e.g. comments (green here))
  showspaces=false,                % show spaces everywhere adding particular underscores; it overrides 'showstringspaces'
  showstringspaces=false,          % underline spaces within strings only
  showtabs=false,                  % show tabs within strings adding particular underscores
  stepnumber=1,                    % the step between two line-numbers. If it's 1, each line will be numbered
  stringstyle=\color{mymauve},     % string literal style
  tabsize=2%,	                   % sets default tabsize to 2 spaces
  %title=\lstname                   % show the filename of files included with \lstinputlisting; also try caption instead of title
}

\lstdefinestyle{customc}{
  belowcaptionskip=1\baselineskip,
  breaklines=true,
  frame=L,
  xleftmargin=\parindent,
  language=C,
  showstringspaces=false,
  basicstyle=\footnotesize\ttfamily,
  keywordstyle=\bfseries\color{green!40!black},
  commentstyle=\itshape\color{purple!40!black},
  identifierstyle=\color{blue},
  stringstyle=\color{orange},
}

\lstset{escapechar=£,style=customc}

\usepackage{algorithm2e}
\usepackage{pdfpages}
\usepackage{gensymb} % for \degree
%\usepackage{setspace}

\usepackage{caption}
\usepackage{booktabs}
\usepackage{multirow}
\newcommand{\MR}[2]{\multirow{#1}{*}{#2}}			%simplifies multirow command - must use the multirow package
\newcommand{\MC}[2]{\multicolumn{#1}{c}{#2}}			%simplifies the multicollumn command - must use the multicollumn package





\usepackage{amssymb}

%\usepackage[firstpage]{draftwatermark}

\usepackage{alltt}

\usepackage{pifont}% http://ctan.org/pkg/pifont
\newcommand{\cm}{\textcolor{green}{\ding{51}}}%
\newcommand{\xm}{\textcolor{red}{\ding{55}}}%


\title{Préparation pour la compilation de RIOT OS sous Ubuntu 20.04 ou Debian 10
(dépendances)}
\author{Hugues Larrive <hlarrive@laas.fr>}
\date{\today}	%defines the date of the document - leave empty for no date


\begin{document}

\maketitle{}

\vspace*{\stretch{1}}

\begin{abstract}
    Les pages \url{https://github.com/RIOT-OS/RIOT/wiki/Creating-your-first-RIOT-project}
    et \url{https://github.com/RIOT-OS/Tutorials} pointent toutes les
    deux vers
    \url{https://github.com/RIOT-OS/RIOT/wiki/Family:-native#dependencies}
    qui traite de l'installation des dépendances sur des versions
    obsolètes comme Debian 7.5 et Ubuntu 14.10 !\\
    
    Voici donc un court tuto pour la compilation de RIOT OS sur les 
    versions actuelles de ces systèmes.
\end{abstract}

\vspace*{\stretch{1}}

{\footnotesize
\begin{verbatim}
    Author: Hugues Larrive <hugues.larrive@laas.fr>

	Copyright (C) 2020 LAAS-CNRS.
	Permission is granted to copy, distribute and/or modify this document
	under the terms of the GNU Free Documentation License, Version 1.3
	or any later version published by the Free Software Foundation;
	with no Invariant Sections, no Front-Cover Texts, and no Back-Cover Texts.
	A copy of the license is included in the section entitled "GNU
	Free Documentation License".
\end{verbatim}
}

\newpage
%\cleardoublepage

\pdfbookmark[section]{\contentsname}{toc}
\renewcommand{\contentsname}{Sommaire}
\tableofcontents{}


\section{Architectures}
RIOT OS supporte de nombreuse architectures matériel comme stm32, avr,
mips, xtensa, msp430, etc.\\

En plus de ces architectures, il offre la possibilité de compiler un
projet sous la forme d'un binaire elf 32bits exécutable sous linux pour
effectuer des tests (architecture \enquote{native}).\\

Les PC actuels étant d'architecture 64 bits, il est nécessaire d'activer
le support multi-architecture sur les systèmes Debian / Ubuntu :
\begin{verbatim}
sudo dpkg --add-architecture i386
\end{verbatim}

Cette commande ne retourne rien mais on peut vérifier les architectures
supportées. La commande \texttt{dpkg ---print-architecture} retourne
l'architecture principale du système et la commande
\texttt{dpkg ---print-foreign-architectures} retourne les autres
architectures activées par \texttt{---add-architecture}.\\

Un fois l'architecture activée il est nécessaire de mettre à jour la
base de données des packages :
\begin{verbatim}
sudo apt-get update
\end{verbatim}

Après ça on peut installer un package d'architecture i386 en ajoutant
\enquote{:i386} après son nom dans une commande apt-get.

\section{Installation des dépendances}
\subsection{Essentiel et architecture native}
\begin{verbatim}
sudo apt-get install libc6-dev-i386 libc6-dbg:i386 build-essential pkg-config \
uml-utilities bridge-utils git unzip
\end{verbatim}
\subsection{Pour l'architecture stm32}
\begin{verbatim}
sudo apt-get install openocd gcc-arm-none-eabi
\end{verbatim}

\section{Test}
Télécharger RIOT OS et créer un projet comme indiqué dans 
\url{https://github.com/RIOT-OS/RIOT/wiki/Creating-your-first-RIOT-project} :
\begin{verbatim}
git clone https://github.com/RIOT-OS/RIOT RIOT
cd RIOT/examples
cp -R default my_project
cd my_project
\end{verbatim}

Le compiler :
{\small
\begin{verbatim}
hugues@debian10base:~/RIOT/examples/my_project$ make
Building application "default" for "native" with MCU "native".

"make" -C /home/hugues/RIOT/boards/native
"make" -C /home/hugues/RIOT/boards/native/drivers
"make" -C /home/hugues/RIOT/core
"make" -C /home/hugues/RIOT/cpu/native
"make" -C /home/hugues/RIOT/cpu/native/netdev_tap
"make" -C /home/hugues/RIOT/cpu/native/periph
"make" -C /home/hugues/RIOT/cpu/native/stdio_native
"make" -C /home/hugues/RIOT/cpu/native/vfs
"make" -C /home/hugues/RIOT/drivers
"make" -C /home/hugues/RIOT/drivers/netdev_eth
"make" -C /home/hugues/RIOT/drivers/periph_common
"make" -C /home/hugues/RIOT/drivers/saul
"make" -C /home/hugues/RIOT/drivers/saul/init_devs
"make" -C /home/hugues/RIOT/sys
"make" -C /home/hugues/RIOT/sys/auto_init
"make" -C /home/hugues/RIOT/sys/fmt
"make" -C /home/hugues/RIOT/sys/iolist
"make" -C /home/hugues/RIOT/sys/net/gnrc
"make" -C /home/hugues/RIOT/sys/net/gnrc/netapi
"make" -C /home/hugues/RIOT/sys/net/gnrc/netif
"make" -C /home/hugues/RIOT/sys/net/gnrc/netif/ethernet
"make" -C /home/hugues/RIOT/sys/net/gnrc/netif/hdr
"make" -C /home/hugues/RIOT/sys/net/gnrc/netif/init_devs
"make" -C /home/hugues/RIOT/sys/net/gnrc/netreg
"make" -C /home/hugues/RIOT/sys/net/gnrc/pkt
"make" -C /home/hugues/RIOT/sys/net/gnrc/pktbuf
"make" -C /home/hugues/RIOT/sys/net/gnrc/pktbuf_static
"make" -C /home/hugues/RIOT/sys/net/gnrc/pktdump
"make" -C /home/hugues/RIOT/sys/net/link_layer/l2util
"make" -C /home/hugues/RIOT/sys/net/netif
"make" -C /home/hugues/RIOT/sys/od
"make" -C /home/hugues/RIOT/sys/phydat
"make" -C /home/hugues/RIOT/sys/ps
"make" -C /home/hugues/RIOT/sys/saul_reg
"make" -C /home/hugues/RIOT/sys/shell
"make" -C /home/hugues/RIOT/sys/shell/commands
   text	   data	    bss	    dec	    hex	filename
  91657	   1288	  72088	 165033	  284a9	/home/hugues/RIOT/examples/my_projec
t/bin/native/default.elf
\end{verbatim}
}


%\bibliographystyle{plain}
%\bibliography{tests_d_rtos}


\end{document}
