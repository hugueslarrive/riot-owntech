\subsection{Tableau de synthèse de la présélection}
{\footnotesize
\begin{tabular}{lcccccc}
\toprule

		 & Licence		& Dépôt	 & Code style  & Doxygen	& Plateforme cible 	& Gestionnaire		\\
\midrule
\MR{2}{ChibiOS} & Apache 2.0 & \MR{2}{svn}	& K\&R	  &	\MR{2}{oui}	& \MR{2}{nucleo-f746zg}	& Giovanni		\\
				& GPLv3	     &				& modifié &				&						& Di Sirio (IT)	\\
FreeRTOS & MIT (Expat)  & github & MISRA	   &	non		& ARM\_CM7			& Amazon AWS (US)		\\
NuttX	 & Apache 2.0	& github & NuttX	   &	non		& nucleo-f746zg		& Apache ASF (US)		\\
\MR{2}{RIOT} & \MR{2}{LGPL 2.1}  & \MR{2}{github} & \MR{2}{Linux} &	\MR{2}{oui}	& \MR{2}{nucleo-f746zg} & Freie Universität \\
			 &					 &				  &				  &				&						& Berlin (DE)		\\
RTEMS	 & GPLv2/...	& git	 & RTEMS	   &	oui		& ARMv7-M			& OAR Corp. (US)	\\
Mynewt	 & Apache2/...	& github & Mynewt	   &	oui		& nucleo-f746zg		& Apache ASF (US)	\\
Zephyr	 & Apache 2.0	& github & Linux	   &	oui		& nucleo-f746zg		& Linux Foundation (US)	\\
\bottomrule
\end{tabular}
}
\subsubsection{Licences}
On retrouve fréquemment la licence Apache version 2.0 qui semble être la plus
adaptée pour ce genre de projet et un bon compromis entre la GPL trop restrictive
et la licence MIT trop permissive.\\

La solution employée par ChibiOS avec la HAL sous licence Apache 2.0 et le noyau sous
licence GPLv3 est également intéressante car elle est permissive juste là où c'est
nécessaire.

\subsubsection{Code style}
Du moment que c'est homogène et bien défini, ce n'est pas déterminant dans la mesure
où nous n'avons pas de code préexistant à porter. 

\subsubsection{Type de dépôt}
SVN (Apache Subversion) est une solution de versionning centralisée qui peut poser
problème en cas de panne du serveur et ne permet pas de travailler hors ligne.\\

Le fait que NuttX et Mynewt utilisent git montre que l'ASF n'est pas sectaire...\\

Ce n'est pas un critère déterminant.

\subsubsection{Doxygen}
Le fait qu'un projet utilise ou non Doxygen ne présage pas de la qualité de sa
documentation. Ce n'est donc pas non-plus un critère déterminant.\\

\subsubsection{Plateformes cible}
C'est plus du domaine de la toolchain et de l'environnement de développement que de
l'OS.\\

ChibiOS par exemple fournit l'environnement de développement libre ChibiStudio basé
sur Eclipse, GCC, OpenOCD et inclut donc des BSP (Board Support Package).\\

FreeRTOS au contraire s'appuie sur des outils propriétaires comme STM32Cube et leur
laisse donc développer et gérer le support pour leur matériel. Ces outils ne sont
généralement disponible que pour windows ce qui représente un inconvénient majeur
pour les développeurs de logiciels libres... Cela oblige également à changer d'outils
en cas de changement d'architecture ce qui complique le portage.\\

NuttX, RIOT, et Zephyr intègrent l'outil Kconfig de Linux que je trouve magnifique.\\

RTEMS et Mynewt fournissent leurs propres outils qui semblent être des genres de
\enquote{ package manager }.\\

Beaucoup d'électroniciens travaillent sous windows et seront probablement plus
attirés par ChibiOS ou FreeRTOS.\\

Les informaticiens et développeurs de logiciels libres seront plus attirés par les
autres.\\

\subsubsection{Gestionnaire}
Certains projets sont gérés par un individu, d'autres par des sociétés commerciales,
d'autres par des organisations communautaires, certains sont américains et d'autres
européens... quels sont les implications en termes d'éthique et de pérennité ?

\subsection{Exigences pour un système d'exploitation temps réel}
%Requirements for a real time operating system
%The OS must be Hard real time and support:
%- Task management
%- Task communication
%- Semaphores
%- Networking
%- File storage
(Suggérées par Clément)\\
L'OS doit supporter :
\begin{itemize}
	\item le temps réel strict ;
	\item la gestion des tâches ;
	\item la communication entre tâches ;
	\item les sémaphores ;
	\item la communication réseau ;
	\item le stockage de fichiers.\\
\end{itemize}

{\footnotesize
\begin{tabular}{lcccccc}
\toprule

		 & temps 	   & gestion	& communication & \MR{2}{sémaphores} & communication & stockage		\\
		 & réel strict & des tâches & entre tâches  &					 & réseau		 & de fichiers	\\
\midrule
ChibiOS	 &		oui	   &	oui		&		oui	 	&		oui			 &		lib		 &		lib		\\
FreeRTOS &		oui	   &	oui		&		oui	 	&		oui	 		 &		lib		 &		lib		\\
NuttX	 &		oui	   &	oui		&		oui	 	&		oui			 &		oui		 &		oui		\\
RIOT	 &		oui	   &	oui		&		oui	 	&		oui			 &		oui		 &		oui		\\
RTEMS	 &		oui	   &	oui		&		oui	 	&		oui			 &		oui		 &		oui		\\
Mynewt	 &		oui	   &	oui		&		oui	 	&		oui			 &		oui		 &		oui		\\
Zephyr	 &		oui	   &	oui		&		oui	 	&		oui			 &		oui		 &		oui		\\
\bottomrule
\end{tabular}
}
\\\\

Il apparaît que ChibiOS et FreeRTOS intègrent uniquement des services de bas niveau,
un peu à la manière de mach\_kernel dans XNU de OSX/IOS. Il est probablement plus
confortable de travailler avec un système dont les outils gèrent directement les
services de haut niveau dont nous aurons besoin (à définir).\\

%\subsection{En vrac}
%\subsubsection{Réseau}
%\begin{itemize}
%	\item IPv4
%	\item IPv6
%	\item LoRaWAN
%	\item MQTT
%\end{itemize}

%\subsubsection{Flash Circular Buffer}
%Au hasard de mes recherches dans la dcumentation de Mynewt je suis tombé sur cette
%page :\\

%\url{https://mynewt.apache.org/latest/os/modules/fcb/fcb.html}.\\

%Je pense que ce type de buffer peut être utilisé en cas de difficulté à initier un
%enregistrement de données en temps réel suite à un événement.

\subsection{Remarques concernant RTEMS}
Remarque de Clément :
\enquote{J'ai utilisé à l'époque (j'avais fait un portage hardware de l'OS), le code
est illisible si tu veux rentrer dans le noyau. Au niveau utilisateur, je sais plus
trop.}\\

En ce qui me concerne, même l'organisation du dépôt me semble difficile à
appréhender...\\

Son origine militaire (commande de missile) me dérange un peu aussi.

\subsection{Nouvelle sélection}
Au vu de ces éléments les 4 meilleurs candidats sont :
\begin{itemize}
	\item NuttX
	\item RIOT
	\item Mynewt
	\item Zephyr\\
\end{itemize}

\subsection{Tests}
Les tests approfondi d'un RTOS prenant beaucoup de temps il est probable que le
premier testé soit retenu s'il convient.\\

Parmi les 4 options restantes, RIOT se démarque par son origine européenne et
universitaire \cite{ref11} :
\begin{verbatim}
Il a été initialement développé par l'université libre de Berlin, l'Institut
national de recherche en informatique et en automatique (INRIA) et l'université
de sciences appliquées d'Hambourg (HAW Hamburg).
\end{verbatim} 

C'est pour cette raison qu'il a été décidé de manière collégiale qu'il serait le
premier testé.\\

Ces tests feront l'objet d'un autre document.
