\subsubsection{Licences}
Principalement GNU GPLv2 modifiée :
{\small
\begin{verbatim}
As a special exception, including RTEMS header files in a file, instantiating RTEMS
generics or templates, or linking other files with RTEMS objects to produce an
executable application, does not by itself cause the resulting executable application
to be covered by the GNU General Public License. This exception does not however
invalidate any other reasons why the executable file might be covered by the GNU
Public License.
\end{verbatim}\\
}
Two-Clause BSD license : \enquote{Completely new submissions are encouraged to use
this license.}

\subsubsection{Code source}
Dépot git sur le site : \url{https://git.rtems.org/}.\\

Les commentaires comportent des tags doxygen.\\

Il y a un standard de codage :
\url{https://docs.rtems.org/branches/master/eng/coding.html}.

\subsubsection{Plateformes cible}
Pas précisément le STM32F746ZG mais on peut lire \enquote{STMicroelectronics STM32 F4
and STM32F105} sur la page
\url{https://devel.rtems.org/wiki/TBR/Website/Board\_Support\_Packages}.

\subsubsection{Communauté}
La communauté est active.\\

Le site internet \url{https://www.rtems.org/} comporte 6 pages :
\begin{itemize}
	\item Home ;
	\item Wiki ;
	\item Git ;
	\item Training ;
	\item News Archive ;
	\item Licenses.\\
\end{itemize}

Il y a une boîte \enquote{Community} sur la gauche de la page d'accueil :
\begin{itemize}
	\item Mailing Lists ;
	\item IRC \#rtems (freenode) + logs sur rtems.org ;
	\item Linkedin ;
	\item Twitter ;
	\item Facebook.\\
\end{itemize}

Il est utilisé dans l'industrie spatiale, notamment par les acteurs européens du
domaine selon l'article Wikipédia.

\subsubsection{Acceptabilité}
\begin{tabular}{lll}
\toprule
	Critère				&	Validé		&	Commentaire	\\
\midrule
	Licence				&	oui			&		\\
	Code source			&	oui			&		\\
	Plateformes cible	&	oui			&	STM32F4	\\
	Communauté			&	oui			&		\\
\bottomrule
\end{tabular}

