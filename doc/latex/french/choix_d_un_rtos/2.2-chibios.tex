\subsubsection{Licence}
Apache 2.0 pour la HAL et GNU GPLv3 pour le reste.\\

La licence Apache 2.0 est reconnue comme une licence de logiciel libre par le projet
GNU \cite{ref2}

\subsubsection{Code source}
Dépot Subversion : \url{https://osdn.net/projects/chibios/scm/svn/tree/head/}.\\

Il semble très fourni et bien organisé, un fichier readme.txt à la racine montre
cette organisation :
\begin{verbatim}
*****************************************************************************
*** ChibiOS products directory organization                               ***
*****************************************************************************

--{root}                - Distribution directory.
  +--os/                - ChibiOS products, this directory.
  |  +--rt/             - ChibiOS/RT product.
  |  |  +--include/     - RT kernel headers.
  |  |  +--src/         - RT kernel sources.
  |  |  +--templates/   - RT kernel port template files.
  |  |  +--ports/       - RT kernel port files.
  |  |  +--osal/        - RT kernel OSAL module for HAL interface.
  |  +--nil/            - ChibiOS/NIL product.
  |  |  +--include/     - Nil kernel headers.
  |  |  +--src/         - Nil kernel sources.
  |  |  +--templates/   - Nil kernel port template files.
  |  |  +--ports/       - Nil kernel port files.
  |  |  +--osal/        - Nil kernel OSAL module for HAL interface.
  |  +--hal/            - ChibiOS/HAL product.
  |  |  +--include/     - HAL high level headers.
  |  |  +--src/         - HAL high level sources.
  |  |  +--templates/   - HAL port template files.
  |  |  +--ports/       - HAL port files (low level drivers implementations).
  |  |  +--boards/      - HAL board files.
  |  +--common/         - Files used by multiple ChibiOS products.
  |  |  +--ports        - Common port files for various architectures and
  |  |                    compilers.
  |  +--various/        - Various portable support files.
  |  +--ext/            - Vendor files used by ChibiOS products.
\end{verbatim}

Les commentaires comportent des tags doxygen.\\

Le code est propre, clair, et homogène mais ne respecte pas les \enquote{GNU Coding
Standards}\cite{ref4} (moi non-plus mais je voulais m'y mettre).\\

\begin{verbatim}
ChibiOS follows the K&R style indentation style with few modifications: Only two
spaces are used for indentation. TAB characters are forbidden. Non UTF-8 characters
are forbidden. EOL must be CRLF (windows convention). The else statement goes to the
line after the closing } .
\end{verbatim}

\subsubsection{Plateformes cible}
De nombreuses cartes STM32 dont la ST\_NUCLEO144\_746ZG.

\subsubsection{Communauté}
La communauté semble importante au vu de la quantité de matériel supporté.\\

Le projet est actif (environ 2 ou 3 commits par jours).\\

Le site internet \url{http://www.chibios.org} comporte 6 pages :
\begin{itemize}
	\item Home ;
	\item Products ;
	\item Downloads ;
	\item Documentation ;
	\item Articles ;
	\item Licensing.\\
\end{itemize}

La page \enquote{Documentation} offre un livre \enquote{ChibiOS/RT 3.0 - The Ultimate
Guide}, et des manuels de références complets pour les version 20.3 et 19.1 ainsi que
ceux de 5 versions obsolètes.\\

La page \enquote{Articles} commence par une section \enquote{Getting started} qui
contient 13 articles dont 12 incluant STM32 dans le titre.\\

Il y a aussi un wiki technique, un forum de support, un canal IRC freenode.

\subsubsection{Acceptabilité}
\begin{tabular}{lll}
\toprule
	Critère				&	Validé		&	Commentaire	\\
\midrule
	Licence				&	oui			&		\\
	Code source			&	oui			&		\\
	Plateformes cible	&	oui			&		\\
	Communauté			&	oui			&		\\
\bottomrule
\end{tabular}

