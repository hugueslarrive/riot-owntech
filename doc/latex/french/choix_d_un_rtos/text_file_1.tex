\section{Referencing with Latex}\label{section_ref}

Latex packs a great punch when it come to the automatic structuring of a text. 
It can handle referencing automatically, which will be given in the sub-section below.

	\subsection{Referencing}
	\begin{mytable}{Fréquences des canaux dans la bande 863-865MHz}{tbl_canaux_863}{c c}
		\tables{canaux_863-865.tex}
	\end{mytable}
	As said in section \ref{section_ref} in page \pageref{section_writing}, this document can be changed at will to fit your needs.
	But you will also find out that you reference figure \ref{fig_panneau} in page \pageref{fig_panneau} before it appears. 

		\subsubsection{Test subsubsection}
		blalbanbaodinadbs 
	
\section{Adding content with Latex}\label{section_content}

One of the main purposes of this template is to help you inserting content easily in Latex. 
Generally speaking, the content you will insert will either be tables with data, equations or figures. 
Templates for each of these can be found at the bottom of this .tex file. 
Since they are all commented, they will not appear in the final .pdf file.

	\subsection{Inserting a figure in Latex}
	
	The template below is based on the \dbquote{myfigure} environment. 
	In this environment, you can easily define the caption of the figure, its label and its size. 
	
	The caption is visible below the figure, its numbering is handled automatically by the latex compiler.
	This figure has the number \ref{fig_panneau}, which can be referenced easily in the text by the ref command in the text.
	
	The \textbf{figs} command used in the example below was defined in the \textbf{definition\_and\_includes\_1.tex} file.
	Note that to write the character \dbquote{\_} you will need to add a bar since the underscore is used for other applications in latex.
	
	Test \_ \& \* \% test	
	
	This command searches for any given file name within the file path \dbquote{/1-images/}.
	In the case below, the file name is \dbquote{pdf\_figure\_1.pdf}.
	
	Since the original image of the figure below is an A4 sized pdf file, the figure must be shrunk to fit in the file.
	The number in front of the \textbf{figs} command gives the shrinking of the file in a scale from 0 to 1, 
	where 1 is the original size of the inserted file.  
	
	\begin{itemize}
		\item test
		\item test2
		\item test3
	\end{itemize}
	
	
	\begin{myfigure}{Example of a figure caption}{fig_panneau}    
%			\figs{0.5}{test_1.pdf}
	\end{myfigure}		
	
	
	\subsection{Inserting two figures side-by-side in Latex}
	
	This example is very similar to the insert one figure above.
	The only change is the \dbquote{mbox} environment which aligns the figures for you, as you can see in figure \ref{fig_2_figures}.
	You can also reference the figures \ref{fig_2_figures_1} and \ref{fig_2_figures_2} individually. 
	
	\begin{myfigure}{Example of a global figure caption}{fig_2_figures}    
		\mbox{	
%			\subfigure[This is a pdf figure		\label{fig_2_figures_1}]{\figs{0.15}{pdf_figure_1.pdf}} \quad
%			\subfigure[This is a pdf figure		\label{fig_2_figures_1}]{\figs{0.15}{maison_test.png}} \quad
%			\subfigure[This is another pdf figure	\label{fig_2_figures_2}]{\figs{0.15}{pdf_figure_1.pdf}}
		} \\
	\end{myfigure}		
	
	
	\subsection{Inserting an equation}
	
	Latex has its own equation environment as shown below. 
	Once inside this environment, you can use the latex syntax to write equations which have a very good rendering as shows equation \ref{eq_equation_label}.
	More details on how to write complex equations are available on the Internet. 
	
	\begin{equation}
		Variable_{1} = \dfrac{Variable_{2} \cdot Variable_{3}}{Variable_{four}}   
		\label{eq_equation_label}
	\end{equation}
	
	
	\subsection{Inserting a table}
	
	You will notice that by far the most complex task in latex is to insert tables. 
	I will not lie to you, this can be tough and frustrating at times, specially if you have a lot of data to write down.
	However, this fact also has very interesting repercussions in your approach to using tables.
	You will notice that if you can put your data on an image, it will be much better for you and comprehensible for your audience.
	
	To help you cope with the challenge of doing tables in latex, I propose the template below. 
	Its main idea is that you write you table on a separate .tex file and then introduce it on the \dbquote{mytable} environment. 
	The catch here is that you will declare the width of the columns OUTSIDE of your table, which can be tricky. 
	Still I think this is a much easier way of making tables than what you can find out there. 
	
	Here is the example of a simple table.

	%Important note : 
	% M{5cm} defines a column of 5 cm with the text centered
	% P{3cm} defines a column of 3 cm with the text on the left
	% c defines a column with an automatic size according to the text with the text on the left

	\begin{mytable}{Example of a simple table}{tbl_table_simple}{c c c c c}
%		\tables{table_example_3.tex}
	\end{mytable}
	
	Here is the example of a more elaborated table.
	
	\begin{mytable}{Example of a more elaborated table}{tbl_table_complicated}{M{2cm} M{3cm} M{3cm} M{3cm}}
%		\tables{table_example_1.tex}
	\end{mytable}
	
	Notice the absence of vertical lines in the tables.
	This is style gives tables that are less aggressive visually, but the lack of vertical lines calls for even more caution when building tables. 
	

\section{Citations}

Latex is also very good for citing authors. For example, the work done in \cite{Meyer} will appear automatically at the Reference section in the end of your document.
This is quite useful, specially for scientific documents. 
The standard used in this template is the IEEE, but this can also be changed if necessary.
	
\section{General recommendations and final thoughts}

Using Latex is changing your text-writing philosophy.
You will no longer worry about the format of your text but rather its content, since Latex handles the whole formatting process for you.
It implies in stopping seeing you text as you do on Word and start thinking about how your text will look like. 
This means that you will need a front-end software which will help you writing down your text. 

My recommendation in Windows is to use Texnick Center (I'm not sure if that's spelled right). 
You'll find the installation procedures on the Internet. 

The mindset behind this template is that you only edit the files that contain the text that will be compiled. 
The other includes should be kept as they are, along with the placement and name of the folders of this suit. 
As long as you move everyone around together, stock your images in the 1-image folder, edit your tables in the 2-tables folder, this template should compile with no problems. 

I hope you enjoy writing in Latex and that you find this example useful for you.

All the best and good luck.




	
	
	
	

\newpage


%-----------------------------GENERAL TEMPLATES---------------------------------------------
% Description - The general templates can be used to quickly add new figures, tables or equations
%-----------------------------------------------------------------------------------------------
% %----------------------------------ONE FIGURE TEMPLATE-------------------------------------------------------------
% % Description - Introduces 1 figure in the text. Replace name_of_figure by the name of the file you wish to include.
% %            	Remember that all figure files should be in the /1-images/ folder
% % 		fig_figure_label is the label of the figure
% % 		the number in \figs{0.3} is the size of the figure in comparison with the original
% %---------------------------------------------------------------------------------------------------------------------
% 
%	\begin{myfigure}{Figure caption}{fig_figure_label}    
%			\figs{0.3}{name_of_figure.pdf}
%	\end{myfigure}		
% 


% %----------------------------------TWO FIGURES TEMPLATE-------------------------------------------------------------
% Description - Introduces 2 figures in the text. Replace name_of_figure by the name of the file you wish to include.
% %            	Remember that all figure files should be in the /1-images/ folder
% % 		fig_figure_label is the label of the general figure
% % 		fig_figure_label_1 is the label of figure (a)
% % 		fig_figure_label_2 is the label of figure (b)
% % 		the number in \figs{0.25} is the size of the figure in comparison with the original
% %---------------------------------------------------------------------------------------------------------------------
%
%	\begin{myfigure}{Global figure caption}{fig_figure_label}    
%		\mbox{	\subfigure[Figure (a) caption	\label{fig_figure_label_1}]{\figs{0.25}{name_of_figure_1.pdf}} \quad
%			\subfigure[Figure (b) caption 	\label{fig_figure_label_2}]{\figs{0.25}{name_of_figure_2.pdf}}} \\
%	\end{myfigure}		
%




% %----------------------------------EQUATION TEMPLATE-------------------------------------------------------------
% % Description - Introduces 1 equation in the text. Write down the equation in the place of the Variables below. 
% % 		  eq_equation_label is the label of the figure
% %---------------------------------------------------------------------------------------------------------------------
% % 
%	\begin{equation}
%		Variable_{1} = Variable_{2} \cdot Variable_{3}   
%		\label{eq_equation_label}
%	\end{equation}



% %----------------------------------TABLE TEMPLATE-------------------------------------------------------------
% % Description - Introduces 1 table in the text. Replace name_of_table by the name of the file you wish to include.
% %            	Remember that all table files should be in the /2-tables/ folder
% % 		tbl_table_label is the label of the figure
% % 		M{5cm} defines a column of 5 cm with the text centered
% % 		P{5cm} defines a column of 5 cm with the text on the left
% % 		c defines a column with an automatic size according to the text with the text on the left
% %---------------------------------------------------------------------------------------------------------------------
%
%	\begin{mytable}{Table caption}{tbl_table_label}{c M{5cm} P{5cm}}
%		\input{../2-tables/name_of_table.tex}
%	\end{mytable}












