\begin{verbatim}
Xenomai is a Free Software project in which engineers from a wide
background collaborate to build a versatile real-time framework for
the Linux© platform.
\end{verbatim}

À priori nous avions déjà écarté une solution basée sur linux lors d'une discussion
informelle car linux n'est pas un noyau temps réel.\\

Toutefois Xenomai semble y apporter des fonctionnalités de temps réel dur.\\

Bien que linux réponde largement à tous les autre critères technique, je crains que
ça ne soit au prix d'une complexité et d'un poids exagéré du firmware. C'est pourquoi
je l'ai écarté dans un premier temps.\\

Pour moi, il ne fait aucun doute que le STM32F746ZG puisse faire tourner Linux (j'ai 
effectué ma première installation de linux en 1997 sur un 80486@66MHz bien moins
rapide) mais ça nécessiterait d'emblée l'ajout de mémoire externe, le STM32F746ZG ne
disposant que de 320KB de SRAM en interne.

Par exemple on peut lire à propos de uClinux \cite{ref9} :
{\small
\begin{verbatim}
The size of a practical bootable image, with Ethernet, TCP/IP and a reasonable set of
user-space tools and applications confugured, would be in a 1.5 - 2 MBytes ballpark.
With the "two-chips Linux design" concept in mind, a 1.5 MBytes image could possibly
fit into internal Flash of today's Cortex-M microcontrollers. One example of a device
that can hold an image of the size is the STM32F429 Cortex-M4 microcontroller.

Size of external RAM required for run-time Linux operation. The answer we give to our
customers when asked how much RAM is needed is the more the better, but no less than
8 MBytes. Admittedly, it may be possible to run some very basic configurations with
rootfs mounted from NFS or some external device even out of 2 MBytes but frankly this
is more of a joke than a configuration one can build a practical uClinux design on.
\end{verbatim}
}

Pour ceux qui se demande quelle-est la configuration minimale pour un noyau Linux :
\url{http://dmitry.gr/?r=05.Projects&proj=07.%20Linux%20on%208bit} 

\subsubsection{Acceptabilité}
\begin{tabular}{lll}
\toprule
	Critère				&	Validé		&	Commentaire	\\
\midrule
	Licence				&	-			&		\\
	Code source			&	-			&		\\
	Plateformes cible	&	non			&	Linux	\\
	Communauté			&	-			&		\\
\bottomrule
\end{tabular}

