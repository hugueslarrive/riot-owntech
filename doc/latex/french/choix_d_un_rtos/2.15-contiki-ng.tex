\subsubsection{Licence}
Licence BSD 3 Clauses \cite{ref8}.\\

Cette licence est compatible avec la GNU GPL.

\subsubsection{Code source}
Le code est hébergé sur github : \url{https://github.com/contiki-ng/contiki-ng}\\

Les commentaires comportent des tags doxygen.\\

Il y a un \enquote{Code style} documenté.\\

Le code semble peu commenté mais très lisible.

\subsubsection{Plateformes cible}
Bien qu'il soit avant tout destiné à des SoC de capteurs miniature comme TI CC2538,
il semble supporter l'architecture Cortex-M7 :
\url{https://github.com/contiki-ng/contiki-ng/blob/develop/arch/cpu/arm/cortex-m/CMSIS/core_cm7.h}.

\subsubsection{Communauté}
Le dépot github indique 15345 commits et 193 contributeurs.\\

On y trouve un fichier \enquote{CONTRIBUTING.md} donc il accepte les contributions.\\

Il y a un wiki et un site internet.

\subsubsection{Acceptabilité}
\begin{tabular}{lll}
\toprule
	Critère				&	Validé		&	Commentaire	\\
\midrule
	Licence				&	oui			&		\\
	Code source			&	oui			&		\\
	Plateformes cible	&	oui			&		\\
	Communauté			&	oui			&		\\
\bottomrule
\end{tabular}

Bien qu'il réponde à tous les critères définis à ce stade, je vais l'écarter d'emblée
car ce n'est pas un RTOS : 
\url{https://fr.wikipedia.org/wiki/Contiki#Ex%C3%A9cution_d'applications_en_temps_r%C3%A9el},
\enquote{Contiki n'est pas un système d'exploitation permettant l'exécution
d'applications en temps réel}.

