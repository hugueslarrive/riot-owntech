\subsubsection{Licence}
As of May 2002, eCos is released under a modified version of the well known GNU
General Public License (GPL). The eCos license is officially recognised as a
GPL-compatible Free Software License and is an OSI-approved Open Source License. An
exception clause has been added which limits the circumstances in which the license
applies to other code when used in conjunction with eCos. The exception clause is as
follows:
\begin{verbatim}
As a special exception, if other files instantiate templates or use macros or
inline functions from this file, or you compile this file and link it with other
works to produce a work based on this file, this file does not by itself cause
the resulting work to be covered by the GNU General Public License. However the
source code for this file must still be made available in accordance with
section (3) of the GNU General Public License v2.

This exception does not invalidate any other reasons why a work based on this
file might be covered by the GNU General Public License.
\end{verbatim}

Le copyright a été transféré par Red Hat à la FSF en 2005 selon Wikipédia.

\subsubsection{Code source}
Le code est hébergé dans un dépôt CVS inaccessible par le lien du site
\url{http://sourceware.org/viewvc/ecos} (403 Forbidden) ce qui est un mauvais point,
et l'occasion de réinstaller cvs (bon vieil outil des 90').\\

Le source semble fourni et bien rangé, à première vue il respecte les \enquote{GNU
Coding Standards}\cite{ref4}.\\

Les dernières modifications datent de 2013.

\subsubsection{Plateformes cible}
Elles sont nombreuses et anciennes, on trouve quand même 3 cartes dans
packages/hal/cortexm/stm32.

\subsubsection{Communauté}
Elle semble éteinte (comme les dinosaures ?).\\

Le site n'est plus maintenu.

\subsubsection{Acceptabilité}
\begin{tabular}{lll}
\toprule
	Critère				&	Validé		&	Commentaire	\\
\midrule
	Licence				&	oui			&		\\
	Code source			&	oui			&		\\
	Plateformes cible	&	oui			&		\\
	Communauté			&	non			&		\\
\bottomrule
\end{tabular}

