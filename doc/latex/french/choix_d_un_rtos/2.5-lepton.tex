\subsubsection{Licence}
Mozilla Public License Version 1.1.\\

Incompatible avec la GNU GPL selon gnu.org\cite{ref2} :

{\small
\begin{verbatim}
Il s'agit d'une licence de logiciel libre, pas très stricte en tant que copyleft.
Contrairement à la licence X11 elle présente des restrictions complexes qui la
rendent incompatibles avec la GNU GPL. En effet on ne peut pas, légalement, lier un
module couvert par la GPL et un module couvert par la MPL. C'est pourquoi nous vous
demandons instamment de ne pas utiliser la MPL 1.1.

Cependant, la licence MPL 1.1 permet (article 13) à un programme ou à des portions du
programme d'offrir le choix d'une licence alternative. Si une portion du programme
offre la GNU GPL (ou toute autre licence compatible avec la GPL) comme choix
possible, alors la licence de cette portion du programme est compatible avec la GPL.
\end{verbatim}
}

Ouf ! On a eu chaud, il y a une licence alternative :

{\footnotesize
\begin{verbatim}
Alternatively, the contents of this file may be used under the terms of the eCos GPL license 
(the  [eCos GPL] License), in which case the provisions of [eCos GPL] License are applicable 
instead of those above. If you wish to allow use of your version of this file only under the
terms of the [eCos GPL] License and not to allow others to use your version of this file under 
the MPL, indicate your decision by deleting  the provisions above and replace 
them with the notice and other provisions required by the [eCos GPL] License. 
If you do not delete the provisions above, a recipient may use your version of this file under 
either the MPL or the [eCos GPL] License.
\end{verbatim}
}

Tiens, revoilà le dinosaure !

C'est quand même \enquote{borderline} niveau licence.

\subsubsection{Code source}
Le code source est sur github.com :
\url{https://github.com/lepton-distribution/lepton}.

Dernière modification du code en 2015.

\subsubsection{Plateformes cible}
Simulateur sur PC et ARMs dont :
\enquote{ARM Cortex-M4: STM32F4 Discovery (ST stm32F407)}

\subsubsection{Communauté}
Dernière modification du code en 2015...

Je n'ai pas trouvé de site.

\subsubsection{Acceptabilité}
\begin{tabular}{lll}
\toprule
	Critère				&	Validé		&	Commentaire	\\
\midrule
	Licence				&	oui			&		\\
	Code source			&	oui			&		\\
	Plateformes cible	&	oui			&		\\
	Communauté			&	non			&	Inactif	\\
\bottomrule
\end{tabular}

