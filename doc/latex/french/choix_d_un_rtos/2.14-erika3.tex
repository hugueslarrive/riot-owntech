\subsubsection{Licences}
ERIKA v3 is now distributed under the following three licensing schemes :
\begin{itemize}
	\item GPLv2, with a linking exception allowing the inclusion and linking of
			specific silicon vendor source code (e.g., the register include file, device
			drivers, and so on);
	\item GPLv2 + linking exception license, which will allow a customer to link
			proprietary source code;
	\item Commercial license.
\end{itemize}

\subsubsection{Code source}
Le code est hébergé sur github.com :
\url{https://github.com/evidence/erika3}.\\

Les commentaires comportent des tags Doxygen.\\

On retrouve un peu les mêmes bizarreries que dans le code de Trampoline :
\begin{verbatim}
FUNC(void, OS_CODE) osEE_cortex_m_system_init(void)
{
\end{verbatim}


\subsubsection{Plateformes cible}
Il y a un dossier cortex-m dans erika3/pkg/arch/.


\subsubsection{Communauté}
Le projet semble jeune (environ 2 ans), du moins pour la version 3.\\

Il semble peu actif : dernier commit en septembre 2019.\\

Il y a un wiki et un forum.\\

Le projet semble open-source mais fermé à la contribution.\\

Rien de très excitant en somme.

\subsubsection{Acceptabilité}
\begin{tabular}{lll}
\toprule
	Critère				&	Validé		&	Commentaire	\\
\midrule
	Licence				&	oui			&		\\
	Code source			&	oui			&		\\
	Plateformes cible	&	oui			&	cortex-m	\\
	Communauté			&	non			&	dernier commit en septembre 2019	\\
\bottomrule
\end{tabular}

