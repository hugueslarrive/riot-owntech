Cette architecture permet de faire fonctionner un projet sous forme de processus sous
Linux (32bits).\\

Sur mon système Debian 10 64bits (\enquote{crossgradée} depuis l'architecture 32bits)
j'ai pu effectuer cette procédure en quelques minutes car tous les outils nécessaires
étaient déjà installés.\\

Toutefois le lien concernant les dépendances ne traite que de la Debian 7.5 assez
ancienne et n'est donc plus fiable. Je vais donc recommencer la procédure de zéro
dans une VM 64 bits afin de déterminer ce qui est réellement nécessaire.

\subsubsection{Installation du système de base Debian 10 amd64}
Création d'une VM virtualbox 64bits avec les options par défaut.\\

Téléchargement d'une image iso de l'installeur :\\
{\footnotesize
\url{https://cdimage.debian.org/debian-cd/current/amd64/iso-cd/debian-10.3.0-amd64-netinst.iso}\\
}\\
%15h28
Installation du système de base avec uniquement \enquote{serveur SSH} et
\enquote{utilitaires usuels du système} lors de la sélection des logiciels.\\

Passer l'interface réseau en mode bridge.\\

Au premier démarrage installer net-tools, et identifier l'adresse réseau de la vm
avec ifconfig.\\

On peut maintenant s'y connecter par ssh.

\subsubsection{Installation des dépendances}
Dans un shell root :
%{\footnotesize
\begin{verbatim}
dpkg --add-architecture i386
apt-get update
apt-get install libc6-dev-i386 libc6-dbg:i386 build-essential pkg-config \
uml-utilities bridge-utils git unzip
\end{verbatim}
%}

\subsubsection{Préparation}
%{\scriptsize
\begin{verbatim}
git clone https://github.com/RIOT-OS/RIOT RIOT
cd RIOT/examples
cp -R default my_project
cd my_project
\end{verbatim}
%}
La page dit : \enquote{From this directory, you will compile everything you need,
including the RIOT OS itself– one small make and everything is ready.}.\\

Essayons :
%{\scriptsize
\begin{verbatim}
hugues@debian10base:~/RIOT/examples/my_project$ make
/home/hugues/RIOT/examples/my_project/../../Makefile.include:271: *** Neither
unzip nor 7z is installed.. Arrêt.
\end{verbatim}
%}
Je rajoute donc unzip dans la commande apt-get install de la section précédente.\\

Cette fois ça fonctionne :
%{\scriptsize
\begin{verbatim}
hugues@debian10base:~/RIOT/examples/my_project$ make
Building application "default" for "native" with MCU "native".

"make" -C /home/hugues/RIOT/boards/native
"make" -C /home/hugues/RIOT/boards/native/drivers
"make" -C /home/hugues/RIOT/core
"make" -C /home/hugues/RIOT/cpu/native
"make" -C /home/hugues/RIOT/cpu/native/netdev_tap
"make" -C /home/hugues/RIOT/cpu/native/periph
"make" -C /home/hugues/RIOT/cpu/native/stdio_native
"make" -C /home/hugues/RIOT/cpu/native/vfs
"make" -C /home/hugues/RIOT/drivers
"make" -C /home/hugues/RIOT/drivers/netdev_eth
"make" -C /home/hugues/RIOT/drivers/periph_common
"make" -C /home/hugues/RIOT/drivers/saul
"make" -C /home/hugues/RIOT/drivers/saul/init_devs
"make" -C /home/hugues/RIOT/sys
"make" -C /home/hugues/RIOT/sys/auto_init
"make" -C /home/hugues/RIOT/sys/fmt
"make" -C /home/hugues/RIOT/sys/iolist
"make" -C /home/hugues/RIOT/sys/net/gnrc
"make" -C /home/hugues/RIOT/sys/net/gnrc/netapi
"make" -C /home/hugues/RIOT/sys/net/gnrc/netif
"make" -C /home/hugues/RIOT/sys/net/gnrc/netif/ethernet
"make" -C /home/hugues/RIOT/sys/net/gnrc/netif/hdr
"make" -C /home/hugues/RIOT/sys/net/gnrc/netif/init_devs
"make" -C /home/hugues/RIOT/sys/net/gnrc/netreg
"make" -C /home/hugues/RIOT/sys/net/gnrc/pkt
"make" -C /home/hugues/RIOT/sys/net/gnrc/pktbuf
"make" -C /home/hugues/RIOT/sys/net/gnrc/pktbuf_static
"make" -C /home/hugues/RIOT/sys/net/gnrc/pktdump
"make" -C /home/hugues/RIOT/sys/net/link_layer/l2util
"make" -C /home/hugues/RIOT/sys/net/netif
"make" -C /home/hugues/RIOT/sys/od
"make" -C /home/hugues/RIOT/sys/phydat
"make" -C /home/hugues/RIOT/sys/ps
"make" -C /home/hugues/RIOT/sys/saul_reg
"make" -C /home/hugues/RIOT/sys/shell
"make" -C /home/hugues/RIOT/sys/shell/commands
   text	   data	    bss	    dec	    hex	filename
  91657	   1288	  72088	 165033	  284a9	/home/hugues/RIOT/examples/my_projec
t/bin/native/default.elf
\end{verbatim}
%}

\subsubsection{Test}
Créer l'interface tap0, dans le shell root :
%{\scriptsize
\begin{verbatim}
root@debian10base:~# tunctl -u hugues
Set 'tap0' persistent and owned by uid 1000
\end{verbatim}
%}

Lancer le projet :
{\small
\begin{verbatim}
hugues@debian10base:~/RIOT/examples/my_project$ ./bin/native/default.elf tap0
RIOT native interrupts/signals initialized.
Native RTC initialized.
LED_RED_OFF
LED_GREEN_ON
RIOT native board initialized.
RIOT native hardware initialization complete.

main(): This is RIOT! (Version: 2020.07-devel-244-gbb668)
Welcome to RIOT!
> help
help
Command              Description
---------------------------------------
reboot               Reboot the node
version              Prints current RIOT_VERSION
ps                   Prints information about running threads.
rtc                  control RTC peripheral interface
ifconfig             Configure network interfaces
txtsnd               Sends a custom string as is over the link layer
saul                 interact with sensors and actuators using SAUL
> ps
ps
	pid | name                 | state    Q | pri | stack  ( used) | base addr  | current     
	  - | isr_stack            | -        - |   - |   8192 (   -1) | 0x565e3520 | 0x565e3520
	  1 | idle                 | pending  Q |  15 |   8192 (  436) | 0x565e1240 | 0x565e30a0 
	  2 | main                 | running  Q |   7 |  12288 ( 3148) | 0x565de240 | 0x565e10a0 
	  3 | pktdump              | bl rx    _ |   6 |  12288 (  992) | 0x565e9c80 | 0x565ecae0 
	  4 | gnrc_netdev_tap      | bl rx    _ |   2 |   8192 ( 2444) | 0x565ecd00 | 0x565eeb60 
	    | SUM                  |            |     |  49152 ( 7020)
> rtc
rtc
usage: rtc <command> [arguments]
commands:
	poweron		power the interface on
	poweroff	power the interface off
	clearalarm	deactivate the current alarm
	getalarm	print the currently alarm time
	setalarm YYYY-MM-DD HH:MM:SS
			set an alarm for the specified time
	gettime		print the current time
	settime YYYY-MM-DD HH:MM:SS
			set the current time
> rtc gettime
rtc gettime
2020-04-29 11:45:16
> ifconfig
ifconfig
Iface  4  HWaddr: 3A:BE:D0:73:C8:37 
          L2-PDU:1500 Source address length: 6
          
> txtsnd help
txtsnd help
usage: txtsnd <if> [<L2 addr>|bcast] <data>
> txtsnd 4 bcast test    
txtsnd 4 bcast test
> 
\end{verbatim}
}

On voit un processus \enquote{pktdump} qui revoit probablement ce qui passe dans le
réseau sur la console. Si on active l'interface tap0 sur le système linux hôte
(ifconfig tap0 up), la console riot est immédiatement pourrie de paquets ipv6 :
{\small
\begin{verbatim}
> PKTDUMP: data received:
~~ SNIP  0 - size:  76 byte, type: NETTYPE_UNDEF (0)
00000000  60  00  00  00  00  24  00  01  00  00  00  00  00  00  00  00
00000010  00  00  00  00  00  00  00  00  FF  02  00  00  00  00  00  00
00000020  00  00  00  00  00  00  00  16  3A  00  05  02  00  00  01  00
00000030  8F  00  A6  E0  00  00  00  01  04  00  00  00  FF  02  00  00
00000040  00  00  00  00  00  00  00  01  FF  73  C8  36
~~ SNIP  1 - size:  20 byte, type: NETTYPE_NETIF (-1)
if_pid: 4  rssi: 0  lqi: 0
flags: 0x0
src_l2addr: 3A:BE:D0:73:C8:36
dst_l2addr: 33:33:00:00:00:16
~~ PKT    -  2 snips, total size:  96 byte
PKTDUMP: data received:
~~ SNIP  0 - size:  72 byte, type: NETTYPE_UNDEF (0)
00000000  60  00  00  00  00  20  3A  FF  00  00  00  00  00  00  00  00
00000010  00  00  00  00  00  00  00  00  FF  02  00  00  00  00  00  00
00000020  00  00  00  01  FF  73  C8  36  87  00  7C  10  00  00  00  00
\end{verbatim}
}

Pour éviter ça on peut désactiver l'ipv6 dans le noyau linux :
\begin{verbatim}
root@debian10base:~# echo 1 > /proc/sys/net/ipv6/conf/all/disable_ipv6 
\end{verbatim}

OK, on va essayer d'en faire communiquer 2 vite fait.\\

Pour cela on va créer une seconde interface et bridger les 2 :
%{\scriptsize
\begin{verbatim}
root@debian10base:~# tunctl -u hugues
Set 'tap1' persistent and owned by uid 1000
root@debian10base:~# ifconfig tap1 up
root@debian10base:~# brctl addbr br0
root@debian10base:~# brctl addif br0 tap0 tap1
root@debian10base:~# ifconfig br0 up
\end{verbatim}
%}
Puis on lance un second RIOT et on retente la commande txtsnd :
%{\scriptsize
\begin{verbatim}
hugues@debian10base:~/RIOT/examples/my_project$ ./bin/native/default.elf tap1
RIOT native interrupts/signals initialized.
Native RTC initialized.
LED_RED_OFF
LED_GREEN_ON
RIOT native board initialized.
RIOT native hardware initialization complete.

main(): This is RIOT! (Version: 2020.07-devel-244-gbb668)
Welcome to RIOT!
> txtsnd 4 bcast test 
txtsnd 4 bcast test 
> 
\end{verbatim}
%}
Dans la console du premier on voit apparaître :
%{\scriptsize
\begin{verbatim}
PKTDUMP: data received:
~~ SNIP  0 - size:   4 byte, type: NETTYPE_UNDEF (0)
00000000  74  65  73  74
~~ SNIP  1 - size:  20 byte, type: NETTYPE_NETIF (-1)
if_pid: 4  rssi: 0  lqi: 0
flags: 0x0
src_l2addr: 1E:FF:60:4B:51:E2
dst_l2addr: FF:FF:FF:FF:FF:FF
~~ PKT    -  2 snips, total size:  24 byte
^C
native: exiting
hugues@debian10base:~/RIOT/examples/my_project$ echo -n test | hd
00000000  74 65 73 74                                       |test|
00000004
hugues@debian10base:~/RIOT/examples/my_project$ /sbin/ifconfig tap1
tap1: flags=4099<UP,BROADCAST,MULTICAST>  mtu 1500
        ether 1e:ff:60:4b:51:e1  txqueuelen 1000  (Ethernet)
        RX packets 2  bytes 36 (36.0 B)
        RX errors 0  dropped 0  overruns 0  frame 0
        TX packets 0  bytes 0 (0.0 B)
        TX errors 0  dropped 0 overruns 0  carrier 0  collisions 0
\end{verbatim}
%}
On observe que \enquote{00000000  74  65  73  74} correspond bien à la chaîne de
caractères \enquote{test} envoyée et que \enquote{src\_l2addr: 1E:FF:60:4B:51:E2}
est l'adresse MAC de la source.
