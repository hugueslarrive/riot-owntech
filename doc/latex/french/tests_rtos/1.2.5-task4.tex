Utiliser \texttt{xtimer} pour créer un thread qui affiche le temps
système courant toutes les 2 secondes.\\

Le thread\_handler :
\begin{lstlisting}
void *thread_handler(void *arg)
{
    /* Thread that print the current system time every 2 seconds... */
    (void)arg;
    
    while(true){
		printf("Current system time is %ld microseconds\n",xtimer_now_usec());
		xtimer_sleep(2);
	}
    
    return NULL;
}
\end{lstlisting}

Le résultat :
{\scriptsize
\begin{verbatim}
2020-05-07 16:22:47,041 # Current system time is 5090 microseconds
> 2020-05-07 16:22:49,044 #  Current system time is 2008736 microseconds
2020-05-07 16:22:51,048 # Current system time is 4012642 microseconds
2020-05-07 16:22:53,051 # Current system time is 6016549 microseconds
2020-05-07 16:22:55,055 # Current system time is 8020456 microseconds
2020-05-07 16:22:57,059 # Current system time is 10024362 microseconds
ps
2020-05-07 16:22:57,812 # ps
2020-05-07 16:22:57,819 # 	pid | name                 | state    Q | pri | stack  ( used) | base addr  | current     
2020-05-07 16:22:57,827 # 	  - | isr_stack            | -        - |   - |    512 (  156) | 0x20000000 | 0x200001c8
2020-05-07 16:22:57,835 # 	  1 | idle                 | pending  Q |  15 |    256 (  132) | 0x200003f8 | 0x20000474 
2020-05-07 16:22:57,844 # 	  2 | main                 | running  Q |   7 |   1536 (  676) | 0x200004f8 | 0x2000094c 
2020-05-07 16:22:57,852 # 	  3 | thread               | bl mutex _ |   6 |   1536 (  356) | 0x20000afc | 0x2000102c 
2020-05-07 16:22:57,858 # 	    | SUM                  |            |     |   3840 ( 1320)
> 2020-05-07 16:22:59,063 #  Current system time is 12028356 microseconds
2020-05-07 16:23:01,066 # Current system time is 14032358 microseconds
2020-05-07 16:23:03,071 # Current system time is 16036360 microseconds
2020-05-07 16:23:05,074 # Current system time is 18040362 microseconds
\end{verbatim}
}

Il y a un écart d'environ 4ms. J'ai modifié le programme pour mesurer
la durée de la boucle :
{\scriptsize
\begin{verbatim}
2020-05-07 16:51:05,767 # Loop duration 7673 microseconds
2020-05-07 16:51:07,771 # Loop duration 2002858 microseconds
2020-05-07 16:51:09,774 # Loop duration 2003119 microseconds
2020-05-07 16:51:11,777 # Loop duration 2003118 microseconds
2020-05-07 16:51:13,780 # Loop duration 2003118 microseconds
2020-05-07 16:51:15,783 # Loop duration 2003118 microseconds
2020-05-07 16:51:17,786 # Loop duration 2003118 microseconds
2020-05-07 16:51:19,789 # Loop duration 2003118 microseconds
2020-05-07 16:51:21,792 # Loop duration 2003118 microseconds
2020-05-07 16:51:23,795 # Loop duration 2003118 microseconds
2020-05-07 16:51:25,798 # Loop duration 2003118 microseconds
2020-05-07 16:51:27,800 # Loop duration 2003118 microseconds
\end{verbatim}
}

On peut voir qu'il a l'air de se stabiliser après 3 itérations.
Maintenant un essai de compensation avec
\texttt{xtimer\_usleep(1996882);} (2000000-3118=1996882):
{\scriptsize
\begin{verbatim}
2020-05-07 17:00:23,165 # Loop duration 7673 microseconds
2020-05-07 17:00:25,166 # Loop duration 1999738 microseconds
2020-05-07 17:00:27,165 # Loop duration 2000007 microseconds
2020-05-07 17:00:29,165 # Loop duration 2000007 microseconds
2020-05-07 17:00:31,165 # Loop duration 2000006 microseconds
2020-05-07 17:00:33,165 # Loop duration 2000007 microseconds
2020-05-07 17:00:35,165 # Loop duration 2000007 microseconds
2020-05-07 17:00:37,165 # Loop duration 2000006 microseconds
2020-05-07 17:00:39,164 # Loop duration 2000007 microseconds
2020-05-07 17:00:41,164 # Loop duration 2000007 microseconds
\end{verbatim}
}

On y arrive pas de cette manière : avec un sleep time de 1996875
à 1996881 on obtient 1999998, et ça passe à 2000007 à partir de
1996882.\\

Le printf imprime sur la console à 115200 bauds ce qui prend environ
87 $\mu$s par caractère.\\

\begin{comment}
Bof... en enlevant encore 7 ?
{\scriptsize
\begin{verbatim}
2020-05-07 17:06:51,271 # Loop duration 7673 microseconds
2020-05-07 17:06:53,272 # Loop duration 1999730 microseconds
2020-05-07 17:06:55,272 # Loop duration 1999998 microseconds
2020-05-07 17:06:57,272 # Loop duration 1999998 microseconds
2020-05-07 17:06:59,272 # Loop duration 1999998 microseconds
2020-05-07 17:07:01,272 # Loop duration 1999998 microseconds
2020-05-07 17:07:03,271 # Loop duration 1999998 microseconds
2020-05-07 17:07:05,271 # Loop duration 1999998 microseconds
\end{verbatim}
}

Toujours pas... +2 ?
{\scriptsize
\begin{verbatim}
2020-05-07 17:09:14,078 # Loop duration 1999738 microseconds
2020-05-07 17:09:16,077 # Loop duration 1999998 microseconds
2020-05-07 17:09:18,077 # Loop duration 1999998 microseconds
2020-05-07 17:09:20,077 # Loop duration 1999998 microseconds
2020-05-07 17:09:22,077 # Loop duration 1999998 microseconds
2020-05-07 17:09:24,077 # Loop duration 1999998 microseconds
\end{verbatim}
}

Avec des valeurs de sleep de 1996875 à 1996881 on obtient 1999998, et ça
passe à 2000007 à partir de 1996882, on y arrivera pas.

L'essentiel de la durée d'exécution de la boucle est lié au 
\texttt{printf}. On peut donc obtenir quelque chose de
beaucoup plus précis le mettant dans un autre thread :
{\scriptsize
\begin{verbatim}
2020-05-07 17:35:48,433 # Loop duration 2000024 microseconds
2020-05-07 17:35:50,436 # Loop duration 2000025 microseconds
2020-05-07 17:35:52,439 # Loop duration 2000025 microseconds
2020-05-07 17:35:54,442 # Loop duration 2000025 microseconds
ps
2020-05-07 17:35:56,135 # ps
2020-05-07 17:35:56,141 # 	pid | name                 | state    Q | pri | stack  ( used) | base addr  | current     
2020-05-07 17:35:56,149 # 	  - | isr_stack            | -        - |   - |    512 (  156) | 0x20000000 | 0x200001c8
2020-05-07 17:35:56,157 # 	  1 | idle                 | pending  Q |  15 |    256 (  132) | 0x200003f8 | 0x20000474 
2020-05-07 17:35:56,166 # 	  2 | main                 | running  Q |   7 |   1536 (  688) | 0x200004f8 | 0x2000094c 
2020-05-07 17:35:56,175 # 	  3 | thread               | bl mutex _ |   5 |   1536 (  208) | 0x20000b04 | 0x20001034 
2020-05-07 17:35:56,183 # 	  4 | thread2              | bl mutex _ |   6 |   1536 (  356) | 0x20001104 | 0x20001634 
2020-05-07 17:35:56,189 # 	    | SUM                  |            |     |   5376 ( 1540)
> 2020-05-07 17:35:56,444 #  Loop duration 2000025 microseconds
2020-05-07 17:35:58,447 # Loop duration 2000025 microseconds
2020-05-07 17:36:00,450 # Loop duration 2000025 microseconds
2020-05-07 17:36:02,453 # Loop duration 2000025 microseconds
\end{verbatim}
}

C'est mieux ! Voyons si on peut compenser ces 25$\mu$ :
{\scriptsize
\begin{verbatim}
2020-05-07 17:46:00,323 # Loop duration 1999999 microseconds
2020-05-07 17:46:02,327 # Loop duration 2000000 microseconds
2020-05-07 17:46:04,329 # Loop duration 2000000 microseconds
2020-05-07 17:46:06,332 # Loop duration 2000000 microseconds
2020-05-07 17:46:08,335 # Loop duration 2000000 microseconds
2020-05-07 17:46:10,338 # Loop duration 2000000 microseconds
2020-05-07 17:46:12,341 # Loop duration 2000000 microseconds
2020-05-07 17:46:14,344 # Loop duration 2000000 microseconds
2020-05-07 17:46:16,347 # Loop duration 2000000 microseconds
2020-05-07 17:46:18,350 # Loop duration 2000000 microseconds
2020-05-07 17:46:20,353 # Loop duration 2000001 microseconds
2020-05-07 17:46:22,356 # Loop duration 2000000 microseconds
2020-05-07 17:46:24,358 # Loop duration 2000000 microseconds
2020-05-07 17:46:26,361 # Loop duration 2000000 microseconds
\end{verbatim}
}

On arrive quand même à quelque chose de relativement précis et stable.
Là j'étais sur la bleupill, évidament ce n'est pas portable, sur la
nucleo ça donne :
{\scriptsize
\begin{verbatim}
2020-05-07 17:59:44,341 # Loop duration 1999993 microseconds
2020-05-07 17:59:46,344 # Loop duration 1999994 microseconds
2020-05-07 17:59:48,347 # Loop duration 1999993 microseconds
2020-05-07 17:59:50,350 # Loop duration 1999993 microseconds
2020-05-07 17:59:52,353 # Loop duration 1999993 microseconds
2020-05-07 17:59:54,355 # Loop duration 1999993 microseconds
2020-05-07 17:59:56,359 # Loop duration 1999993 microseconds
2020-05-07 17:59:58,362 # Loop duration 1999993 microseconds
\end{verbatim}
}
\end{comment}

J'ai donc sorti le printf dans un autre thread en utilisant le système
de message IPC de RIOT (voir
\url{https://github.com/RIOT-OS/RIOT/tree/master/examples/ipc_pingpong}.
\\

Le main.c :
\begin{lstlisting}
#include <stdio.h>
#include <string.h>

#include "shell.h"
#include "thread.h"
#include "xtimer.h"
#include "msg.h"

#define PERIOD_US 2000000

char stack_timer2s[THREAD_STACKSIZE_MAIN];
char stack_printf2s[THREAD_STACKSIZE_MAIN];

void *printf2s_handler(void *arg)
{
    /* Thread that print the current system time every 2 seconds... */
    (void)arg;
    msg_t t;
    uint32_t time; // the current system time received from timer2s
    uint32_t duration; // the real loop duration from previous time
    uint32_t prev_time = 0; // to save the previous time
    uint32_t gap = 0; // the gap between sleep_time and PERIOD_US
    uint32_t sleep_time = PERIOD_US;
    uint32_t before_printf_time = 0;
    uint32_t after_printf_time = 0;
    uint32_t prev_before_printf_time = 0;

    while (1) {
        msg_receive(&t);
        time = t.content.value;
        duration = time - prev_time;
        gap = prev_time ? duration - PERIOD_US : 0;// not the first time
        if (gap) {
            sleep_time -= gap;
        }
        t.content.value = sleep_time;
        msg_reply(&t, &t);
        before_printf_time = xtimer_now_usec();
        printf("system time: %lu £\color{orange}$\mu$£s, last period: %lu £\color{orange}$\mu$£s, next sleep\
 time: %lu £\color{orange}$\mu$£s, time gap: %lu £\color{orange}$\mu$£s, last printf: %lu £\color{orange}$\mu$£s\n",
            time, duration, sleep_time, PERIOD_US - sleep_time,
            after_printf_time - prev_before_printf_time);
        after_printf_time = xtimer_now_usec();
        prev_before_printf_time = before_printf_time;
        prev_time = time;
    }
    
    return NULL;
}

void *timer2s_handler(void *arg)
{
    /* Thread that save the current system time every 2 seconds... */
    (void)arg;

    msg_t t;
    
    kernel_pid_t pid = thread_create(stack_printf2s,
                            sizeof(stack_printf2s),
                            THREAD_PRIORITY_MAIN - 1,
                            THREAD_CREATE_STACKTEST,
                            printf2s_handler, NULL,
                            "printf2s");
    
    while (1) {
        t.content.value = xtimer_now_usec();
        msg_send_receive(&t, &t, pid);
        xtimer_usleep(t.content.value);
    }

    return NULL;
}

int main(void)
{
    puts("This is Task-04");

    thread_create(stack_timer2s, sizeof(stack_timer2s),
                  THREAD_PRIORITY_MAIN - 2,
                  THREAD_CREATE_STACKTEST,
                  timer2s_handler, NULL,
                  "timer2s");
 
    char line_buf[SHELL_DEFAULT_BUFSIZE];
    shell_run(NULL, line_buf, SHELL_DEFAULT_BUFSIZE);

    return 0;
}
\end{lstlisting}

La durée du sleep est ajustée automatique lors des premières itérations
ce qui rend le code portable.\\

Le résultat sur la bluepill :
{\tiny
\begin{alltt}
2020-05-13 10:36:46,578 # main(): This is RIOT! (Version: 2020.04)
2020-05-13 10:36:46,581 # This is Task-04
2020-05-13 10:36:46,590 # system time: 5123 \(\mu\)s, last period: 5123 \(\mu\)s, next sleep time: 2000000 \(\mu\)s, time gap: 0 \(\mu\)s, last printf: 0 \(\mu\)s
> 2020-05-13 10:36:48,592 #  system time: 2005164 \(\mu\)s, last period: 2000041 \(\mu\)s, next sleep time: 1999959 \(\mu\)s, time gap: 41 \(\mu\)s, last printf: 9911 \(\mu\)s
2020-05-13 10:36:50,592 # system time: 4005160 \(\mu\)s, last period: 1999996 \(\mu\)s, next sleep time: 1999963 \(\mu\)s, time gap: 37 \(\mu\)s, last printf: 10791 \(\mu\)s
ps
2020-05-13 10:36:51,307 # ps
2020-05-13 10:36:51,313 # 	pid | name                 | state    Q | pri | stack  ( used) | base addr  | current     
2020-05-13 10:36:51,322 # 	  - | isr_stack            | -        - |   - |    512 (  156) | 0x20000000 | 0x200001c8
2020-05-13 10:36:51,330 # 	  1 | idle                 | pending  Q |  15 |    256 (  132) | 0x200003f8 | 0x20000474 
2020-05-13 10:36:51,339 # 	  2 | main                 | running  Q |   7 |   1536 (  676) | 0x200004f8 | 0x2000094c 
2020-05-13 10:36:51,347 # 	  3 | timer2s              | bl mutex _ |   5 |   1536 (  232) | 0x200010fc | 0x20001614 
2020-05-13 10:36:51,356 # 	  4 | printf2s             | bl rx    _ |   6 |   1536 (  404) | 0x20000afc | 0x20001034 
2020-05-13 10:36:51,361 # 	    | SUM                  |            |     |   5376 ( 1600)
> 2020-05-13 10:36:52,591 #  system time: 6005160 \(\mu\)s, last period: 2000000 \(\mu\)s, next sleep time: 1999963 \(\mu\)s, time gap: 37 \(\mu\)s, last printf: 10871 \(\mu\)s
2020-05-13 10:36:54,591 # system time: 8005160 \(\mu\)s, last period: 2000000 \(\mu\)s, next sleep time: 1999963 \(\mu\)s, time gap: 37 \(\mu\)s, last printf: 10877 \(\mu\)s
2020-05-13 10:36:56,591 # system time: 10005160 \(\mu\)s, last period: 2000000 \(\mu\)s, next sleep time: 1999963 \(\mu\)s, time gap: 37 \(\mu\)s, last printf: 10875 \(\mu\)s
2020-05-13 10:36:58,591 # system time: 12005160 \(\mu\)s, last period: 2000000 \(\mu\)s, next sleep time: 1999963 \(\mu\)s, time gap: 37 \(\mu\)s, last printf: 10960 \(\mu\)s
2020-05-13 10:37:00,591 # system time: 14005160 \(\mu\)s, last period: 2000000 \(\mu\)s, next sleep time: 1999963 \(\mu\)s, time gap: 37 \(\mu\)s, last printf: 10958 \(\mu\)s
2020-05-13 10:37:02,590 # system time: 16005160 \(\mu\)s, last period: 2000000 \(\mu\)s, next sleep time: 1999963 \(\mu\)s, time gap: 37 \(\mu\)s, last printf: 10965 \(\mu\)s
\end{alltt}
}

Sur la nucleo :
{\tiny
\begin{alltt}
2020-05-13 15:32:06,195 # This is Task-04
2020-05-13 15:32:06,205 # system time: 5035 \(\mu\)s, last period: 5035 \(\mu\)s, next sleep time: 2000000 \(\mu\)s, time gap: 0 \(\mu\)s, last printf: 0 \(\mu\)s
> 2020-05-13 15:32:08,206 #  system time: 2005065 \(\mu\)s, last period: 2000030 \(\mu\)s, next sleep time: 1999970 \(\mu\)s, time gap: 30 \(\mu\)s, last printf: 9807 \(\mu\)s
2020-05-13 15:32:10,206 # system time: 4005062 \(\mu\)s, last period: 1999997 \(\mu\)s, next sleep time: 1999973 \(\mu\)s, time gap: 27 \(\mu\)s, last printf: 10679 \(\mu\)s
2020-05-13 15:32:12,206 # system time: 6005061 \(\mu\)s, last period: 1999999 \(\mu\)s, next sleep time: 1999974 \(\mu\)s, time gap: 26 \(\mu\)s, last printf: 10766 \(\mu\)s
2020-05-13 15:32:14,206 # system time: 8005061 \(\mu\)s, last period: 2000000 \(\mu\)s, next sleep time: 1999974 \(\mu\)s, time gap: 26 \(\mu\)s, last printf: 10766 \(\mu\)s
2020-05-13 15:32:16,206 # system time: 10005061 \(\mu\)s, last period: 2000000 \(\mu\)s, next sleep time: 1999974 \(\mu\)s, time gap: 26 \(\mu\)s, last printf: 10766 \(\mu\)s
2020-05-13 15:32:18,206 # system time: 12005061 \(\mu\)s, last period: 2000000 \(\mu\)s, next sleep time: 1999974 \(\mu\)s, time gap: 26 \(\mu\)s, last printf: 10853 \(\mu\)s
2020-05-13 15:32:20,207 # system time: 14005061 \(\mu\)s, last period: 2000000 \(\mu\)s, next sleep time: 1999974 \(\mu\)s, time gap: 26 \(\mu\)s, last printf: 10853 \(\mu\)s
\end{alltt}
}

J'ai aussi effectué un test avec \enquote{BOARD=native}, évidemment ça
ne se stabilise jamais.\\

On peut voir qu'il faut 2 ou 3 périodes pour que le "sleep\_time" soit
ajusté. C'est dû au fait que le thread printf2s n'est pas totalement
prêt à traiter l'IPC lors de la première itération de la boucle. On peut
facilement résoudre ce problème en ajoutant un délai de 30$\mu$s entre
la création du thread et la boucle. Ainsi la période est ajustée dès la
seconde itération.\\

J'ai aussi rajouté une constante qui permet de prédéfinir l'écart s'il
est connu pour que la période soit juste dès la première itération. Ça a
bien fonctionné sur la bluepill, un peu mois sur la nucleo où la
première fais 1$\mu$s de trop et la seconde 1$\mu$s de moins avec ce
code.
