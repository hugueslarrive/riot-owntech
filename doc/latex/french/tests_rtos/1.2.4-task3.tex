Il s'agit de lancer un thread.\\

On ajoute le code donné dans la fonction main() et lorsque le handler
est exécuté il affiche \enquote{I'm in "thread" now} dans le terminal.\\

J'ai trouvé ça un peu nul donc j'ai ajouté un \texttt{ps();} dans le
handler pour pouvoir mieux l'observer (il faut aussi ajouter
\texttt{\#include "ps.h"}) :
{\scriptsize
\begin{verbatim}
 2020-05-07 01:25:33,090 # reboot
2020-05-07 01:25:33,093 # main(): This is RIOT! (Version: 2020.04)
2020-05-07 01:25:33,095 # This is Task-03
2020-05-07 01:25:33,096 # I'm in "thread" now
2020-05-07 01:25:33,104 # 	pid | name                 | state    Q | pri | stack  ( used) | base addr  | current     
2020-05-07 01:25:33,113 # 	  - | isr_stack            | -        - |   - |    512 (  104) | 0x20000000 | 0x200001c8
2020-05-07 01:25:33,121 # 	  1 | idle                 | pending  Q |  15 |    256 (  124) | 0x200003c0 | 0x20000444 
2020-05-07 01:25:33,130 # 	  2 | main                 | pending  Q |   7 |   1536 (  360) | 0x200004c0 | 0x20000974 
2020-05-07 01:25:33,136 # 	  3 | thread               | running  Q |   6 |   1536 (  448) | 0x20000ac4 | 0x2000104c 
2020-05-07 01:25:33,143 # 	    | SUM                  |            |     |   3840 ( 1036)
> ps
2020-05-07 01:25:41,334 #  ps
2020-05-07 01:25:41,340 # 	pid | name                 | state    Q | pri | stack  ( used) | base addr  | current     
2020-05-07 01:25:41,348 # 	  - | isr_stack            | -        - |   - |    512 (  116) | 0x20000000 | 0x200001c8
2020-05-07 01:25:41,357 # 	  1 | idle                 | pending  Q |  15 |    256 (  132) | 0x200003c0 | 0x2000043c 
2020-05-07 01:25:41,365 # 	  2 | main                 | running  Q |   7 |   1536 (  676) | 0x200004c0 | 0x20000914 
2020-05-07 01:25:41,372 # 	    | SUM                  |            |     |   2304 (  924)
\end{verbatim}
}
J'observe une forte augmentation de la mémoire utilisée dans la pile du
thread main (360$\rightarrow$676) suite à la commande \texttt{ps}.\\

Au début du programme donné il y a :
\texttt{char stack[THREAD\_STACKSIZE\_MAIN];}.

Cette constante est définie dans le fichier :
\texttt{RIOT/blob/master/core/include/thread.h}.

On y voit que :\\
\texttt{THREAD\_STACKSIZE\_MAIN} =
\texttt{THREAD\_STACKSIZE\_DEFAULT} +
\texttt{THREAD\_EXTRA\_STACKSIZE\_PRINTF}\\

Sur le stm32 \texttt{THREAD\_STACKSIZE\_MAIN} = 1024 et 
\texttt{THREAD\_EXTRA\_STACKSIZE\_PRINTF} = 512.\\

On y voit qu'il y a aussi :\\
\begin{itemize}
\item \texttt{THREAD\_STACKSIZE\_LARGE = THREAD\_STACKSIZE\_MEDIUM x 2}
\item \texttt{THREAD\_STACKSIZE\_MEDIUM = THREAD\_STACKSIZE\_DEFAULT}
\item \texttt{THREAD\_STACKSIZE\_SMALL = THREAD\_STACKSIZE\_MEDIUM / 2}
\item \texttt{THREAD\_STACKSIZE\_TINY = THREAD\_STACKSIZE\_MEDIUM / 4}
\end{itemize}~\\

Je l'ai donc définie à \texttt{THREAD\_STACKSIZE\_SMALL} (512) ce qui
est suffisant.\\

Sur le nano c'est insuffisant : vu que j'ai réduit
\texttt{THREAD\_STACKSIZE\_MAIN} de moitié on ser retrouve avec
l'équivalent de \texttt{THREAD\_EXTRA\_STACKSIZE\_PRINTF} et il nous 
faudrait un peu plus.\\

Le bon compromis est donc \texttt{THREAD\_STACKSIZE\_TINY} +
\texttt{THREAD\_EXTRA\_STACKSIZE\_PRINTF} ce qui nous donnera 768 sur le
stm32 et 392 sur l'avr.
