\documentclass[a4paper,12pt, twoside]{article}

\usepackage[T1]{fontenc}
\usepackage[utf8]{inputenc}
%~ \usepackage{lmodern}
\usepackage{helvet}
\renewcommand{\familydefault}{\sfdefault}
\usepackage[francais]{babel}
\usepackage{verbatim}
\usepackage{tikz}
\usepackage{eurosym} % pour € \euro{}
\usepackage{csquotes} % pour les guillemets

%\usepackage{newtxmath}

\usepackage[
  % Some remarks:
  % * drivers like 'pdftex' that can be detected automatically
  %   are not necessary
  % * breaklinks is rather an internal option.
  %   If a driver does not support it, then forcing the option
  %   let the text break across lines, but also the link
  %   areas are "broken". If the driver supports the option,
  %   then the option is enabled anyway.
  % * Information entries should be set outside,
  %   because LaTeX expands the package options,
  %   hyperref does not like them, if they are
  %   prematurely expanded.
  % * Hyperref has a new option for hiding links: hidelinks
  hidelinks,
  pagebackref,
  bookmarksopen,
  bookmarksnumbered,
  a4paper,
]{hyperref}
\hypersetup{
  pdfauthor={Hugues Larrive},
  % ...
}
% Adding package bookmark improves bookmarks handling.
% More features and faster updated bookmarks.
\usepackage{bookmark}

\usepackage{layout}
\oddsidemargin=-1cm
\usepackage[top=2cm, bottom=2cm, left=2cm, right=2cm]{geometry}

\usepackage{wrapfig}
\usepackage{graphicx}


\usepackage{listings}
\lstset{
     literate=%
		{à}{{\`a}}1
		{é}{{\'e}}1
		{ê}{{\^e}}1
		{è}{{\`e}}1
		{É}{{\'E}}1
		{ç}{{\c{c}}}1
}
\usepackage{color}

\definecolor{mygreen}{rgb}{0,0.6,0}
\definecolor{mygray}{rgb}{0.5,0.5,0.5}
\definecolor{mymauve}{rgb}{0.58,0,0.82}

\lstset{ 
  backgroundcolor=\color{white},   % choose the background color; you must add \usepackage{color} or \usepackage{xcolor}; should come as last argument
  basicstyle=\footnotesize,        % the size of the fonts that are used for the code
  breakatwhitespace=false,         % sets if automatic breaks should only happen at whitespace
  breaklines=true,                 % sets automatic line breaking
  captionpos=b,                    % sets the caption-position to bottom
  commentstyle=\color{mygreen},    % comment style
  deletekeywords={...},            % if you want to delete keywords from the given language
  escapeinside={\%*}{*)},          % if you want to add LaTeX within your code
  extendedchars=true,              % lets you use non-ASCII characters; for 8-bits encodings only, does not work with UTF-8
  frame=single,	                   % adds a frame around the code
  keepspaces=true,                 % keeps spaces in text, useful for keeping indentation of code (possibly needs columns=flexible)
  keywordstyle=\color{blue},       % keyword style
  language=Octave,                 % the language of the code
  morekeywords={*,...},            % if you want to add more keywords to the set
  numbers=left,                    % where to put the line-numbers; possible values are (none, left, right)
  numbersep=5pt,                   % how far the line-numbers are from the code
  numberstyle=\tiny\color{mygray}, % the style that is used for the line-numbers
  rulecolor=\color{black},         % if not set, the frame-color may be changed on line-breaks within not-black text (e.g. comments (green here))
  showspaces=false,                % show spaces everywhere adding particular underscores; it overrides 'showstringspaces'
  showstringspaces=false,          % underline spaces within strings only
  showtabs=false,                  % show tabs within strings adding particular underscores
  stepnumber=1,                    % the step between two line-numbers. If it's 1, each line will be numbered
  stringstyle=\color{mymauve},     % string literal style
  tabsize=2%,	                   % sets default tabsize to 2 spaces
  %title=\lstname                   % show the filename of files included with \lstinputlisting; also try caption instead of title
}

\lstdefinestyle{customc}{
  belowcaptionskip=1\baselineskip,
  breaklines=true,
  frame=L,
  xleftmargin=\parindent,
  language=C,
  showstringspaces=false,
  basicstyle=\footnotesize\ttfamily,
  keywordstyle=\bfseries\color{green!40!black},
  commentstyle=\itshape\color{purple!40!black},
  identifierstyle=\color{blue},
  stringstyle=\color{orange},
}

\lstset{escapechar=£,style=customc}

\usepackage{algorithm2e}
\usepackage{pdfpages}
\usepackage{gensymb} % for \degree
%\usepackage{setspace}

\usepackage{caption}
\usepackage{booktabs}
\usepackage{multirow}
\newcommand{\MR}[2]{\multirow{#1}{*}{#2}}			%simplifies multirow command - must use the multirow package
\newcommand{\MC}[2]{\multicolumn{#1}{c}{#2}}			%simplifies the multicollumn command - must use the multicollumn package





\usepackage{amssymb}

\usepackage[firstpage]{draftwatermark}

\usepackage{alltt}

\usepackage{pifont}% http://ctan.org/pkg/pifont
\newcommand{\cm}{\textcolor{green}{\ding{51}}}%
\newcommand{\xm}{\textcolor{red}{\ding{55}}}%


\title{Tests d'RTOS pour le firmware OwnTech}
\author{Hugues Larrive <hugues.larrive@laas.fr>}
\date{\today}	%defines the date of the document - leave empty for no date


\begin{document}

\maketitle{}

\vspace*{\stretch{1}}

\begin{abstract}
	Ce document rapporte les tests effectués sur un ou plusieurs RTOS pour le
	développement du firmware OwnTech, en commençant par RIOT OS. D'autres RTOS
	pourront être testés soit s'il apparaît que RIOT OS ne convient pas, soit si l'on
	veut avoir un ou des points de comparaison.
\end{abstract}

\vspace*{\stretch{1}}

{\footnotesize
\begin{verbatim}
    Author: Hugues Larrive <hugues.larrive@laas.fr>

	Copyright (C) 2020 LAAS-CNRS.
	Permission is granted to copy, distribute and/or modify this document
	under the terms of the GNU Free Documentation License, Version 1.3
	or any later version published by the Free Software Foundation;
	with no Invariant Sections, no Front-Cover Texts, and no Back-Cover Texts.
	A copy of the license is included in the section entitled "GNU
	Free Documentation License".
\end{verbatim}
}

\newpage
%\cleardoublepage

\pdfbookmark[section]{\contentsname}{toc}
\renewcommand{\contentsname}{Sommaire}
\tableofcontents{}

%\newpage


%\begin{figure}[!h]
%	\begin{center}
%		\includegraphics[scale=.35,angle=270]{images/DSC_0053.JPG}
%	\end{center}
%	\caption[]{\label{fig:result} Le résultat}
%\end{figure}

\section{RIOT}
Sur le site on peut trouver un bel article en introduction \cite{ref10}.

\subsection{Premier projet \cite{ref12} (architecture native)}
Cette architecture permet de faire fonctionner un projet sous forme de processus sous
Linux (32bits).\\

Sur mon système Debian 10 64bits (\enquote{crossgradée} depuis l'architecture 32bits)
j'ai pu effectuer cette procédure en quelques minutes car tous les outils nécessaires
étaient déjà installés.\\

Toutefois le lien concernant les dépendances ne traite que de la Debian 7.5 assez
ancienne et n'est donc plus fiable. Je vais donc recommencer la procédure de zéro
dans une VM 64 bits afin de déterminer ce qui est réellement nécessaire.

\subsubsection{Installation du système de base Debian 10 amd64}
Création d'une VM virtualbox 64bits avec les options par défaut.\\

Téléchargement d'une image iso de l'installeur :\\
{\footnotesize
\url{https://cdimage.debian.org/debian-cd/current/amd64/iso-cd/debian-10.3.0-amd64-netinst.iso}\\
}\\
%15h28
Installation du système de base avec uniquement \enquote{serveur SSH} et
\enquote{utilitaires usuels du système} lors de la sélection des logiciels.\\

Passer l'interface réseau en mode bridge.\\

Au premier démarrage installer net-tools, et identifier l'adresse réseau de la vm
avec ifconfig.\\

On peut maintenant s'y connecter par ssh.

\subsubsection{Installation des dépendances}
Dans un shell root :
%{\footnotesize
\begin{verbatim}
dpkg --add-architecture i386
apt-get update
apt-get install libc6-dev-i386 libc6-dbg:i386 build-essential pkg-config \
uml-utilities bridge-utils git unzip
\end{verbatim}
%}

\subsubsection{Préparation}
%{\scriptsize
\begin{verbatim}
git clone https://github.com/RIOT-OS/RIOT RIOT
cd RIOT/examples
cp -R default my_project
cd my_project
\end{verbatim}
%}
La page dit : \enquote{From this directory, you will compile everything you need,
including the RIOT OS itself– one small make and everything is ready.}.\\

Essayons :
%{\scriptsize
\begin{verbatim}
hugues@debian10base:~/RIOT/examples/my_project$ make
/home/hugues/RIOT/examples/my_project/../../Makefile.include:271: *** Neither
unzip nor 7z is installed.. Arrêt.
\end{verbatim}
%}
Je rajoute donc unzip dans la commande apt-get install de la section précédente.\\

Cette fois ça fonctionne :
%{\scriptsize
\begin{verbatim}
hugues@debian10base:~/RIOT/examples/my_project$ make
Building application "default" for "native" with MCU "native".

"make" -C /home/hugues/RIOT/boards/native
"make" -C /home/hugues/RIOT/boards/native/drivers
"make" -C /home/hugues/RIOT/core
"make" -C /home/hugues/RIOT/cpu/native
"make" -C /home/hugues/RIOT/cpu/native/netdev_tap
"make" -C /home/hugues/RIOT/cpu/native/periph
"make" -C /home/hugues/RIOT/cpu/native/stdio_native
"make" -C /home/hugues/RIOT/cpu/native/vfs
"make" -C /home/hugues/RIOT/drivers
"make" -C /home/hugues/RIOT/drivers/netdev_eth
"make" -C /home/hugues/RIOT/drivers/periph_common
"make" -C /home/hugues/RIOT/drivers/saul
"make" -C /home/hugues/RIOT/drivers/saul/init_devs
"make" -C /home/hugues/RIOT/sys
"make" -C /home/hugues/RIOT/sys/auto_init
"make" -C /home/hugues/RIOT/sys/fmt
"make" -C /home/hugues/RIOT/sys/iolist
"make" -C /home/hugues/RIOT/sys/net/gnrc
"make" -C /home/hugues/RIOT/sys/net/gnrc/netapi
"make" -C /home/hugues/RIOT/sys/net/gnrc/netif
"make" -C /home/hugues/RIOT/sys/net/gnrc/netif/ethernet
"make" -C /home/hugues/RIOT/sys/net/gnrc/netif/hdr
"make" -C /home/hugues/RIOT/sys/net/gnrc/netif/init_devs
"make" -C /home/hugues/RIOT/sys/net/gnrc/netreg
"make" -C /home/hugues/RIOT/sys/net/gnrc/pkt
"make" -C /home/hugues/RIOT/sys/net/gnrc/pktbuf
"make" -C /home/hugues/RIOT/sys/net/gnrc/pktbuf_static
"make" -C /home/hugues/RIOT/sys/net/gnrc/pktdump
"make" -C /home/hugues/RIOT/sys/net/link_layer/l2util
"make" -C /home/hugues/RIOT/sys/net/netif
"make" -C /home/hugues/RIOT/sys/od
"make" -C /home/hugues/RIOT/sys/phydat
"make" -C /home/hugues/RIOT/sys/ps
"make" -C /home/hugues/RIOT/sys/saul_reg
"make" -C /home/hugues/RIOT/sys/shell
"make" -C /home/hugues/RIOT/sys/shell/commands
   text	   data	    bss	    dec	    hex	filename
  91657	   1288	  72088	 165033	  284a9	/home/hugues/RIOT/examples/my_projec
t/bin/native/default.elf
\end{verbatim}
%}

\subsubsection{Test}
Créer l'interface tap0, dans le shell root :
%{\scriptsize
\begin{verbatim}
root@debian10base:~# tunctl -u hugues
Set 'tap0' persistent and owned by uid 1000
\end{verbatim}
%}

Lancer le projet :
{\small
\begin{verbatim}
hugues@debian10base:~/RIOT/examples/my_project$ ./bin/native/default.elf tap0
RIOT native interrupts/signals initialized.
Native RTC initialized.
LED_RED_OFF
LED_GREEN_ON
RIOT native board initialized.
RIOT native hardware initialization complete.

main(): This is RIOT! (Version: 2020.07-devel-244-gbb668)
Welcome to RIOT!
> help
help
Command              Description
---------------------------------------
reboot               Reboot the node
version              Prints current RIOT_VERSION
ps                   Prints information about running threads.
rtc                  control RTC peripheral interface
ifconfig             Configure network interfaces
txtsnd               Sends a custom string as is over the link layer
saul                 interact with sensors and actuators using SAUL
> ps
ps
	pid | name                 | state    Q | pri | stack  ( used) | base addr  | current     
	  - | isr_stack            | -        - |   - |   8192 (   -1) | 0x565e3520 | 0x565e3520
	  1 | idle                 | pending  Q |  15 |   8192 (  436) | 0x565e1240 | 0x565e30a0 
	  2 | main                 | running  Q |   7 |  12288 ( 3148) | 0x565de240 | 0x565e10a0 
	  3 | pktdump              | bl rx    _ |   6 |  12288 (  992) | 0x565e9c80 | 0x565ecae0 
	  4 | gnrc_netdev_tap      | bl rx    _ |   2 |   8192 ( 2444) | 0x565ecd00 | 0x565eeb60 
	    | SUM                  |            |     |  49152 ( 7020)
> rtc
rtc
usage: rtc <command> [arguments]
commands:
	poweron		power the interface on
	poweroff	power the interface off
	clearalarm	deactivate the current alarm
	getalarm	print the currently alarm time
	setalarm YYYY-MM-DD HH:MM:SS
			set an alarm for the specified time
	gettime		print the current time
	settime YYYY-MM-DD HH:MM:SS
			set the current time
> rtc gettime
rtc gettime
2020-04-29 11:45:16
> ifconfig
ifconfig
Iface  4  HWaddr: 3A:BE:D0:73:C8:37 
          L2-PDU:1500 Source address length: 6
          
> txtsnd help
txtsnd help
usage: txtsnd <if> [<L2 addr>|bcast] <data>
> txtsnd 4 bcast test    
txtsnd 4 bcast test
> 
\end{verbatim}
}

On voit un processus \enquote{pktdump} qui revoit probablement ce qui passe dans le
réseau sur la console. Si on active l'interface tap0 sur le système linux hôte
(ifconfig tap0 up), la console riot est immédiatement pourrie de paquets ipv6 :
{\small
\begin{verbatim}
> PKTDUMP: data received:
~~ SNIP  0 - size:  76 byte, type: NETTYPE_UNDEF (0)
00000000  60  00  00  00  00  24  00  01  00  00  00  00  00  00  00  00
00000010  00  00  00  00  00  00  00  00  FF  02  00  00  00  00  00  00
00000020  00  00  00  00  00  00  00  16  3A  00  05  02  00  00  01  00
00000030  8F  00  A6  E0  00  00  00  01  04  00  00  00  FF  02  00  00
00000040  00  00  00  00  00  00  00  01  FF  73  C8  36
~~ SNIP  1 - size:  20 byte, type: NETTYPE_NETIF (-1)
if_pid: 4  rssi: 0  lqi: 0
flags: 0x0
src_l2addr: 3A:BE:D0:73:C8:36
dst_l2addr: 33:33:00:00:00:16
~~ PKT    -  2 snips, total size:  96 byte
PKTDUMP: data received:
~~ SNIP  0 - size:  72 byte, type: NETTYPE_UNDEF (0)
00000000  60  00  00  00  00  20  3A  FF  00  00  00  00  00  00  00  00
00000010  00  00  00  00  00  00  00  00  FF  02  00  00  00  00  00  00
00000020  00  00  00  01  FF  73  C8  36  87  00  7C  10  00  00  00  00
\end{verbatim}
}

Pour éviter ça on peut désactiver l'ipv6 dans le noyau linux :
\begin{verbatim}
root@debian10base:~# echo 1 > /proc/sys/net/ipv6/conf/all/disable_ipv6 
\end{verbatim}

OK, on va essayer d'en faire communiquer 2 vite fait.\\

Pour cela on va créer une seconde interface et bridger les 2 :
%{\scriptsize
\begin{verbatim}
root@debian10base:~# tunctl -u hugues
Set 'tap1' persistent and owned by uid 1000
root@debian10base:~# ifconfig tap1 up
root@debian10base:~# brctl addbr br0
root@debian10base:~# brctl addif br0 tap0 tap1
root@debian10base:~# ifconfig br0 up
\end{verbatim}
%}
Puis on lance un second RIOT et on retente la commande txtsnd :
%{\scriptsize
\begin{verbatim}
hugues@debian10base:~/RIOT/examples/my_project$ ./bin/native/default.elf tap1
RIOT native interrupts/signals initialized.
Native RTC initialized.
LED_RED_OFF
LED_GREEN_ON
RIOT native board initialized.
RIOT native hardware initialization complete.

main(): This is RIOT! (Version: 2020.07-devel-244-gbb668)
Welcome to RIOT!
> txtsnd 4 bcast test 
txtsnd 4 bcast test 
> 
\end{verbatim}
%}
Dans la console du premier on voit apparaître :
%{\scriptsize
\begin{verbatim}
PKTDUMP: data received:
~~ SNIP  0 - size:   4 byte, type: NETTYPE_UNDEF (0)
00000000  74  65  73  74
~~ SNIP  1 - size:  20 byte, type: NETTYPE_NETIF (-1)
if_pid: 4  rssi: 0  lqi: 0
flags: 0x0
src_l2addr: 1E:FF:60:4B:51:E2
dst_l2addr: FF:FF:FF:FF:FF:FF
~~ PKT    -  2 snips, total size:  24 byte
^C
native: exiting
hugues@debian10base:~/RIOT/examples/my_project$ echo -n test | hd
00000000  74 65 73 74                                       |test|
00000004
hugues@debian10base:~/RIOT/examples/my_project$ /sbin/ifconfig tap1
tap1: flags=4099<UP,BROADCAST,MULTICAST>  mtu 1500
        ether 1e:ff:60:4b:51:e1  txqueuelen 1000  (Ethernet)
        RX packets 2  bytes 36 (36.0 B)
        RX errors 0  dropped 0  overruns 0  frame 0
        TX packets 0  bytes 0 (0.0 B)
        TX errors 0  dropped 0 overruns 0  carrier 0  collisions 0
\end{verbatim}
%}
On observe que \enquote{00000000  74  65  73  74} correspond bien à la chaîne de
caractères \enquote{test} envoyée et que \enquote{src\_l2addr: 1E:FF:60:4B:51:E2}
est l'adresse MAC de la source.


\subsection{Initial Setup and Tutorials \cite{ref13}}
\subsubsection{Préparations (Configuration sans machine virtuelle)}
Liste des prérequis\\
(depuis \url{https://github.com/RIOT-OS/RIOT/wiki/Creating-your-first-RIOT-project}) :
\begin{itemize}
	\item Install and set up git
	\item Install the build-essential packet (make, gcc etc.). This varies based on
			the operating system in use.
	\item Install Native dependencies
	\item Install OpenOCD
	\item Install GCC Arm Embedded Toolchain
	\item On OS X: install Tuntap for OS X
	\item additional tweaks necessary to work with the targeted hardware (ATSAMR21)
	\item Install netcat with IPv6 support (if necessary)
			\begin{verbatim}sudo apt-get install netcat-openbsd}\end{verbatim}
	\item \begin{verbatim}git clone --recursive https://github.com/RIOT-OS/Tutorials\end{verbatim}
	\item Go to the Tutorials directory: cd Tutorials\\
\end{itemize}

Les 3 premiers points sont déjà OK sur ma machine.
\paragraph{OpenOCD}~\\

\url{https://github.com/RIOT-OS/RIOT/wiki/OpenOCD} :
\begin{verbatim}
OpenOCD (the open on-chip debugger) is an open source tool for debugging and
flashing microcontrollers. In RIOT we try to use this tool for as many
platforms as possible to reduce the overhead of having to keep track of many
different (and sometimes proprietary) tools.
\end{verbatim}

C'est donc l'outil qui devrait nous permettre de téléverser notre firmware. La page
explique son installation à partir des sources car les versions empaquetés dans les
distributions ne supporte pas les cartes les plus récentes supportées par RIOT. Cette
page datant de 2018 elle parle de la version 0.7.0 sur Mint 17 et de la version 0.9.0
compilée depuis les sources.\\

La version distribuée sur Debian 10 étant la 0.10.0 je vais m'en contenter dans un
premier temps et j'y reviendrai si nécessaire.

\paragraph{GCC Arm Embedded Toolchain}~\\

Sur Debian c'est le package gcc-arm-none-eabi déjà présent sur ma machine en versions
7-2018-q2, la version actuelle étant la 9-2019-q4. Comme pour OpenOCD je vais essayer
de commencer avec ça.

\paragraph{En résumé}~\\

Dans un shell root :
%{\scriptsize
\begin{verbatim}
dpkg --add-architecture i386
apt-get update
apt-get install libc6-dev-i386 libc6-dbg:i386 build-essential pkg-config \
uml-utilities bridge-utils git unzip
apt-get install openocd gcc-arm-none-eabi
\end{verbatim}
%}
Dans un shell utilisateur :
%{\scriptsize
\begin{verbatim}
git clone --recursive https://github.com/RIOT-OS/Tutorials
cd Tutorials
\end{verbatim}
%}

Facile !



\subsubsection{Task 1: Starting the RIOT}
\input{1.2.2-task1.tex}

\subsubsection{Task 2: Custom shell handlers}
Il s'agit d'écrire 2 gestionnaires de commande shell :
\begin{itemize}
	\item une commande echo ;
	\item une commande toggle pour une led.\\
\end{itemize}

Lors de la compilation j'ai recontré le problème suivant :
{\scriptsize
\begin{verbatim}
hugues@W520:~/Tutorials/task-02$ make all flash term
Building application "Task02" for "bluepill" with MCU "stm32f1".

In file included from /home/hugues/Tutorials/RIOT/boards/bluepill/include/board.h:32:0,
                 from /home/hugues/Tutorials/RIOT/drivers/include/led.h:35,
                 from /home/hugues/Tutorials/task-02/main.c:5:
/home/hugues/Tutorials/task-02/main.c: In function 'toggle':
/home/hugues/Tutorials/RIOT/boards/common/blxxxpill/include/board_common.h:36:29: error: 'GPIOC' undeclared (first u
se in this function)
 #define LED0_PORT           GPIOC                                   /**< GPIO-Port the LED is connected to */
                             ^
/home/hugues/Tutorials/RIOT/boards/common/blxxxpill/include/board_common.h:49:30: note: in expansion of macro 'LED0_
PORT'
 #define LED0_TOGGLE         (LED0_PORT->ODR  ^= LED0_MASK)          /**< Toggle LED0 */
                              ^~~~~~~~~
/home/hugues/Tutorials/task-02/main.c:33:2: note: in expansion of macro 'LED0_TOGGLE'
  LED0_TOGGLE;
  ^~~~~~~~~~~
/home/hugues/Tutorials/RIOT/boards/common/blxxxpill/include/board_common.h:36:29: note: each undeclared identifier i
s reported only once for each function it appears in
 #define LED0_PORT           GPIOC                                   /**< GPIO-Port the LED is connected to */
                             ^
/home/hugues/Tutorials/RIOT/boards/common/blxxxpill/include/board_common.h:49:30: note: in expansion of macro 'LED0_
PORT'
 #define LED0_TOGGLE         (LED0_PORT->ODR  ^= LED0_MASK)          /**< Toggle LED0 */
                              ^~~~~~~~~
/home/hugues/Tutorials/task-02/main.c:33:2: note: in expansion of macro 'LED0_TOGGLE'
  LED0_TOGGLE;
  ^~~~~~~~~~~
make[1]: *** [/home/hugues/Tutorials/RIOT/Makefile.base:110: /home/hugues/Tutorials/task-02/bin/bluepill/application
_Task01/main.o] Error 1
make: *** [/home/hugues/Tutorials/task-02/../RIOT/Makefile.include:538: /home/hugues/Tutorials/task-02/bin/bluepill/
application_Task01.a] Error 2
\end{verbatim}
}
Le problème ne se produit qu'avec BOARD=bluepill, \texttt{make all}
fonctionne bien avec \enquote{native}, \enquote{nucleo-f746zg}, ou même
\enquote{nucleo-f103rb}.\\

Le problème est peut-être lié à l'ordre de traitement des dépendances
 : \url{https://github.com/RIOT-OS/RIOT/issues/9913}.\\
 
J'ai pu le contourner par une inclusion conditionnelle :
\begin{lstlisting}
#ifdef BOARD_BLUEPILL
#include "vendor/stm32f103xe.h"
#endif
\end{lstlisting}


Voilà ce que donne le fichier main.c :
\begin{lstlisting}
#include <stdio.h>
#include <string.h>

#include "shell.h"
#include "led.h"

#ifdef BOARD_BLUEPILL
#include "vendor/stm32f103xe.h"
#endif

int echo(int argc, char **argv)
{
    /* Print a line of text */
    (void)argc;
    (void)argv;
    
    if (argc > 1) {
    	for (int i = 1; i < argc-1; i++) {
	        printf("%s ", argv[i]);
	    }
		printf("%s\n", argv[argc-1]);
    }
    else {
		printf("\n");
	}
	
    return 0;
}

int toggle(int argc, char **argv)
{
	/** Toggles the primary LED on the board **/
    (void)argc;
    (void)argv;

	LED0_TOGGLE;
	
	return 0;
}

static const shell_command_t commands[] = {
    { "echo", "Print a line of text", echo },
    { "toggle", "Toggles the primary LED on the board", toggle },
    { NULL, NULL, NULL }
};

int main(void)
{
    puts("This is Task-02");

    char line_buf[SHELL_DEFAULT_BUFSIZE];
    shell_run(commands, line_buf, SHELL_DEFAULT_BUFSIZE);

    return 0;
}
\end{lstlisting}

Sur l'arduino ça ne passe plus :
{\scriptsize
\begin{verbatim}
/usr/lib/gcc/avr/5.4.0/../../../avr/bin/ld : l'adresse 0x800871 de /home/hugues/Tutorials/task-02/bin/arduino-nano
/Task02.elf de la section «.bss» n'est pas dans la région «data»
\end{verbatim}
}
Pour que ça passe on peut réduire la valeur de
\texttt{THREAD\_STACKSIZE\_DEFAULT} dans le fichier
\texttt{RIOT/cpu/atmega\_common/include/cpu\_conf.h} de 512 à 256 :
{\scriptsize
\begin{verbatim}
2020-05-06 15:24:43,923 # 	pid | name                 | state    Q | pri | stack  ( used) | base addr  | current     
2020-05-06 15:24:44,023 # 	  1 | idle                 | pending  Q |  15 |    128 (   84) |      0x4b2 |      0x4df 
2020-05-06 15:24:44,123 # 	  2 | main                 | running  Q |   7 |    384 (  336) |      0x532 |      0x5a9 
2020-05-06 15:24:44,166 # 	    | SUM                  |            |     |    512 (  420)
\end{verbatim}
}
La commande \texttt{toggle} ne fonctionne pas.\\

Il y avait une erreur dans le fichier :\\
\texttt{RIOT/boards/common/arduino-atmega/include/board\_common.h}.\\

Le patch pour que ça fonctionne sur le nano :
\begin{lstlisting}
--- ../RIOT/boards/common/arduino-atmega/include/board_common.h	2020-04-27 21:01:08.376660474 +0200
+++ RIOT/boards/common/arduino-atmega/include/board_common.h	2020-05-06 16:14:12.768798005 +0200
@@ -76,9 +76,9 @@
 #define LED2_ON             (PORTD &= ~LED2_MASK)
 #define LED2_TOGGLE         (PORTD ^=  LED2_MASK)
 #else
-#define LED0_ON             (PORTD |=  LED0_MASK)
-#define LED0_OFF            (PORTD &= ~LED0_MASK)
-#define LED0_TOGGLE         (PORTD ^=  LED0_MASK)
+#define LED0_ON             (PORTB |=  LED0_MASK)
+#define LED0_OFF            (PORTB &= ~LED0_MASK)
+#define LED0_TOGGLE         (PORTB ^=  LED0_MASK)
 #endif
 /** @} */
\end{lstlisting}


\subsubsection{Task 3: Multithreading}
\input{1.2.4-task3.tex}

\subsubsection{Task 4: Timers}
Utiliser \texttt{xtimer} pour créer un thread qui affiche le temps
système courant toutes les 2 secondes.\\

Le thread\_handler :
\begin{lstlisting}
void *thread_handler(void *arg)
{
    /* Thread that print the current system time every 2 seconds... */
    (void)arg;
    
    while(true){
		printf("Current system time is %ld microseconds\n",xtimer_now_usec());
		xtimer_sleep(2);
	}
    
    return NULL;
}
\end{lstlisting}

Le résultat :
{\scriptsize
\begin{verbatim}
2020-05-07 16:22:47,041 # Current system time is 5090 microseconds
> 2020-05-07 16:22:49,044 #  Current system time is 2008736 microseconds
2020-05-07 16:22:51,048 # Current system time is 4012642 microseconds
2020-05-07 16:22:53,051 # Current system time is 6016549 microseconds
2020-05-07 16:22:55,055 # Current system time is 8020456 microseconds
2020-05-07 16:22:57,059 # Current system time is 10024362 microseconds
ps
2020-05-07 16:22:57,812 # ps
2020-05-07 16:22:57,819 # 	pid | name                 | state    Q | pri | stack  ( used) | base addr  | current     
2020-05-07 16:22:57,827 # 	  - | isr_stack            | -        - |   - |    512 (  156) | 0x20000000 | 0x200001c8
2020-05-07 16:22:57,835 # 	  1 | idle                 | pending  Q |  15 |    256 (  132) | 0x200003f8 | 0x20000474 
2020-05-07 16:22:57,844 # 	  2 | main                 | running  Q |   7 |   1536 (  676) | 0x200004f8 | 0x2000094c 
2020-05-07 16:22:57,852 # 	  3 | thread               | bl mutex _ |   6 |   1536 (  356) | 0x20000afc | 0x2000102c 
2020-05-07 16:22:57,858 # 	    | SUM                  |            |     |   3840 ( 1320)
> 2020-05-07 16:22:59,063 #  Current system time is 12028356 microseconds
2020-05-07 16:23:01,066 # Current system time is 14032358 microseconds
2020-05-07 16:23:03,071 # Current system time is 16036360 microseconds
2020-05-07 16:23:05,074 # Current system time is 18040362 microseconds
\end{verbatim}
}

Il y a un écart d'environ 4ms. J'ai modifié le programme pour mesurer
la durée de la boucle :
{\scriptsize
\begin{verbatim}
2020-05-07 16:51:05,767 # Loop duration 7673 microseconds
2020-05-07 16:51:07,771 # Loop duration 2002858 microseconds
2020-05-07 16:51:09,774 # Loop duration 2003119 microseconds
2020-05-07 16:51:11,777 # Loop duration 2003118 microseconds
2020-05-07 16:51:13,780 # Loop duration 2003118 microseconds
2020-05-07 16:51:15,783 # Loop duration 2003118 microseconds
2020-05-07 16:51:17,786 # Loop duration 2003118 microseconds
2020-05-07 16:51:19,789 # Loop duration 2003118 microseconds
2020-05-07 16:51:21,792 # Loop duration 2003118 microseconds
2020-05-07 16:51:23,795 # Loop duration 2003118 microseconds
2020-05-07 16:51:25,798 # Loop duration 2003118 microseconds
2020-05-07 16:51:27,800 # Loop duration 2003118 microseconds
\end{verbatim}
}

On peut voir qu'il a l'air de se stabiliser après 3 itérations.
Maintenant un essai de compensation avec
\texttt{xtimer\_usleep(1996882);} (2000000-3118=1996882):
{\scriptsize
\begin{verbatim}
2020-05-07 17:00:23,165 # Loop duration 7673 microseconds
2020-05-07 17:00:25,166 # Loop duration 1999738 microseconds
2020-05-07 17:00:27,165 # Loop duration 2000007 microseconds
2020-05-07 17:00:29,165 # Loop duration 2000007 microseconds
2020-05-07 17:00:31,165 # Loop duration 2000006 microseconds
2020-05-07 17:00:33,165 # Loop duration 2000007 microseconds
2020-05-07 17:00:35,165 # Loop duration 2000007 microseconds
2020-05-07 17:00:37,165 # Loop duration 2000006 microseconds
2020-05-07 17:00:39,164 # Loop duration 2000007 microseconds
2020-05-07 17:00:41,164 # Loop duration 2000007 microseconds
\end{verbatim}
}

On y arrive pas de cette manière : avec un sleep time de 1996875
à 1996881 on obtient 1999998, et ça passe à 2000007 à partir de
1996882.\\

Le printf imprime sur la console à 115200 bauds ce qui prend environ
87 $\mu$s par caractère.\\

\begin{comment}
Bof... en enlevant encore 7 ?
{\scriptsize
\begin{verbatim}
2020-05-07 17:06:51,271 # Loop duration 7673 microseconds
2020-05-07 17:06:53,272 # Loop duration 1999730 microseconds
2020-05-07 17:06:55,272 # Loop duration 1999998 microseconds
2020-05-07 17:06:57,272 # Loop duration 1999998 microseconds
2020-05-07 17:06:59,272 # Loop duration 1999998 microseconds
2020-05-07 17:07:01,272 # Loop duration 1999998 microseconds
2020-05-07 17:07:03,271 # Loop duration 1999998 microseconds
2020-05-07 17:07:05,271 # Loop duration 1999998 microseconds
\end{verbatim}
}

Toujours pas... +2 ?
{\scriptsize
\begin{verbatim}
2020-05-07 17:09:14,078 # Loop duration 1999738 microseconds
2020-05-07 17:09:16,077 # Loop duration 1999998 microseconds
2020-05-07 17:09:18,077 # Loop duration 1999998 microseconds
2020-05-07 17:09:20,077 # Loop duration 1999998 microseconds
2020-05-07 17:09:22,077 # Loop duration 1999998 microseconds
2020-05-07 17:09:24,077 # Loop duration 1999998 microseconds
\end{verbatim}
}

Avec des valeurs de sleep de 1996875 à 1996881 on obtient 1999998, et ça
passe à 2000007 à partir de 1996882, on y arrivera pas.

L'essentiel de la durée d'exécution de la boucle est lié au 
\texttt{printf}. On peut donc obtenir quelque chose de
beaucoup plus précis le mettant dans un autre thread :
{\scriptsize
\begin{verbatim}
2020-05-07 17:35:48,433 # Loop duration 2000024 microseconds
2020-05-07 17:35:50,436 # Loop duration 2000025 microseconds
2020-05-07 17:35:52,439 # Loop duration 2000025 microseconds
2020-05-07 17:35:54,442 # Loop duration 2000025 microseconds
ps
2020-05-07 17:35:56,135 # ps
2020-05-07 17:35:56,141 # 	pid | name                 | state    Q | pri | stack  ( used) | base addr  | current     
2020-05-07 17:35:56,149 # 	  - | isr_stack            | -        - |   - |    512 (  156) | 0x20000000 | 0x200001c8
2020-05-07 17:35:56,157 # 	  1 | idle                 | pending  Q |  15 |    256 (  132) | 0x200003f8 | 0x20000474 
2020-05-07 17:35:56,166 # 	  2 | main                 | running  Q |   7 |   1536 (  688) | 0x200004f8 | 0x2000094c 
2020-05-07 17:35:56,175 # 	  3 | thread               | bl mutex _ |   5 |   1536 (  208) | 0x20000b04 | 0x20001034 
2020-05-07 17:35:56,183 # 	  4 | thread2              | bl mutex _ |   6 |   1536 (  356) | 0x20001104 | 0x20001634 
2020-05-07 17:35:56,189 # 	    | SUM                  |            |     |   5376 ( 1540)
> 2020-05-07 17:35:56,444 #  Loop duration 2000025 microseconds
2020-05-07 17:35:58,447 # Loop duration 2000025 microseconds
2020-05-07 17:36:00,450 # Loop duration 2000025 microseconds
2020-05-07 17:36:02,453 # Loop duration 2000025 microseconds
\end{verbatim}
}

C'est mieux ! Voyons si on peut compenser ces 25$\mu$ :
{\scriptsize
\begin{verbatim}
2020-05-07 17:46:00,323 # Loop duration 1999999 microseconds
2020-05-07 17:46:02,327 # Loop duration 2000000 microseconds
2020-05-07 17:46:04,329 # Loop duration 2000000 microseconds
2020-05-07 17:46:06,332 # Loop duration 2000000 microseconds
2020-05-07 17:46:08,335 # Loop duration 2000000 microseconds
2020-05-07 17:46:10,338 # Loop duration 2000000 microseconds
2020-05-07 17:46:12,341 # Loop duration 2000000 microseconds
2020-05-07 17:46:14,344 # Loop duration 2000000 microseconds
2020-05-07 17:46:16,347 # Loop duration 2000000 microseconds
2020-05-07 17:46:18,350 # Loop duration 2000000 microseconds
2020-05-07 17:46:20,353 # Loop duration 2000001 microseconds
2020-05-07 17:46:22,356 # Loop duration 2000000 microseconds
2020-05-07 17:46:24,358 # Loop duration 2000000 microseconds
2020-05-07 17:46:26,361 # Loop duration 2000000 microseconds
\end{verbatim}
}

On arrive quand même à quelque chose de relativement précis et stable.
Là j'étais sur la bleupill, évidament ce n'est pas portable, sur la
nucleo ça donne :
{\scriptsize
\begin{verbatim}
2020-05-07 17:59:44,341 # Loop duration 1999993 microseconds
2020-05-07 17:59:46,344 # Loop duration 1999994 microseconds
2020-05-07 17:59:48,347 # Loop duration 1999993 microseconds
2020-05-07 17:59:50,350 # Loop duration 1999993 microseconds
2020-05-07 17:59:52,353 # Loop duration 1999993 microseconds
2020-05-07 17:59:54,355 # Loop duration 1999993 microseconds
2020-05-07 17:59:56,359 # Loop duration 1999993 microseconds
2020-05-07 17:59:58,362 # Loop duration 1999993 microseconds
\end{verbatim}
}
\end{comment}

J'ai donc sorti le printf dans un autre thread en utilisant le système
de message IPC de RIOT (voir
\url{https://github.com/RIOT-OS/RIOT/tree/master/examples/ipc_pingpong}.
\\

Le main.c :
\begin{lstlisting}
#include <stdio.h>
#include <string.h>

#include "shell.h"
#include "thread.h"
#include "xtimer.h"
#include "msg.h"

#define PERIOD_US 2000000

char stack_timer2s[THREAD_STACKSIZE_MAIN];
char stack_printf2s[THREAD_STACKSIZE_MAIN];

void *printf2s_handler(void *arg)
{
    /* Thread that print the current system time every 2 seconds... */
    (void)arg;
    msg_t t;
    uint32_t time; // the current system time received from timer2s
    uint32_t duration; // the real loop duration from previous time
    uint32_t prev_time = 0; // to save the previous time
    uint32_t gap = 0; // the gap between sleep_time and PERIOD_US
    uint32_t sleep_time = PERIOD_US;
    uint32_t before_printf_time = 0;
    uint32_t after_printf_time = 0;
    uint32_t prev_before_printf_time = 0;

    while (1) {
        msg_receive(&t);
        time = t.content.value;
        duration = time - prev_time;
        gap = prev_time ? duration - PERIOD_US : 0;// not the first time
        if (gap) {
            sleep_time -= gap;
        }
        t.content.value = sleep_time;
        msg_reply(&t, &t);
        before_printf_time = xtimer_now_usec();
        printf("system time: %lu £\color{orange}$\mu$£s, last period: %lu £\color{orange}$\mu$£s, next sleep\
 time: %lu £\color{orange}$\mu$£s, time gap: %lu £\color{orange}$\mu$£s, last printf: %lu £\color{orange}$\mu$£s\n",
            time, duration, sleep_time, PERIOD_US - sleep_time,
            after_printf_time - prev_before_printf_time);
        after_printf_time = xtimer_now_usec();
        prev_before_printf_time = before_printf_time;
        prev_time = time;
    }
    
    return NULL;
}

void *timer2s_handler(void *arg)
{
    /* Thread that save the current system time every 2 seconds... */
    (void)arg;

    msg_t t;
    
    kernel_pid_t pid = thread_create(stack_printf2s,
                            sizeof(stack_printf2s),
                            THREAD_PRIORITY_MAIN - 1,
                            THREAD_CREATE_STACKTEST,
                            printf2s_handler, NULL,
                            "printf2s");
    
    while (1) {
        t.content.value = xtimer_now_usec();
        msg_send_receive(&t, &t, pid);
        xtimer_usleep(t.content.value);
    }

    return NULL;
}

int main(void)
{
    puts("This is Task-04");

    thread_create(stack_timer2s, sizeof(stack_timer2s),
                  THREAD_PRIORITY_MAIN - 2,
                  THREAD_CREATE_STACKTEST,
                  timer2s_handler, NULL,
                  "timer2s");
 
    char line_buf[SHELL_DEFAULT_BUFSIZE];
    shell_run(NULL, line_buf, SHELL_DEFAULT_BUFSIZE);

    return 0;
}
\end{lstlisting}

La durée du sleep est ajustée automatique lors des premières itérations
ce qui rend le code portable.\\

Le résultat sur la bluepill :
{\tiny
\begin{alltt}
2020-05-13 10:36:46,578 # main(): This is RIOT! (Version: 2020.04)
2020-05-13 10:36:46,581 # This is Task-04
2020-05-13 10:36:46,590 # system time: 5123 \(\mu\)s, last period: 5123 \(\mu\)s, next sleep time: 2000000 \(\mu\)s, time gap: 0 \(\mu\)s, last printf: 0 \(\mu\)s
> 2020-05-13 10:36:48,592 #  system time: 2005164 \(\mu\)s, last period: 2000041 \(\mu\)s, next sleep time: 1999959 \(\mu\)s, time gap: 41 \(\mu\)s, last printf: 9911 \(\mu\)s
2020-05-13 10:36:50,592 # system time: 4005160 \(\mu\)s, last period: 1999996 \(\mu\)s, next sleep time: 1999963 \(\mu\)s, time gap: 37 \(\mu\)s, last printf: 10791 \(\mu\)s
ps
2020-05-13 10:36:51,307 # ps
2020-05-13 10:36:51,313 # 	pid | name                 | state    Q | pri | stack  ( used) | base addr  | current     
2020-05-13 10:36:51,322 # 	  - | isr_stack            | -        - |   - |    512 (  156) | 0x20000000 | 0x200001c8
2020-05-13 10:36:51,330 # 	  1 | idle                 | pending  Q |  15 |    256 (  132) | 0x200003f8 | 0x20000474 
2020-05-13 10:36:51,339 # 	  2 | main                 | running  Q |   7 |   1536 (  676) | 0x200004f8 | 0x2000094c 
2020-05-13 10:36:51,347 # 	  3 | timer2s              | bl mutex _ |   5 |   1536 (  232) | 0x200010fc | 0x20001614 
2020-05-13 10:36:51,356 # 	  4 | printf2s             | bl rx    _ |   6 |   1536 (  404) | 0x20000afc | 0x20001034 
2020-05-13 10:36:51,361 # 	    | SUM                  |            |     |   5376 ( 1600)
> 2020-05-13 10:36:52,591 #  system time: 6005160 \(\mu\)s, last period: 2000000 \(\mu\)s, next sleep time: 1999963 \(\mu\)s, time gap: 37 \(\mu\)s, last printf: 10871 \(\mu\)s
2020-05-13 10:36:54,591 # system time: 8005160 \(\mu\)s, last period: 2000000 \(\mu\)s, next sleep time: 1999963 \(\mu\)s, time gap: 37 \(\mu\)s, last printf: 10877 \(\mu\)s
2020-05-13 10:36:56,591 # system time: 10005160 \(\mu\)s, last period: 2000000 \(\mu\)s, next sleep time: 1999963 \(\mu\)s, time gap: 37 \(\mu\)s, last printf: 10875 \(\mu\)s
2020-05-13 10:36:58,591 # system time: 12005160 \(\mu\)s, last period: 2000000 \(\mu\)s, next sleep time: 1999963 \(\mu\)s, time gap: 37 \(\mu\)s, last printf: 10960 \(\mu\)s
2020-05-13 10:37:00,591 # system time: 14005160 \(\mu\)s, last period: 2000000 \(\mu\)s, next sleep time: 1999963 \(\mu\)s, time gap: 37 \(\mu\)s, last printf: 10958 \(\mu\)s
2020-05-13 10:37:02,590 # system time: 16005160 \(\mu\)s, last period: 2000000 \(\mu\)s, next sleep time: 1999963 \(\mu\)s, time gap: 37 \(\mu\)s, last printf: 10965 \(\mu\)s
\end{alltt}
}

Sur la nucleo :
{\tiny
\begin{alltt}
2020-05-13 15:32:06,195 # This is Task-04
2020-05-13 15:32:06,205 # system time: 5035 \(\mu\)s, last period: 5035 \(\mu\)s, next sleep time: 2000000 \(\mu\)s, time gap: 0 \(\mu\)s, last printf: 0 \(\mu\)s
> 2020-05-13 15:32:08,206 #  system time: 2005065 \(\mu\)s, last period: 2000030 \(\mu\)s, next sleep time: 1999970 \(\mu\)s, time gap: 30 \(\mu\)s, last printf: 9807 \(\mu\)s
2020-05-13 15:32:10,206 # system time: 4005062 \(\mu\)s, last period: 1999997 \(\mu\)s, next sleep time: 1999973 \(\mu\)s, time gap: 27 \(\mu\)s, last printf: 10679 \(\mu\)s
2020-05-13 15:32:12,206 # system time: 6005061 \(\mu\)s, last period: 1999999 \(\mu\)s, next sleep time: 1999974 \(\mu\)s, time gap: 26 \(\mu\)s, last printf: 10766 \(\mu\)s
2020-05-13 15:32:14,206 # system time: 8005061 \(\mu\)s, last period: 2000000 \(\mu\)s, next sleep time: 1999974 \(\mu\)s, time gap: 26 \(\mu\)s, last printf: 10766 \(\mu\)s
2020-05-13 15:32:16,206 # system time: 10005061 \(\mu\)s, last period: 2000000 \(\mu\)s, next sleep time: 1999974 \(\mu\)s, time gap: 26 \(\mu\)s, last printf: 10766 \(\mu\)s
2020-05-13 15:32:18,206 # system time: 12005061 \(\mu\)s, last period: 2000000 \(\mu\)s, next sleep time: 1999974 \(\mu\)s, time gap: 26 \(\mu\)s, last printf: 10853 \(\mu\)s
2020-05-13 15:32:20,207 # system time: 14005061 \(\mu\)s, last period: 2000000 \(\mu\)s, next sleep time: 1999974 \(\mu\)s, time gap: 26 \(\mu\)s, last printf: 10853 \(\mu\)s
\end{alltt}
}

J'ai aussi effectué un test avec \enquote{BOARD=native}, évidemment ça
ne se stabilise jamais.\\

On peut voir qu'il faut 2 ou 3 périodes pour que le "sleep\_time" soit
ajusté. C'est dû au fait que le thread printf2s n'est pas totalement
prêt à traiter l'IPC lors de la première itération de la boucle. On peut
facilement résoudre ce problème en ajoutant un délai de 30$\mu$s entre
la création du thread et la boucle. Ainsi la période est ajustée dès la
seconde itération.\\

J'ai aussi rajouté une constante qui permet de prédéfinir l'écart s'il
est connu pour que la période soit juste dès la première itération. Ça a
bien fonctionné sur la bluepill, un peu mois sur la nucleo où la
première fais 1$\mu$s de trop et la seconde 1$\mu$s de moins avec ce
code.


\subsubsection{Task 5: Using network devices}
\input{1.2.6-task5.tex}

\subsubsection{Task 6: UDP Client / Server}
Création du réseau virtuel :
\begin{verbatim}
root@W520:~# tunctl -u 1000
Set 'tap0' persistent and owned by uid 1000
root@W520:~# tunctl -u 1000
Set 'tap1' persistent and owned by uid 1000
root@W520:~# brctl addbr br0
root@W520:~# brctl addif br0 tap0 tap1
root@W520:~# ifconfig br0 up
root@W520:~# ifconfig tap0 up
root@W520:~# ifconfig tap1 up
\end{verbatim}

La communication entre les 2 riot natif fonctionner comme prévue.\\

La commande netcat sur ma machine n'implémentait pas l'IPv6 :
{\footnotesize
\begin{verbatim}
hugues@W520:~/Tutorials/task-06$ echo "hello" | nc -6u fe80::ac51:57ff:fe42:8286%br0 8888
nc: invalid option -- '6'
nc -h for help
\end{verbatim}
}
J'y ai donc installé le package \texttt{netcat-openbsd}.\\

J'ai utilisé \texttt{br0} au lieu de \texttt{tapbr0} dans la commandes
linux.\\

Dans ma configuration br0 a la même adresse IPv6 que tap1, c'est donc
cette adresse que j'ai du utiliser pour envoyer un message de RIOT 
$\rightarrow$ l'hôte linux.\\

Avec la commande \texttt{nc -6lu 8888} je ne pouvais recevoir q'un seul
message, après quoi elle restait active mais ne recevait plus rien. On
peut recevoir plusieurs messages de suite en ajoutant l'option
\enquote{k}.\\

\paragraph{Test sur la nucleo\\\\}
D'abord je récupère l'adresse IPv6 de l'interface wifi de mon portable 
et je lance un serveur UDP avec netcat :
{\scriptsize
\begin{verbatim}
hugues@W520:~/Tutorials/task-06$ /sbin/ifconfig wlp3s0
wlp3s0: flags=4163<UP,BROADCAST,RUNNING,MULTICAST>  mtu 1500
        inet 192.168.21.32  netmask 255.255.255.0  broadcast 192.168.21.255
        inet6 2a01:e34:ed4f:42d0:a054:e4fc:dea1:f876  prefixlen 64  scopeid 0x0<global>
        inet6 fe80::9588:ffe2:5b3e:7c2b  prefixlen 64  scopeid 0x20<link>
        ether 08:11:96:24:86:9c  txqueuelen 1000  (Ethernet)
        RX packets 62172  bytes 19221714 (18.3 MiB)
        RX errors 0  dropped 0  overruns 0  frame 0
        TX packets 56365  bytes 8802188 (8.3 MiB)
        TX errors 0  dropped 0 overruns 0  carrier 0  collisions 0
hugues@W520:~/Tutorials/task-06$ nc -6luk 8888
\end{verbatim}
}
Il faut prendre la seconde (\texttt{<link>}) : 
\texttt{fe80::9588:ffe2:5b3e:7c2b}.\\

Puis je flash la nucleo, j'essaie un ping vers mon portable et un petit
\enquote{\texttt{hello}} :
{\scriptsize
\begin{verbatim}
> ping6 fe80::9588:ffe2:5b3e:7c2b
2020-05-15 00:11:50,930 #  ping6 fe80::9588:ffe2:5b3e:7c2b
2020-05-15 00:11:50,967 # 12 bytes from fe80::9588:ffe2:5b3e:7c2b%5: icmp_seq=0 ttl=64 time=30.498 ms
2020-05-15 00:11:51,938 # 12 bytes from fe80::9588:ffe2:5b3e:7c2b%5: icmp_seq=1 ttl=64 time=1.171 ms
2020-05-15 00:11:52,938 # 12 bytes from fe80::9588:ffe2:5b3e:7c2b%5: icmp_seq=2 ttl=64 time=1.153 ms
2020-05-15 00:11:52,938 # 
2020-05-15 00:11:52,942 # --- fe80::9588:ffe2:5b3e:7c2b PING statistics ---
2020-05-15 00:11:52,947 # 3 packets transmitted, 3 packets received, 0% packet loss
2020-05-15 00:11:52,951 # round-trip min/avg/max = 1.153/10.940/30.498 ms
> udp fe80::9588:ffe2:5b3e:7c2b 8888 hello
2020-05-15 00:16:03,049 #  udp fe80::9588:ffe2:5b3e:7c2b 8888 hello
2020-05-15 00:16:03,054 # Success: send 5 byte to fe80::9588:ffe2:5b3e:7c2b
\end{verbatim}
}
Sur le portable j'obtiens bien :
\begin{verbatim}
hugues@W520:~/Tutorials/task-06$ nc -6luk 8888
hello
\end{verbatim}


\subsubsection{Task 7: The GNRC network stack}
C'est un peu la même chose que les précédents en utilisant les
exemples d'applications du dépôt.

\subsubsection{Task 8: CCN-Lite on RIOT}
CCN: Content Centric Networking.\\
\url{https://en.wikipedia.org/wiki/Content_centric_networking}

\subsubsection{Task 9: RIOT and RPL}
RPL: IPv6 Routing Protocol for Low-Power and Lossy Networks.\\
Ça aussi on va se le garder pour plus tard.

\subsubsection{Observations}
\paragraph{Temps réel}~\\

Sur la page d'accueil on peut lire : \texttt{Real-time capability due to
ultra-low interrupt latency (~50 clock cycles) and priority-based
scheduling}. Dans notre cas (216MHz) ça représente 231 ns.\\

Mais dans les \enquote{Release Notes} il est décrit comme : \texttt{RIOT
is a multi-threading operating system which enables \underline{soft}
real-time capabilities}, ce qui est inquiétant vu qu'on a besoin de
temps réel \enquote{hard} pour certaines fonctionnalités.\\

Par ailleurs dans l'article d'introduction \cite{ref10} il est précisé :
\begin{verbatim}
In terms of speed, the delay incurred by using IPC decomposes into the
time for (i) saving and restoring threadcontexts, (ii) the runtime of
the scheduler and (iii) the runtime of the IPC submodule itself. While
(i) is entirely determined by the CPU architecture, (ii) is constant as
described in Section V-B, and a slim design of msgmakes (iii) small
overhead compared to (i) and (ii). [...] In more detail: the time needed
to interrupt and switch to a different thread will not exceed a (small)
upper bound, since context saving, finding the nextthread to run, and
context restoring are all deterministic operations.
\end{verbatim}
ce qui est rassurant.\\

On aura besoin de temps réel \enquote{hard} pour gérer des boucles de
mesures et correction des pwm. Une mesure avec l'ADC MCP3208 prend au
moins 12 $\mu$s et la fréquence maximale prévue pour le PWM est de 200
KHz soit une période de 5 $\mu$s ce qui est grand au regard de la
latence d'interruption d'environ 200 ns.\\

Contrairement aux systèmes basés sur des FPGA pour lesquels les
datasheets donnent des timings extrêmement précis pour chaque
entrée/sortie, dont les synthétiseurs logiques garantissent les durées
de latence et avec lesquels ont peut tout paralléliser, il est très
difficile de démontrer la capacité temps réel dur d'un système
multi-tâches basé sur un microcontrôleur qui dépend pour beaucoup du
matériel et de la manière dont on développe l'application. Ce 
\enquote{soft} tient certainement plus d'une rigueur terminologique que
d'une réelle limitation.\\

\paragraph{Support de la nucleo-f746zg}~\\

Le support de la nucleo-f746zg est incomplet, les cartes nucleo étant
proches les unes des autres, il ne devrait pas être très difficile de
le compléter. Comme vu précédemment le support réseau de la
nucleo-f767zi semble fonctionner et le pwm est implémenté pour la
nucleo-f207zg.\\

Ci-dessous j'ai réalisé un tableau des différentes fonctionnalités
supportées pour chaque carte nuleo : \\
{\tiny
\begin{tabular}{lcc      c     c     c     c     c     c        c               c     c      c          c     c     c     c                 c       c      c}
\toprule
%      & adc & arduino & can & dac & dma & eth & i2c & lpuart & motor  & pwm & qdec & riotboot & rtc & rtt & spi & spi\_gpio & timer & uart & usbdev \\
%      &     &         &     &     &     &     &     &        & driver &     &      &          &     &     &     &    mode   &       &      &        \\
 &\MR{2}{adc}&
        \MR{2}{arduino}&
                  \MR{2}{can}&
                        \MR{2}{dac}&
                              \MR{2}{dma}&
                                    \MR{2}{eth}&
                                          \MR{2}{i2c}&
                                                \MR{2}{lpuart}& motor  &
                                                                  \MR{2}{pwm}&
                                                                        \MR{2}{qdec}&
%      & adc & arduino & can & dac & dma & eth & i2c & lpuart & motor  & pwm & qdec & riotboot & rtc & rtt & spi & spi\_gpio & timer & uart & usbdev \\
%      &     &         &     &     &     &     &     &        & driver &     &      &          &     &     &     &    mode   &       &      &        \\
                                                                               \MR{2}{riotboot}&
                                                                                          \MR{2}{rtc}&
                                                                                                \MR{2}{rtt}&
                                                                                                      \MR{2}{spi}& spi\_gpio &
                                                                                                                        \MR{2}{timer}&
                                                                                                                                \MR{2}{uart}&
                                                                                                                                       \MR{2}{usbdev}\\
       &     &         &     &     &     &     &     &        & driver &     &      &          &     &     &     &    mode   &       &      &        \\
%      & adc & arduino & can & dac & dma & eth & i2c & lpuart & motor  & pwm & qdec & riotboot & rtc & rtt & spi & spi\_gpio & timer & uart & usbdev \\
%      &     &         &     &     &     &     &     &        & driver &     &      &          &     &     &     &    mode   &       &      &        \\
\midrule
f030r8 & \cm &   \cm   & \xm & \xm & \xm & \xm & \xm &  \xm   &  \xm   & \cm & \xm  &   \xm    & \cm & \xm & \xm &    \xm    &  \cm  & \cm  &  \xm   \\
f031k6 & \cm &   \cm   & \xm & \xm & \xm & \xm & \xm &  \xm   &  \xm   & \cm & \xm  &   \xm    & \cm & \xm & \cm &    \xm    &  \cm  & \cm  &  \xm   \\
f042k6 & \cm &   \cm   & \xm & \xm & \xm & \xm & \xm &  \xm   &  \xm   & \cm & \xm  &   \xm    & \cm & \xm & \cm &    \xm    &  \cm  & \cm  &  \xm   \\
f070rb & \cm &   \cm   & \xm & \xm & \xm & \xm & \cm &  \xm   &  \xm   & \cm & \xm  &   \xm    & \cm & \xm & \xm &    \xm    &  \cm  & \cm  &  \xm   \\
f072rb & \cm &   \cm   & \xm & \xm & \xm & \xm & \cm &  \xm   &  \xm   & \cm & \xm  &   \xm    & \cm & \xm & \cm &    \xm    &  \cm  & \cm  &  \xm   \\
f091rc & \cm &   \cm   & \xm & \xm & \cm & \xm & \cm &  \xm   &  \xm   & \cm & \xm  &   \xm    & \cm & \xm & \cm &    \xm    &  \cm  & \cm  &  \xm   \\
f103rb & \xm &   \cm   & \xm & \xm & \xm & \xm & \cm &  \xm   &  \xm   & \xm & \xm  &   \xm    & \xm & \cm & \xm &    \xm    &  \cm  & \cm  &  \xm   \\
f207zg & \xm &   \cm   & \xm & \xm & \cm & \cm & \cm &  \xm   &  \xm   & \cm & \xm  &   \cm    & \cm & \xm & \cm &    \xm    &  \cm  & \cm  &  \cm   \\
f302r8 & \xm &   \cm   & \xm & \xm & \xm & \xm & \cm &  \xm   &  \xm   & \cm & \xm  &   \cm    & \cm & \xm & \cm &    \xm    &  \cm  & \cm  &  \xm   \\
f303k8 & \xm &   \cm   & \xm & \xm & \xm & \xm & \xm &  \xm   &  \xm   & \cm & \xm  &   \cm    & \cm & \xm & \cm &    \xm    &  \cm  & \cm  &  \xm   \\
f303re & \xm &   \cm   & \xm & \xm & \xm & \xm & \cm &  \xm   &  \xm   & \cm & \xm  &   \cm    & \cm & \xm & \cm &    \xm    &  \cm  & \cm  &  \xm   \\
f303ze & \xm &   \cm   & \xm & \xm & \xm & \xm & \xm &  \xm   &  \xm   & \cm & \xm  &   \cm    & \cm & \xm & \cm &    \xm    &  \cm  & \cm  &  \xm   \\
%      & adc & arduino & can & dac & dma & eth & i2c & lpuart & motor  & pwm & qdec & riotboot & rtc & rtt & spi & spi\_gpio & timer & uart & usbdev \\
%      &     &         &     &     &     &     &     &        & driver &     &      &          &     &     &     &    mode   &       &      &        \\
f334r8 & \xm &   \cm   & \xm & \xm & \xm & \xm & \xm &  \xm   &  \xm   & \cm & \xm  &   \cm    & \cm & \xm & \cm &    \xm    &  \cm  & \cm  &  \xm   \\
f401re & \cm &   \cm   & \xm & \xm & \xm & \xm & \cm &  \xm   &  \xm   & \cm & \cm  &   \xm    & \cm & \xm & \cm &    \xm    &  \cm  & \cm  &  \xm   \\
f410rb & \cm &   \xm   & \xm & \xm & \xm & \xm & \cm &  \xm   &  \xm   & \xm & \xm  &   \xm    & \cm & \xm & \cm &    \xm    &  \cm  & \cm  &  \xm   \\
f411re & \cm &   \xm   & \xm & \xm & \xm & \xm & \cm &  \xm   &  \xm   & \cm & \xm  &   \xm    & \cm & \xm & \cm &    \xm    &  \cm  & \cm  &  \xm   \\
f412zg & \cm &   \cm   & \xm & \xm & \xm & \xm & \cm &  \xm   &  \xm   & \cm & \xm  &   \xm    & \cm & \xm & \cm &    \xm    &  \cm  & \cm  &  \cm   \\
f413zh & \cm &   \cm   & \cm & \xm & \cm & \xm & \cm &  \xm   &  \xm   & \cm & \xm  &   \xm    & \cm & \cm & \cm &    \xm    &  \cm  & \cm  &  \cm   \\
f429zi & \cm &   \cm   & \xm & \xm & \xm & \xm & \cm &  \xm   &  \xm   & \cm & \xm  &   \xm    & \cm & \xm & \cm &    \xm    &  \cm  & \cm  &  \cm   \\
f446re & \cm &   \cm   & \xm & \xm & \xm & \xm & \cm &  \xm   &  \cm   & \cm & \cm  &   \cm    & \cm & \xm & \cm &    \xm    &  \cm  & \cm  &  \xm   \\
f446ze & \xm &   \cm   & \xm & \xm & \xm & \xm & \cm &  \xm   &  \xm   & \cm & \xm  &   \xm    & \cm & \xm & \cm &    \xm    &  \cm  & \cm  &  \cm   \\
f722ze & \xm &   \cm   & \xm & \xm & \xm & \xm & \cm &  \xm   &  \xm   & \xm & \xm  &   \cm    & \cm & \cm & \xm &    \xm    &  \cm  & \cm  &  \cm   \\
\midrule
f746zg & \xm &   \cm   & \xm & \xm & \xm & \xm & \cm &  \xm   &  \xm   & \xm & \xm  &   \cm    & \cm & \cm & \xm &    \xm    &  \cm  & \cm  &  \cm   \\
\midrule
f767zi & \xm &   \cm   & \xm & \xm & \cm & \cm & \cm &  \xm   &  \xm   & \xm & \xm  &   \cm    & \cm & \cm & \cm &    \xm    &  \cm  & \cm  &  \cm   \\
l031k6 & \cm &   \cm   & \xm & \xm & \xm & \xm & \cm &  \xm   &  \xm   & \cm & \xm  &   \xm    & \cm & \cm & \cm &    \xm    &  \cm  & \cm  &  \xm   \\
%      & adc & arduino & can & dac & dma & eth & i2c & lpuart & motor  & pwm & qdec & riotboot & rtc & rtt & spi & spi\_gpio & timer & uart & usbdev \\
%      &     &         &     &     &     &     &     &        & driver &     &      &          &     &     &     &    mode   &       &      &        \\
l053r8 & \xm &   \cm   & \xm & \xm & \xm & \xm & \xm &  \xm   &  \xm   & \cm & \xm  &   \xm    & \cm & \cm & \cm &    \xm    &  \cm  & \cm  &  \xm   \\
l073rz & \cm &   \cm   & \xm & \xm & \xm & \xm & \cm &  \cm   &  \xm   & \cm & \xm  &   \cm    & \cm & \cm & \cm &    \cm    &  \cm  & \cm  &  \xm   \\
l152re & \cm &   \cm   & \xm & \cm & \cm & \xm & \cm &  \xm   &  \xm   & \cm & \xm  &   \cm    & \cm & \xm & \cm &    \cm    &  \cm  & \cm  &  \xm   \\
l412kb & \xm &   \cm   & \xm & \xm & \xm & \xm & \cm &  \xm   &  \xm   & \cm & \xm  &   \cm    & \xm & \cm & \cm &    \xm    &  \cm  & \cm  &  \xm   \\
l432kc & \xm &   \cm   & \xm & \xm & \xm & \xm & \cm &  \xm   &  \xm   & \cm & \xm  &   \cm    & \cm & \cm & \cm &    \xm    &  \cm  & \cm  &  \xm   \\
l433rc & \xm &   \cm   & \xm & \xm & \xm & \xm & \cm &  \cm   &  \xm   & \cm & \xm  &   \cm    & \cm & \cm & \cm &    \xm    &  \cm  & \cm  &  \xm   \\
l452re & \xm &   \cm   & \xm & \xm & \xm & \xm & \xm &  \xm   &  \xm   & \cm & \xm  &   \cm    & \cm & \cm & \cm &    \xm    &  \cm  & \cm  &  \xm   \\
l476rg & \cm &   \cm   & \cm & \xm & \cm & \xm & \cm &  \xm   &  \xm   & \cm & \xm  &   \cm    & \cm & \cm & \cm &    \xm    &  \cm  & \cm  &  \xm   \\
l496zg & \xm &   \cm   & \xm & \xm & \xm & \xm & \cm &  \cm   &  \xm   & \cm & \xm  &   \cm    & \cm & \cm & \cm &    \xm    &  \cm  & \cm  &  \xm   \\
l4r5zi & \xm &   \cm   & \xm & \xm & \xm & \xm & \cm &  \cm   &  \xm   & \xm & \xm  &   \cm    & \cm & \cm & \cm &    \xm    &  \cm  & \cm  &  \xm   \\
\bottomrule
\end{tabular}
}

Il est important de noter qu'il s'agit des fonctionnalités supportées
pour les cartes, les périphériques des MCU étant définis dans les
fichiers \texttt{cpu/stm32f?/include/vendor/stm32f*.h} qui sont fournis
par ST sous licence BSD 3 clauses et sont donc complets.


\subsection{Exemples}
Un petit tour d'horizon des exemples d'application qui pourraient nous
intéresser.

\subsubsection{default}
Une application pour tester le support hardware des cartes.

\subsubsection{suit\_update}
Pour la mise-à-jour de firmware par reseau !

\subsubsection{[asymcute|emcute]\_mqttsn}
Asymcute est un client MQTT-SN asynchrone.\\
Emcute est un client MQTT-SN plus léger (et plus restreint) que
Asymcute.

\subsubsection{dtls-[echo|sock|wolfssl]}
Datagram Transport Layer Security.

\subsubsection{filesystem}
Exemple d'utilisation des systèmes de fichiers spiffs, littlefs, et
constfs.

\subsubsection{gcoap}
CLI pour l'API gcoap (CoAP = Constrained Application Protocol).

\subsubsection{gnrc\_*}
Tout ce qui concerne la pile réseau générique de RIOT.

\subsubsection{nanocoap\_server}
Une implémentation de serveur CoAP.

\subsubsection{ndn-ping}
Named Data Network.

\subsubsection{nimble\_*}
Pour le Bluetooth.

\subsubsection{openthread}
Thread is an IPv6-based, low-power mesh networking technology for IoT
products.

\subsubsection{posix\_sockets}

\subsubsection{saul}
\begin{verbatim}
[S]ensor [A]ctuator [U]ber [L]ayer
Generic sensor/actuator abstraction layer for RIOT. 
\end{verbatim}

\subsubsection{usbus\_minimal}
Exemple minimaliste pour la pile USB de RIOT.

\subsubsection{wakaama}
Wakaama LwM2M example client.


%\subsection{Ubuntu 20.04}
%Packages supplémentaires à installer : python3-serial avr-libc avrdude

\subsection{Ajout de fonctionnalités pour la nucleo-f746zg}
\subsubsection{eth}
Seulement 2 cartes supportent l'ethernet : la nucleo-f207zg et la
nucleo-f767zi dont j'ai déjà utilisé le support lors des tutos. Je vais
donc me baser principalement sur cette dernière.\\

Je vais tout d’abord identifier les fichiers différents :
{\scriptsize
\begin{verbatim}
hugues@Latitude5400:~/Tutorials/RIOT$ diff -qr boards/nucleo-f746zg/ boards/nucleo-f767zi/
Les fichiers boards/nucleo-f746zg/doc.txt et boards/nucleo-f767zi/doc.txt sont différents
Les fichiers boards/nucleo-f746zg/include/periph\_conf.h et boards/nucleo-f767zi/include/periph\_conf.h sont différents
Les fichiers boards/nucleo-f746zg/Makefile.dep et boards/nucleo-f767zi/Makefile.dep sont différents
Les fichiers boards/nucleo-f746zg/Makefile.features et boards/nucleo-f767zi/Makefile.features sont différents
\end{verbatim}
}
Le fichier Makefile.dep :
{\scriptsize
\begin{verbatim}
hugues@Latitude5400:~/Tutorials/RIOT$ diff -u boards/nucleo-f746zg/Makefile.dep boards/nucleo-f207zg/Makefile.dep 
--- boards/nucleo-f746zg/Makefile.dep	2020-05-25 18:11:52.511615116 +0200
+++ boards/nucleo-f207zg/Makefile.dep	2020-05-04 17:04:48.249357834 +0200
@@ -1 +1,5 @@
+ifneq (,$(filter netdev\_default gnrc\_netdev\_default,$(USEMODULE)))
+  USEMODULE += stm32\_eth
+endif
+
 include $(RIOTBOARD)/common/nucleo/Makefile.dep
hugues@Latitude5400:~/Tutorials/RIOT$ diff -q boards/nucleo-f767zi/Makefile.dep boards/nucleo-f207zg/Makefile.dep 
\end{verbatim}
}
Là on voit qu'il manque juste 3 lignes, ces fichiers sont identiques
entre la nucleo-f767zi et la nucleo-f207zg donc il n'y a pas trop de
question à se poser.\\

Après ça on tente de compiler la \enquote{Task-06} des tutos, on obtient
le message suivant :
{\scriptsize
\begin{verbatim}
hugues@Latitude5400:~/Tutorials/task-06$ make all
There are unsatisfied feature requirements: periph\_dma periph\_eth
/home/hugues/Tutorials/task-06/../RIOT/Makefile.include:841: *** You can let the build continue on expected errors
by setting CONTINUE\_ON\_EXPECTED\_ERRORS=1 to the command line. Arrêt.
\end{verbatim}
}
Ce qui montre que \texttt{periph\_eth} dépend de \texttt{periph\_dma}.\\

J'ajoute donc ces 2 periphériques au fichier
\texttt{Makefile.features} :
\begin{verbatim}
FEATURES\_PROVIDED += periph\_dma
FEATURES\_PROVIDED += periph\_eth
\end{verbatim}

Maintenant la compilation échoue, il faut compléter le fichier
\texttt{include/periph\_conf.h} :
\begin{verbatim}
...
\end{verbatim}

{\scriptsize
\begin{verbatim}
hugues@Latitude5400:~/Tutorials/RIOT$ diff -u boards/nucleo-f746zg/Makefile.dep boards/nucleo-f767zi/Makefile.dep
--- boards/nucleo-f746zg/Makefile.dep	2020-05-25 18:11:52.511615116 +0200
+++ boards/nucleo-f767zi/Makefile.dep	2020-05-04 17:04:48.253357832 +0200
@@ -1 +1,5 @@
+ifneq (,$(filter netdev\_default gnrc\_netdev\_default,$(USEMODULE)))
+  USEMODULE += stm32\_eth
+endif
+
 include $(RIOTBOARD)/common/nucleo/Makefile.dep
hugues@Latitude5400:~/Tutorials/RIOT$ diff -u boards/nucleo-f746zg/Makefile.dep boards/nucleo-f207zg/Makefile.dep 
--- boards/nucleo-f746zg/Makefile.dep	2020-05-25 18:11:52.511615116 +0200
+++ boards/nucleo-f207zg/Makefile.dep	2020-05-04 17:04:48.249357834 +0200
@@ -1 +1,5 @@
+ifneq (,$(filter netdev\_default gnrc\_netdev\_default,$(USEMODULE)))
+  USEMODULE += stm32\_eth
+endif
+
 include $(RIOTBOARD)/common/nucleo/Makefile.dep
\end{verbatim}
}
...

\subsubsection{pwm}
Nous aurons besoins d'au moins 10 signaux pwm et leurs compléments donc
un total de 20.\\

Le stm32 intègre différents timers dont une douzaine exploitables pour
générer des signaux pwm, chacun de ces timers disposant de plusieurs
canaux :
\begin{itemize}
    \item TIM1/TIM8 : 6 canaux ;
    \item TIM2/TIM3/TIM4/TIM5 : 4 canaux ;
    \item TIM9/TIM10/TIM11/TIM12/TIM13/TIM14 : 2 canaux.
\end{itemize}
~\\

Ces canaux ne sont pas tous routables vers des broches de sorties, et
n'ont pas tous de sortie complémentaire : les 2 timers à 6
canaux n'ont que 4 canaux routables et seulement 3 sorties
complémentaire, les autres n'ont pas de sorties complémentaire.\\

Le driver pwm existant dans RIOT est buggé et ne gère pas les sorties
complémentaires. On peut produire des signaux complémentaires en
utilisant 2 canaux avec quelques modifications mineurs, pour utiliser
les sorties complémentaires c'est plus compliqué mais ça permettrait 4
sorties de plus au besoin.\\

Le stm32-f746zg existe en différents boîtiers de 100 à 216 pins et nous
utilisons la version à 144 pins sur laquelle les sorties des timers ne
sont peut-être pas toutes câblées et la carte nucleo dispose d'une
connectique \enquote{ST Zio} à 100 contacts mâle et femelle qu'il serait
préférable d'utiliser plutôt que la connectique \enquote{ST morpho} à
144 contacts pour laquelle il faut souder les contacts à la main.\\

Les sorties pwm disponibles sur les connecteurs \enquote{ST Zio} :\\
{\footnotesize
\begin{tabular}{rcccccl}
\toprule
   & Connector     & Pin & Pin name & Signal name                   & STM32 pin  & Function                  \\
\midrule
1  & \MR{3}{CN7}   & 14  & D11      & SPI\_A\_MOSI / TIMER\_E\_PWM1 & PA7 or PB5   & SPI1\_MOSI / TIM14\_CH1 \\
2  &               & 16  & D10      & SPI\_A\_CS / TIMER\_B\_PWM3   & PD14         & SPI1\_CS / TIM4\_CH3    \\
3  &               & 18  & D9       & TIMER\_B\_PWM2                & PD15         & TIM4\_CH4               \\
\midrule
4  & \MR{10}{CN10} & 29  & D32      & TIMER\_C\_PWM1                & PA0          & TIM2\_CH1               \\
5  &               & 31  & D33      & TIMER\_D\_PWM1                & PB0          & TIM3\_CH3               \\
6  &               & 4   & D6       & TIMER\_A\_PWM1                & PE9          & TIM1\_CH1               \\
7  &               & 6   & D5       & TIMER\_A\_PWM2                & PE11         & TIM1\_CH2               \\
8  &               & 10  & D3       & TIMER\_A\_PWM3                & PE13         & TIM1\_CH3               \\
9  &               & 18  & D42      & TIMER\_A\_PWM1N               & PE8          & TIM1\_CH1N              \\
10 &               & 24  & D40      & TIMER\_A\_PWM2N               & PE10         & TIM1\_CH2N              \\
11 &               & 26  & D39      & TIMER\_A\_PWM3N               & PE12         & TIM1\_CH3N              \\
12 &               & 32  & D36      & TIMER\_C\_PWM2                & PB10         & TIM2\_CH3               \\
13 &               & 34  & D35      & TIMER\_C\_PWM3                & PB11         & TIM2\_CH4               \\
\bottomrule
\end{tabular}
}~\\

Comme on peut le voir, il n'y en a que 13 dont 2 partagées avec un bus
SPI et 3 complémentaires donc il faudra faire un compromis.\\

Les sorties pwm disponibles sur les connecteurs \enquote{ST morpho} :\\
\newcommand\tcr[1]{\textcolor{red}{#1}}
\newcommand\tco[1]{\textcolor{orange}{#1}}
\newcommand\tcpk[1]{\textcolor{pink}{#1}}
\newcommand\tcm[1]{\textcolor{magenta}{#1}}
\newcommand\tcv[1]{\textcolor{violet}{#1}}
\newcommand\tcg[1]{\textcolor{gray}{#1}}
\newcommand\tcgr[1]{\textcolor{green}{#1}}
{\tiny
\begin{tabular}{rccl}
\toprule
   & LQFP144 Pin & Pin name & Functions \\
\midrule
1  & 4           & PE5      & TRACED2, TIM9\_CH1, \tcr{SPI4\_MISO}, SAI1\_SCK\_A, FMC\_A21, DCMI\_D6, LCD\_G0 \\
2  & 5           & PE6      & TRACED3, TIM1\_BKIN2, TIM9\_CH2, \tcr{SPI4\_MOSI}, SAI1\_SD\_A, SAI2\_MCK\_B, FMC\_A22, DCMI\_D7, LCD\_G1 \\
3  & 18          & PF6      & TIM10\_CH1, SPI5\_NSS, SAI1\_SD\_B, UART7\_Rx, QUADSPI\_BK1\_IO3 \\
4  & 19          & PF7      & TIM11\_CH1, \tcr{SPI5\_SCK}, SAI1\_MCLK\_B, UART7\_Tx, QUADSPI\_BK1\_IO2 \\
5  & 20          & PF8      & \tcr{SPI5\_MISO}, SAI1\_SCK\_B, UART7\_RTS, TIM13\_CH1, QUADSPI\_BK1\_IO0 \\
6  & 21          & PF9      & \tcr{SPI5\_MOSI}, SAI1\_FS\_B, UART7\_CTS, TIM14\_CH1, QUADSPI\_BK1\_IO1 \\
7  & 34          & PA0      & TIM2\_CH1/TIM2\_ETR, TIM5\_CH1, TIM8\_ETR, USART2\_CTS, UART4\_TX, SAI2\_SD\_B, ETH\_MII\_CRS \\
8  & 35          & PA1      & TIM2\_CH2, TIM5\_CH2, USART2\_RTS, UART4\_RX, QUADSPI\_BK1\_IO3, SAI2\_MCK\_B, ETH\_MII\_RX\_CLK/\tco{ETH\_RMII\_REF\_CLK}, LCD\_R2 \\
9  & 36          & PA2      & TIM2\_CH3, TIM5\_CH3, TIM9\_CH1, USART2\_TX, SAI2\_SCK\_B, \tco{ETH\_MDIO}, LCD\_R1 \\
10 & 37          & PA3      & TIM2\_CH4, TIM5\_CH4, TIM9\_CH2, USART2\_RX, OTG\_HS\_ULPI\_D0, ETH\_MII\_COL, LCD\_B5 \\
11 & 41          & PA5      & TIM2\_CH1/TIM2\_ETR, TIM8\_CH1N, \tcr{SPI1\_SCK}/I2S1\_CK, OTG\_HS\_ULPI\_CK, LCD\_R4 \\
12 & 42          & PA6      & TIM1\_BKIN, TIM3\_CH1, TIM8\_BKIN, \tcg{SPI1\_MISO}, TIM13\_CH1, DCMI\_PIXCLK, LCD\_G2 \\
13 & 43          & PA7      & TIM1\_CH1N, TIM3\_CH2, TIM8\_CH1N, \tcg{SPI1\_MOSI}/I2S1\_SD, TIM14\_CH1, ETH\_MII\_RX\_DV/\tco{ETH\_RMII\_CRS\_DV}, FMC\_SDNWE \\
14 & 46          & PB0      & TIM1\_CH2N, TIM3\_CH3, TIM8\_CH2N, UART4\_CTS, LCD\_R3, OTG\_HS\_ULPI\_D1, ETH\_MII\_RXD2 \\
15 & 47          & PB1      & TIM1\_CH3N, TIM3\_CH4, TIM8\_CH3N, LCD\_R6, OTG\_HS\_ULPI\_D2, ETH\_MII\_RXD3 \\
16 & 59          & PE8      & TIM1\_CH1N, UART7\_Tx, QUADSPI\_BK2\_IO1, FMC\_D5 \\
17 & 60          & PE9      & TIM1\_CH1, UART7\_RTS, QUADSPI\_BK2\_IO2, FMC\_D6 \\
18 & 63          & PE10     & TIM1\_CH2N, UART7\_CTS, QUADSPI\_BK2\_IO3, FMC\_D7 \\
19 & 64          & PE11     & TIM1\_CH2, SPI4\_NSS, SAI2\_SD\_B, FMC\_D8, LCD\_G3 \\
20 & 65          & PE12     & TIM1\_CH3N, \tcr{SPI4\_SCK}, SAI2\_SCK\_B, FMC\_D9, LCD\_B4 \\
21 & 66          & PE13     & TIM1\_CH3, \tcr{SPI4\_MISO}, SAI2\_FS\_B, FMC\_D10, LCD\_DE \\
22 & 67          & PE14     & TIM1\_CH4, \tcr{SPI4\_MOSI}, SAI2\_MCK\_B, FMC\_D11, LCD\_CLK \\
23 & 69          & PB10     & TIM2\_CH3, I2C2\_SCL, \tcr{SPI2\_SCK}/I2S2\_CK, USART3\_TX, OTG\_HS\_ULPI\_D3, ETH\_MII\_RX\_ER, LCD\_G4 \\
24 & 70          & PB11     & TIM2\_CH4, I2C2\_SDA, USART3\_RX, OTG\_HS\_ULPI\_D4, ETH\_MII\_TX\_EN/ETH\_RMII\_TX\_EN, LCD\_G5 \\
25 & 74          & PB13     & TIM1\_CH1N, \tcg{SPI2\_SCK}/I2S2\_CK, USART3\_CTS, CAN2\_TX, OTG\_HS\_ULPI\_D6, ETH\_MII\_TXD1/\tco{ETH\_RMII\_TXD1} \\
26 & 75          & PB14     & TIM1\_CH2N, TIM8\_CH2N, \tcr{SPI2\_MISO}, USART3\_RTS, TIM12\_CH1, OTG\_HS\_DM \\
27 & 76          & PB15     & RTC\_REFIN, TIM1\_CH3N, TIM8\_CH3N, \tcr{SPI2\_MOSI}/I2S2\_SD, TIM12\_CH2, OTG\_HS\_DP \\
28 & 81          & PD12     & TIM4\_CH1, LPTIM1\_IN1, I2C4\_SCL, USART3\_RTS, QUADSPI\_BK1\_IO1, SAI2\_FS\_A, FMC\_A17/FMC\_ALE \\
29 & 82          & PD13     & TIM4\_CH2, LPTIM1\_OUT, I2C4\_SDA, QUADSPI\_BK1\_IO3, SAI2\_SCK\_A, FMC\_A18 \\
30 & 85          & PD14     & TIM4\_CH3, UART8\_CTS, FMC\_D0 \\
31 & 86          & PD15     & TIM4\_CH4, UART8\_RTS, FMC\_D1 \\
32 & 96          & PC6      & TIM3\_CH1, TIM8\_CH1, I2S2\_MCK, USART6\_TX, \tcv{SDMMC1\_D6}, DCMI\_D0, LCD\_HSYNC \\
33 & 97          & PC7      & TIM3\_CH2, TIM8\_CH2, I2S3\_MCK, USART6\_RX, \tcv{SDMMC1\_D7}, DCMI\_D1, LCD\_G6 \\
34 & 98          & PC8      & TRACED1, TIM3\_CH3, TIM8\_CH3, UART5\_RTS, USART6\_CK, \tcv{SDMMC1\_D0}, DCMI\_D2 \\
35 & 99          & PC9      & MCO2, TIM3\_CH4, TIM8\_CH4, I2C3\_SDA, I2S\_CKIN, UART5\_CTS, QUADSPI\_BK1\_IO0, \tcv{SDMMC1\_D1}, DCMI\_D3 \\
36 & 100         & PA8      & MCO1, TIM1\_CH1, TIM8\_BKIN2, I2C3\_SCL, USART1\_CK, OTG\_FS\_SOF, LCD\_R6 \\
37 & 101         & \tcpk{PA9}& TIM1\_CH2, I2C3\_SMBA, \tcr{SPI2\_SCK}/I2S2\_CK, USART1\_TX, DCMI\_D0 \\
38 & 102         & PA10     & TIM1\_CH3, USART1\_RX, \tcpk{OTG\_FS\_ID}, DCMI\_D1 \\
39 & 103         & PA11     & TIM1\_CH4, USART1\_CTS, CAN1\_RX, \tcpk{OTG\_FS\_DM}, LCD\_R4 \\
40 & 133         & \tcm{PB3}& JTDO/TRACESWO, TIM2\_CH2, \tcg{SPI1\_SCK}/I2S1\_CK, \tcg{SPI3\_SCK}/I2S3\_CK \\
41 & 134         & PB4      & NJTRST, TIM3\_CH1, \tcr{SPI1\_MISO}, \tcg{SPI3\_MISO}, SPI2\_NSS/I2S2\_WS \\
42 & 135         & PB5      & TIM3\_CH2, I2C1\_SMBA, \tcr{SPI1\_MOSI}/I2S1\_SD, \tcg{SPI3\_MOSI}/I2S3\_SD, CAN2\_RX, OTG\_HS\_ULPI\_D7, ETH\_PPS\_OUT, FMC\_SDCKE1, DCMI\_D10 \\
43 & 136         & PB6      & TIM4\_CH1, HDMI-CEC, I2C1\_SCL, USART1\_TX, CAN2\_TX, QUADSPI\_BK1\_NCS, FMC\_SDNE1, DCMI\_D5 \\
44 & 137         & PB7      & TIM4\_CH2, I2C1\_SDA, USART1\_RX, FMC\_NL, DCMI\_VSYNC \\
45 & 139         & PB8      & TIM4\_CH3, TIM10\_CH1, I2C1\_SCL, CAN1\_RX, ETH\_MII\_TXD3, \tcv{SDMMC1\_D4}, DCMI\_D6, LCD\_B6 \\
46 & 140         & PB9      & TIM4\_CH4, TIM11\_CH1, I2C1\_SDA, SPI2\_NSS/I2S2\_WS, CAN1\_TX, \tcv{SDMMC1\_D5}, DCMI\_D7, LCD\_B7 \\
\bottomrule
\end{tabular}
}~\\

46 pins peuvent être reliées à des sorties de timers, toutefois
certaines seront inutilisables du fait qu'elles sont soit déjà utilisées
par un périphérique intégré à la Nucleo comme l'ethernet, l'USB et le
ST-Link, soit du fait fait qu'elles pourraient être utilisées pour des
périphériques dont nous aurons besoin comme les bus SPI pour les MCP3208
ou l'interface SDMMC pour une carte mémoire.\\

J'ai mis ces fonctions en évidence par des couleurs :
\begin{itemize}
    \item \tcr{SPI} (Sauf NSS qu'on utilisera pas et qui peut être
                    déconnecté)
    \item \tco{ETH}
    \item \tcpk{USB}
    \item \tcm{ST-Link}
    \item \tcv{SDMMC}
\end{itemize}
~\\

Après ça il ne reste que 15 lignes sans couleur ce qui serait
insuffisant mais on peut voir que les timers et les bus SPI on plusieurs
options de routage, j'en avais donc éliminer quelques uns (en gris) mais
comme toutes les options pour les SPI ne sont pas présentes ce tableau
ne permettait pas d'aller au bout.\\

J'ai donc réalisé un tableau à partir de la table 12 \textit{STM32F745xx
and STM32F746xx alternate function mapping} de la datasheet
\texttt{DocID027590 Rev 4} en y ajoutant les numéros de broches du
LQFP144 de la table 10, et les borchages des connecteurs de la
nucleo-144.\\

Après avoir repéré tous les ports déjà utilisés sur la nucleo ainsi que
ceux prévus pour relier une carte SD, je me suis efforcé de caser 20
sorties PWM et 4 bus SPI dans ce qui restait :\\
\begin{tabular}{crlllr}
\toprule
                &           & Function      & Port  & Conn. & Pin   \\
\midrule
\MR{20}{Timers} &  1        & TIM1\_CH1     & PE9   & CN10  & 4     \\
                &  2        & TIM1\_CH2     & PE11  & CN10  & 6     \\
                &  3        & TIM1\_CH3     & PE13  & CN10  & 10    \\
                &  4        & TIM1\_CH4     & PE14  & CN10  & 28    \\
                &  5        & TIM2\_CH1     & PA15  & CN7   & 9     \\
                &  6        & TIM2\_CH4     & PB11  & CN10  & 34    \\
                &  7        & TIM3\_CH1     & PA6   & CN7   & 12    \\
                &  8        & TIM3\_CH3     & PB0   & CN10  & 31    \\
                &  9        & TIM3\_CH4     & PB1   & CN10  & 7     \\
                & 10        & TIM4\_CH1     & PD12  & CN10  & 21    \\
                & 11        & TIM4\_CH2     & PD13  & CN10  & 19    \\
                & 12        & TIM4\_CH3     & PD14  & CN7   & 16    \\
                & 13        & TIM4\_CH4     & PD15  & CN7   & 18    \\
                & 14        & TIM5\_CH1     & PA0   & CN10  & 29    \\
                & 15        & TIM5\_CH4     & PA3   & CN9   & 1     \\
                & 16        & TIM8\_CH1     & PC6   & CN7   & 1     \\
                & 17        & TIM8\_CH2     & PC7   & CN7   & 11    \\
                & 18        & TIM10\_CH1    & PB8   & CN7   & 2     \\
                & 19        & TIM11\_CH1    & PB9   & CN7   & 4     \\
                & 20        & TIM14\_CH1    & PF9   & CN9   & 28    \\
\midrule
\MR{12}{SPI}    & \MR{3}{1} & SPI1\_SCK     & PA5   & CN7   & 10    \\
                &           & SPI1\_MISO    & PB4   & CN7   & 19    \\
                &           & SPI1\_MOSI    & PB5   & CN7   & 13    \\
                & \MR{3}{2} & SPI2\_SCK     & PB10  & CN10  & 32    \\
                &           & SPI2\_MISO    & PC2   & CN10  & 9     \\
                &           & SPI2\_MOSI    & PB15  & CN7   & 3     \\
                & \MR{3}{3} & SPI4\_SCK     & PE12  & CN10  & 26    \\
                &           & SPI4\_MISO    & PE5   & CN9   & 18    \\
                &           & SPI4\_MOSI    & PE6   & CN9   & 20    \\
                & \MR{3}{4} & SPI5\_SCK     & PF7   & CN9   & 26    \\
                &           & SPI5\_MISO    & PF8   & CN9   & 24    \\
                &           & SPI5\_MOSI    & PF11  & CN12  & 62    \\
\bottomrule
\end{tabular}



























\bibliographystyle{plain}
\bibliography{tests_d_rtos}


\end{document}
