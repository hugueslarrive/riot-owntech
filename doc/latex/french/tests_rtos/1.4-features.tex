\subsubsection{eth}
Seulement 2 cartes supportent l'ethernet : la nucleo-f207zg et la
nucleo-f767zi dont j'ai déjà utilisé le support lors des tutos. Je vais
donc me baser principalement sur cette dernière.\\

Je vais tout d’abord identifier les fichiers différents :
{\scriptsize
\begin{verbatim}
hugues@Latitude5400:~/Tutorials/RIOT$ diff -qr boards/nucleo-f746zg/ boards/nucleo-f767zi/
Les fichiers boards/nucleo-f746zg/doc.txt et boards/nucleo-f767zi/doc.txt sont différents
Les fichiers boards/nucleo-f746zg/include/periph\_conf.h et boards/nucleo-f767zi/include/periph\_conf.h sont différents
Les fichiers boards/nucleo-f746zg/Makefile.dep et boards/nucleo-f767zi/Makefile.dep sont différents
Les fichiers boards/nucleo-f746zg/Makefile.features et boards/nucleo-f767zi/Makefile.features sont différents
\end{verbatim}
}
Le fichier Makefile.dep :
{\scriptsize
\begin{verbatim}
hugues@Latitude5400:~/Tutorials/RIOT$ diff -u boards/nucleo-f746zg/Makefile.dep boards/nucleo-f207zg/Makefile.dep 
--- boards/nucleo-f746zg/Makefile.dep	2020-05-25 18:11:52.511615116 +0200
+++ boards/nucleo-f207zg/Makefile.dep	2020-05-04 17:04:48.249357834 +0200
@@ -1 +1,5 @@
+ifneq (,$(filter netdev\_default gnrc\_netdev\_default,$(USEMODULE)))
+  USEMODULE += stm32\_eth
+endif
+
 include $(RIOTBOARD)/common/nucleo/Makefile.dep
hugues@Latitude5400:~/Tutorials/RIOT$ diff -q boards/nucleo-f767zi/Makefile.dep boards/nucleo-f207zg/Makefile.dep 
\end{verbatim}
}
Là on voit qu'il manque juste 3 lignes, ces fichiers sont identiques
entre la nucleo-f767zi et la nucleo-f207zg donc il n'y a pas trop de
question à se poser.\\

Après ça on tente de compiler la \enquote{Task-06} des tutos, on obtient
le message suivant :
{\scriptsize
\begin{verbatim}
hugues@Latitude5400:~/Tutorials/task-06$ make all
There are unsatisfied feature requirements: periph\_dma periph\_eth
/home/hugues/Tutorials/task-06/../RIOT/Makefile.include:841: *** You can let the build continue on expected errors
by setting CONTINUE\_ON\_EXPECTED\_ERRORS=1 to the command line. Arrêt.
\end{verbatim}
}
Ce qui montre que \texttt{periph\_eth} dépend de \texttt{periph\_dma}.\\

J'ajoute donc ces 2 periphériques au fichier
\texttt{Makefile.features} :
\begin{verbatim}
FEATURES\_PROVIDED += periph\_dma
FEATURES\_PROVIDED += periph\_eth
\end{verbatim}

Maintenant la compilation échoue, il faut compléter le fichier
\texttt{include/periph\_conf.h} :
\begin{verbatim}
...
\end{verbatim}

{\scriptsize
\begin{verbatim}
hugues@Latitude5400:~/Tutorials/RIOT$ diff -u boards/nucleo-f746zg/Makefile.dep boards/nucleo-f767zi/Makefile.dep
--- boards/nucleo-f746zg/Makefile.dep	2020-05-25 18:11:52.511615116 +0200
+++ boards/nucleo-f767zi/Makefile.dep	2020-05-04 17:04:48.253357832 +0200
@@ -1 +1,5 @@
+ifneq (,$(filter netdev\_default gnrc\_netdev\_default,$(USEMODULE)))
+  USEMODULE += stm32\_eth
+endif
+
 include $(RIOTBOARD)/common/nucleo/Makefile.dep
hugues@Latitude5400:~/Tutorials/RIOT$ diff -u boards/nucleo-f746zg/Makefile.dep boards/nucleo-f207zg/Makefile.dep 
--- boards/nucleo-f746zg/Makefile.dep	2020-05-25 18:11:52.511615116 +0200
+++ boards/nucleo-f207zg/Makefile.dep	2020-05-04 17:04:48.249357834 +0200
@@ -1 +1,5 @@
+ifneq (,$(filter netdev\_default gnrc\_netdev\_default,$(USEMODULE)))
+  USEMODULE += stm32\_eth
+endif
+
 include $(RIOTBOARD)/common/nucleo/Makefile.dep
\end{verbatim}
}
...

\subsubsection{pwm}
Nous aurons besoins d'au moins 10 signaux pwm et leurs compléments donc
un total de 20.\\

Le stm32 intègre différents timers dont une douzaine exploitables pour
générer des signaux pwm, chacun de ces timers disposant de plusieurs
canaux :
\begin{itemize}
    \item TIM1/TIM8 : 6 canaux ;
    \item TIM2/TIM3/TIM4/TIM5 : 4 canaux ;
    \item TIM9/TIM10/TIM11/TIM12/TIM13/TIM14 : 2 canaux.
\end{itemize}
~\\

Ces canaux ne sont pas tous routables vers des broches de sorties, et
n'ont pas tous de sortie complémentaire : les 2 timers à 6
canaux n'ont que 4 canaux routables et seulement 3 sorties
complémentaire, les autres n'ont pas de sorties complémentaire.\\

Le driver pwm existant dans RIOT est buggé et ne gère pas les sorties
complémentaires. On peut produire des signaux complémentaires en
utilisant 2 canaux avec quelques modifications mineurs, pour utiliser
les sorties complémentaires c'est plus compliqué mais ça permettrait 4
sorties de plus au besoin.\\

Le stm32-f746zg existe en différents boîtiers de 100 à 216 pins et nous
utilisons la version à 144 pins sur laquelle les sorties des timers ne
sont peut-être pas toutes câblées et la carte nucleo dispose d'une
connectique \enquote{ST Zio} à 100 contacts mâle et femelle qu'il serait
préférable d'utiliser plutôt que la connectique \enquote{ST morpho} à
144 contacts pour laquelle il faut souder les contacts à la main.\\

Les sorties pwm disponibles sur les connecteurs \enquote{ST Zio} :\\
{\footnotesize
\begin{tabular}{rcccccl}
\toprule
   & Connector     & Pin & Pin name & Signal name                   & STM32 pin  & Function                  \\
\midrule
1  & \MR{3}{CN7}   & 14  & D11      & SPI\_A\_MOSI / TIMER\_E\_PWM1 & PA7 or PB5   & SPI1\_MOSI / TIM14\_CH1 \\
2  &               & 16  & D10      & SPI\_A\_CS / TIMER\_B\_PWM3   & PD14         & SPI1\_CS / TIM4\_CH3    \\
3  &               & 18  & D9       & TIMER\_B\_PWM2                & PD15         & TIM4\_CH4               \\
\midrule
4  & \MR{10}{CN10} & 29  & D32      & TIMER\_C\_PWM1                & PA0          & TIM2\_CH1               \\
5  &               & 31  & D33      & TIMER\_D\_PWM1                & PB0          & TIM3\_CH3               \\
6  &               & 4   & D6       & TIMER\_A\_PWM1                & PE9          & TIM1\_CH1               \\
7  &               & 6   & D5       & TIMER\_A\_PWM2                & PE11         & TIM1\_CH2               \\
8  &               & 10  & D3       & TIMER\_A\_PWM3                & PE13         & TIM1\_CH3               \\
9  &               & 18  & D42      & TIMER\_A\_PWM1N               & PE8          & TIM1\_CH1N              \\
10 &               & 24  & D40      & TIMER\_A\_PWM2N               & PE10         & TIM1\_CH2N              \\
11 &               & 26  & D39      & TIMER\_A\_PWM3N               & PE12         & TIM1\_CH3N              \\
12 &               & 32  & D36      & TIMER\_C\_PWM2                & PB10         & TIM2\_CH3               \\
13 &               & 34  & D35      & TIMER\_C\_PWM3                & PB11         & TIM2\_CH4               \\
\bottomrule
\end{tabular}
}~\\

Comme on peut le voir, il n'y en a que 13 dont 2 partagées avec un bus
SPI et 3 complémentaires donc il faudra faire un compromis.\\

Les sorties pwm disponibles sur les connecteurs \enquote{ST morpho} :\\
\newcommand\tcr[1]{\textcolor{red}{#1}}
\newcommand\tco[1]{\textcolor{orange}{#1}}
\newcommand\tcpk[1]{\textcolor{pink}{#1}}
\newcommand\tcm[1]{\textcolor{magenta}{#1}}
\newcommand\tcv[1]{\textcolor{violet}{#1}}
\newcommand\tcg[1]{\textcolor{gray}{#1}}
\newcommand\tcgr[1]{\textcolor{green}{#1}}
{\tiny
\begin{tabular}{rccl}
\toprule
   & LQFP144 Pin & Pin name & Functions \\
\midrule
1  & 4           & PE5      & TRACED2, TIM9\_CH1, \tcr{SPI4\_MISO}, SAI1\_SCK\_A, FMC\_A21, DCMI\_D6, LCD\_G0 \\
2  & 5           & PE6      & TRACED3, TIM1\_BKIN2, TIM9\_CH2, \tcr{SPI4\_MOSI}, SAI1\_SD\_A, SAI2\_MCK\_B, FMC\_A22, DCMI\_D7, LCD\_G1 \\
3  & 18          & PF6      & TIM10\_CH1, SPI5\_NSS, SAI1\_SD\_B, UART7\_Rx, QUADSPI\_BK1\_IO3 \\
4  & 19          & PF7      & TIM11\_CH1, \tcr{SPI5\_SCK}, SAI1\_MCLK\_B, UART7\_Tx, QUADSPI\_BK1\_IO2 \\
5  & 20          & PF8      & \tcr{SPI5\_MISO}, SAI1\_SCK\_B, UART7\_RTS, TIM13\_CH1, QUADSPI\_BK1\_IO0 \\
6  & 21          & PF9      & \tcr{SPI5\_MOSI}, SAI1\_FS\_B, UART7\_CTS, TIM14\_CH1, QUADSPI\_BK1\_IO1 \\
7  & 34          & PA0      & TIM2\_CH1/TIM2\_ETR, TIM5\_CH1, TIM8\_ETR, USART2\_CTS, UART4\_TX, SAI2\_SD\_B, ETH\_MII\_CRS \\
8  & 35          & PA1      & TIM2\_CH2, TIM5\_CH2, USART2\_RTS, UART4\_RX, QUADSPI\_BK1\_IO3, SAI2\_MCK\_B, ETH\_MII\_RX\_CLK/\tco{ETH\_RMII\_REF\_CLK}, LCD\_R2 \\
9  & 36          & PA2      & TIM2\_CH3, TIM5\_CH3, TIM9\_CH1, USART2\_TX, SAI2\_SCK\_B, \tco{ETH\_MDIO}, LCD\_R1 \\
10 & 37          & PA3      & TIM2\_CH4, TIM5\_CH4, TIM9\_CH2, USART2\_RX, OTG\_HS\_ULPI\_D0, ETH\_MII\_COL, LCD\_B5 \\
11 & 41          & PA5      & TIM2\_CH1/TIM2\_ETR, TIM8\_CH1N, \tcr{SPI1\_SCK}/I2S1\_CK, OTG\_HS\_ULPI\_CK, LCD\_R4 \\
12 & 42          & PA6      & TIM1\_BKIN, TIM3\_CH1, TIM8\_BKIN, \tcg{SPI1\_MISO}, TIM13\_CH1, DCMI\_PIXCLK, LCD\_G2 \\
13 & 43          & PA7      & TIM1\_CH1N, TIM3\_CH2, TIM8\_CH1N, \tcg{SPI1\_MOSI}/I2S1\_SD, TIM14\_CH1, ETH\_MII\_RX\_DV/\tco{ETH\_RMII\_CRS\_DV}, FMC\_SDNWE \\
14 & 46          & PB0      & TIM1\_CH2N, TIM3\_CH3, TIM8\_CH2N, UART4\_CTS, LCD\_R3, OTG\_HS\_ULPI\_D1, ETH\_MII\_RXD2 \\
15 & 47          & PB1      & TIM1\_CH3N, TIM3\_CH4, TIM8\_CH3N, LCD\_R6, OTG\_HS\_ULPI\_D2, ETH\_MII\_RXD3 \\
16 & 59          & PE8      & TIM1\_CH1N, UART7\_Tx, QUADSPI\_BK2\_IO1, FMC\_D5 \\
17 & 60          & PE9      & TIM1\_CH1, UART7\_RTS, QUADSPI\_BK2\_IO2, FMC\_D6 \\
18 & 63          & PE10     & TIM1\_CH2N, UART7\_CTS, QUADSPI\_BK2\_IO3, FMC\_D7 \\
19 & 64          & PE11     & TIM1\_CH2, SPI4\_NSS, SAI2\_SD\_B, FMC\_D8, LCD\_G3 \\
20 & 65          & PE12     & TIM1\_CH3N, \tcr{SPI4\_SCK}, SAI2\_SCK\_B, FMC\_D9, LCD\_B4 \\
21 & 66          & PE13     & TIM1\_CH3, \tcr{SPI4\_MISO}, SAI2\_FS\_B, FMC\_D10, LCD\_DE \\
22 & 67          & PE14     & TIM1\_CH4, \tcr{SPI4\_MOSI}, SAI2\_MCK\_B, FMC\_D11, LCD\_CLK \\
23 & 69          & PB10     & TIM2\_CH3, I2C2\_SCL, \tcr{SPI2\_SCK}/I2S2\_CK, USART3\_TX, OTG\_HS\_ULPI\_D3, ETH\_MII\_RX\_ER, LCD\_G4 \\
24 & 70          & PB11     & TIM2\_CH4, I2C2\_SDA, USART3\_RX, OTG\_HS\_ULPI\_D4, ETH\_MII\_TX\_EN/ETH\_RMII\_TX\_EN, LCD\_G5 \\
25 & 74          & PB13     & TIM1\_CH1N, \tcg{SPI2\_SCK}/I2S2\_CK, USART3\_CTS, CAN2\_TX, OTG\_HS\_ULPI\_D6, ETH\_MII\_TXD1/\tco{ETH\_RMII\_TXD1} \\
26 & 75          & PB14     & TIM1\_CH2N, TIM8\_CH2N, \tcr{SPI2\_MISO}, USART3\_RTS, TIM12\_CH1, OTG\_HS\_DM \\
27 & 76          & PB15     & RTC\_REFIN, TIM1\_CH3N, TIM8\_CH3N, \tcr{SPI2\_MOSI}/I2S2\_SD, TIM12\_CH2, OTG\_HS\_DP \\
28 & 81          & PD12     & TIM4\_CH1, LPTIM1\_IN1, I2C4\_SCL, USART3\_RTS, QUADSPI\_BK1\_IO1, SAI2\_FS\_A, FMC\_A17/FMC\_ALE \\
29 & 82          & PD13     & TIM4\_CH2, LPTIM1\_OUT, I2C4\_SDA, QUADSPI\_BK1\_IO3, SAI2\_SCK\_A, FMC\_A18 \\
30 & 85          & PD14     & TIM4\_CH3, UART8\_CTS, FMC\_D0 \\
31 & 86          & PD15     & TIM4\_CH4, UART8\_RTS, FMC\_D1 \\
32 & 96          & PC6      & TIM3\_CH1, TIM8\_CH1, I2S2\_MCK, USART6\_TX, \tcv{SDMMC1\_D6}, DCMI\_D0, LCD\_HSYNC \\
33 & 97          & PC7      & TIM3\_CH2, TIM8\_CH2, I2S3\_MCK, USART6\_RX, \tcv{SDMMC1\_D7}, DCMI\_D1, LCD\_G6 \\
34 & 98          & PC8      & TRACED1, TIM3\_CH3, TIM8\_CH3, UART5\_RTS, USART6\_CK, \tcv{SDMMC1\_D0}, DCMI\_D2 \\
35 & 99          & PC9      & MCO2, TIM3\_CH4, TIM8\_CH4, I2C3\_SDA, I2S\_CKIN, UART5\_CTS, QUADSPI\_BK1\_IO0, \tcv{SDMMC1\_D1}, DCMI\_D3 \\
36 & 100         & PA8      & MCO1, TIM1\_CH1, TIM8\_BKIN2, I2C3\_SCL, USART1\_CK, OTG\_FS\_SOF, LCD\_R6 \\
37 & 101         & \tcpk{PA9}& TIM1\_CH2, I2C3\_SMBA, \tcr{SPI2\_SCK}/I2S2\_CK, USART1\_TX, DCMI\_D0 \\
38 & 102         & PA10     & TIM1\_CH3, USART1\_RX, \tcpk{OTG\_FS\_ID}, DCMI\_D1 \\
39 & 103         & PA11     & TIM1\_CH4, USART1\_CTS, CAN1\_RX, \tcpk{OTG\_FS\_DM}, LCD\_R4 \\
40 & 133         & \tcm{PB3}& JTDO/TRACESWO, TIM2\_CH2, \tcg{SPI1\_SCK}/I2S1\_CK, \tcg{SPI3\_SCK}/I2S3\_CK \\
41 & 134         & PB4      & NJTRST, TIM3\_CH1, \tcr{SPI1\_MISO}, \tcg{SPI3\_MISO}, SPI2\_NSS/I2S2\_WS \\
42 & 135         & PB5      & TIM3\_CH2, I2C1\_SMBA, \tcr{SPI1\_MOSI}/I2S1\_SD, \tcg{SPI3\_MOSI}/I2S3\_SD, CAN2\_RX, OTG\_HS\_ULPI\_D7, ETH\_PPS\_OUT, FMC\_SDCKE1, DCMI\_D10 \\
43 & 136         & PB6      & TIM4\_CH1, HDMI-CEC, I2C1\_SCL, USART1\_TX, CAN2\_TX, QUADSPI\_BK1\_NCS, FMC\_SDNE1, DCMI\_D5 \\
44 & 137         & PB7      & TIM4\_CH2, I2C1\_SDA, USART1\_RX, FMC\_NL, DCMI\_VSYNC \\
45 & 139         & PB8      & TIM4\_CH3, TIM10\_CH1, I2C1\_SCL, CAN1\_RX, ETH\_MII\_TXD3, \tcv{SDMMC1\_D4}, DCMI\_D6, LCD\_B6 \\
46 & 140         & PB9      & TIM4\_CH4, TIM11\_CH1, I2C1\_SDA, SPI2\_NSS/I2S2\_WS, CAN1\_TX, \tcv{SDMMC1\_D5}, DCMI\_D7, LCD\_B7 \\
\bottomrule
\end{tabular}
}~\\

46 pins peuvent être reliées à des sorties de timers, toutefois
certaines seront inutilisables du fait qu'elles sont soit déjà utilisées
par un périphérique intégré à la Nucleo comme l'ethernet, l'USB et le
ST-Link, soit du fait fait qu'elles pourraient être utilisées pour des
périphériques dont nous aurons besoin comme les bus SPI pour les MCP3208
ou l'interface SDMMC pour une carte mémoire.\\

J'ai mis ces fonctions en évidence par des couleurs :
\begin{itemize}
    \item \tcr{SPI} (Sauf NSS qu'on utilisera pas et qui peut être
                    déconnecté)
    \item \tco{ETH}
    \item \tcpk{USB}
    \item \tcm{ST-Link}
    \item \tcv{SDMMC}
\end{itemize}
~\\

Après ça il ne reste que 15 lignes sans couleur ce qui serait
insuffisant mais on peut voir que les timers et les bus SPI on plusieurs
options de routage, j'en avais donc éliminer quelques uns (en gris) mais
comme toutes les options pour les SPI ne sont pas présentes ce tableau
ne permettait pas d'aller au bout.\\

J'ai donc réalisé un tableau à partir de la table 12 \textit{STM32F745xx
and STM32F746xx alternate function mapping} de la datasheet
\texttt{DocID027590 Rev 4} en y ajoutant les numéros de broches du
LQFP144 de la table 10, et les borchages des connecteurs de la
nucleo-144.\\

Après avoir repéré tous les ports déjà utilisés sur la nucleo ainsi que
ceux prévus pour relier une carte SD, je me suis efforcé de caser 20
sorties PWM et 4 bus SPI dans ce qui restait :\\
\begin{tabular}{crlllr}
\toprule
                &           & Function      & Port  & Conn. & Pin   \\
\midrule
\MR{20}{Timers} &  1        & TIM1\_CH1     & PE9   & CN10  & 4     \\
                &  2        & TIM1\_CH2     & PE11  & CN10  & 6     \\
                &  3        & TIM1\_CH3     & PE13  & CN10  & 10    \\
                &  4        & TIM1\_CH4     & PE14  & CN10  & 28    \\
                &  5        & TIM2\_CH1     & PA15  & CN7   & 9     \\
                &  6        & TIM2\_CH4     & PB11  & CN10  & 34    \\
                &  7        & TIM3\_CH1     & PA6   & CN7   & 12    \\
                &  8        & TIM3\_CH3     & PB0   & CN10  & 31    \\
                &  9        & TIM3\_CH4     & PB1   & CN10  & 7     \\
                & 10        & TIM4\_CH1     & PD12  & CN10  & 21    \\
                & 11        & TIM4\_CH2     & PD13  & CN10  & 19    \\
                & 12        & TIM4\_CH3     & PD14  & CN7   & 16    \\
                & 13        & TIM4\_CH4     & PD15  & CN7   & 18    \\
                & 14        & TIM5\_CH1     & PA0   & CN10  & 29    \\
                & 15        & TIM5\_CH4     & PA3   & CN9   & 1     \\
                & 16        & TIM8\_CH1     & PC6   & CN7   & 1     \\
                & 17        & TIM8\_CH2     & PC7   & CN7   & 11    \\
                & 18        & TIM10\_CH1    & PB8   & CN7   & 2     \\
                & 19        & TIM11\_CH1    & PB9   & CN7   & 4     \\
                & 20        & TIM14\_CH1    & PF9   & CN9   & 28    \\
\midrule
\MR{12}{SPI}    & \MR{3}{1} & SPI1\_SCK     & PA5   & CN7   & 10    \\
                &           & SPI1\_MISO    & PB4   & CN7   & 19    \\
                &           & SPI1\_MOSI    & PB5   & CN7   & 13    \\
                & \MR{3}{2} & SPI2\_SCK     & PB10  & CN10  & 32    \\
                &           & SPI2\_MISO    & PC2   & CN10  & 9     \\
                &           & SPI2\_MOSI    & PB15  & CN7   & 3     \\
                & \MR{3}{3} & SPI4\_SCK     & PE12  & CN10  & 26    \\
                &           & SPI4\_MISO    & PE5   & CN9   & 18    \\
                &           & SPI4\_MOSI    & PE6   & CN9   & 20    \\
                & \MR{3}{4} & SPI5\_SCK     & PF7   & CN9   & 26    \\
                &           & SPI5\_MISO    & PF8   & CN9   & 24    \\
                &           & SPI5\_MOSI    & PF11  & CN12  & 62    \\
\bottomrule
\end{tabular}














