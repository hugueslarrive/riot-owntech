\paragraph{Architecture \enquote{native}\\\\}
Le réseau sur architecture \enquote{native} a déjà été traité à la
section 1.1.

\paragraph{Matériel réel\\\\}
La seule carte avec une interface réseau dont je dispose est la
nucleo-f746zg.\\

Malheureusement ça ne fonctionne pas tout seul : tout semble bien se
passer pendant la compilation mais la commande \texttt{ifconfig} ne
retourne rien.\\

Il y a une page \enquote{Driver for stm32 ethernet} dans la
documentation mais elle est vide. Dans le code source, on trouve le
dossier correspondant \texttt{drivers/stm32\_eth}.\\

Une petite recherche :
\begin{verbatim}
hugues@W520:~/Tutorials/RIOT$ grep -r stm32_eth boards/*
boards/nucleo-f207zg/Makefile.dep:  USEMODULE += stm32_eth
boards/nucleo-f767zi/Makefile.dep:  USEMODULE += stm32_eth
\end{verbatim}

La plus proche des 2 est la nucleo-f767zi.
{\scriptsize
\begin{verbatim}
hugues@W520:~/Tutorials/RIOT$ diff -u boards/nucleo-f746zg/Makefile.features boards/nucleo-f767zi/Makefile.features 
--- boards/nucleo-f746zg/Makefile.features	2020-05-04 17:04:48.253357832 +0200
+++ boards/nucleo-f767zi/Makefile.features	2020-05-04 17:04:48.253357832 +0200
@@ -1,13 +1,16 @@
 CPU = stm32f7
-CPU_MODEL = stm32f746zg
+CPU_MODEL = stm32f767zi
 
 # Put defined MCU peripherals here (in alphabetical order)
+FEATURES_PROVIDED += periph_dma
 FEATURES_PROVIDED += periph_i2c
 FEATURES_PROVIDED += periph_rtc
 FEATURES_PROVIDED += periph_rtt
+FEATURES_PROVIDED += periph_spi
 FEATURES_PROVIDED += periph_timer
 FEATURES_PROVIDED += periph_uart
 FEATURES_PROVIDED += periph_usbdev
+FEATURES_PROVIDED += periph_eth
 
 # Put other features for this board (in alphabetical order)
 FEATURES_PROVIDED += riotboot
\end{verbatim}
}
Le support de la nucleo-f746zg est donc incomplet, en plus de l'ethernet,
on aura aussi besoin du SPI et du PWM (présent pour la nucleo-f207zg).\\

Le stm32f746zg et le stm32f767zi étant très proche, je tente ma chance
en utilisant \texttt{BOARD=nucleo-f767zi make all flash term} dans un
premier temps. Là le \texttt{ifconfig} a l'air de fonctionner :
\begin{verbatim}
2020-05-14 01:36:38,623 # This is Task-05
> ifconfig
2020-05-14 01:36:53,581 #  ifconfig
2020-05-14 01:36:53,585 # Iface  4  HWaddr: 06:17:2A:15:1C:41 
2020-05-14 01:36:53,589 #           L2-PDU:1500 Source address length: 6
2020-05-14 01:36:53,590 #           
\end{verbatim}

Lorsque je connecte un câble réseau je vois immédiatement un tas de
paquets reçus par pktdump.\\

Comme je n'ai qu'une carte, je vais lancer un riot native pour essayer
de les faire communiquer. Il faut d'abord configurer une interface tap
et un bridge :
\begin{verbatim}
root@W520:/home/hugues/Tutorials/RIOT# tunctl -u hugues
Set 'tap0' persistent and owned by uid 1000
root@W520:/home/hugues/Tutorials/RIOT# ifconfig tap0 up
root@W520:/home/hugues/Tutorials/RIOT# brctl addbr br0
root@W520:/home/hugues/Tutorials/RIOT# ifconfig br0 up
root@W520:/home/hugues/Tutorials/RIOT# brctl addif br0 tap0 wlp3s0
can't add wlp3s0 to bridge br0: Operation not supported

\end{verbatim}

J'ai dû débrancher mon portable du réseau filaire pour brancher la
nucleo, il n'est malheureusement pas possible de bridger l'interface
wifi (sauf si elle était en mode AP) pour pouvoir faire communiquer la
carte avec un riot native. Par contre mon routeur maison tourne sous
linux et je l'utilise parfois pour faire du vpn niveau 2 avec ssh.
Je vais donc créer un tunnel ethernet vers mon routeur avec un bridge à
chaque extrémité :
{\scriptsize
\begin{verbatim}
root@W520:/home/hugues/Tutorials/task-05# ssh 192.168.21.254 -o Tunnel=ethernet -w 1:1 -f sleep 1
root@W520:/home/hugues/Tutorials/task-05# ssh 192.168.21.254 ifconfig tap1 up
root@W520:/home/hugues/Tutorials/task-05# ssh 192.168.21.254 brctl addif br0 tap1
root@W520:/home/hugues/Tutorials/task-05# ifconfig tap1 up
root@W520:/home/hugues/Tutorials/task-05# brctl addif br0 tap1
root@W520:/home/hugues/Tutorials/task-05# ifconfig tap0
tap0: flags=4099<UP,BROADCAST,MULTICAST>  mtu 1500
        ether fa:1b:cf:b5:6f:ef  txqueuelen 1000  (Ethernet)
\end{verbatim}
}
J'en ai profité pour récupérer l'adresse ethernet de l'interface qui
sera utilisée par le riot native qu'il ne reste plus qu'à lancer :
\begin{verbatim}
hugues@W520:~/Tutorials/task-05$ BOARD=native make all term PORT=tap0
\end{verbatim}

Maintenant j'essaie d'envoyer un paquet avec
\texttt{txtsnd 4 bcast test}.\\

D’abord de la nucleo $\rightarrow$ native :
{\scriptsize
\begin{verbatim}
PKTDUMP: data received:
~~ SNIP  0 - size:  46 byte, type: NETTYPE_UNDEF (0)
00000000  74  65  73  74  00  00  00  00  00  00  00  00  00  00  00  00
00000010  00  00  00  00  00  00  00  00  00  00  00  00  00  00  00  00
00000020  00  00  00  00  00  00  00  00  00  00  00  00  00  00
~~ SNIP  1 - size:  20 byte, type: NETTYPE_NETIF (-1)
if_pid: 4  rssi: 0  lqi: 0
flags: 0x0
src_l2addr: 06:17:2A:15:1C:41
dst_l2addr: FF:FF:FF:FF:FF:FF
~~ PKT    -  2 snips, total size:  66 byte
\end{verbatim}
}
Puis de native $\rightarrow$ nucleo :
{\scriptsize
\begin{verbatim}
2020-05-14 02:00:16,095 # PKTDUMP: data received:
2020-05-14 02:00:16,100 # ~~ SNIP  0 - size:  46 byte, type: NETTYPE_UNDEF (0)
2020-05-14 02:00:16,106 # 00000000  74  65  73  74  00  00  00  00  00  00  00  00  00  00  00  00
2020-05-14 02:00:16,113 # 00000010  00  00  00  00  00  00  00  00  00  00  00  00  00  00  00  00
2020-05-14 02:00:16,119 # 00000020  00  00  00  00  00  00  00  00  00  00  00  00  00  00
2020-05-14 02:00:16,124 # ~~ SNIP  1 - size:  20 byte, type: NETTYPE_NETIF (-1)
2020-05-14 02:00:16,126 # if_pid: 4  rssi: 0  lqi: 0
2020-05-14 02:00:16,127 # flags: 0x0
2020-05-14 02:00:16,131 # src_l2addr: FA:1B:CF:B5:6F:F0
2020-05-14 02:00:16,133 # dst_l2addr: FF:FF:FF:FF:FF:FF
2020-05-14 02:00:16,136 # ~~ PKT    -  2 snips, total size:  66 byte
\end{verbatim}
}
Enfin un peu de nettoyage dans les config réseau :
{\tiny
\begin{verbatim}
root@W520:/home/hugues/Tutorials/task-05# ssh 192.168.21.254 brctl delif br0 tap1
root@W520:/home/hugues/Tutorials/task-05# ssh 192.168.21.254 ifconfig tap1 down
root@W520:/home/hugues/Tutorials/task-05# brctl delif br0 tap0 tap1
root@W520:/home/hugues/Tutorials/task-05# ifconfig tap1 down
root@W520:/home/hugues/Tutorials/task-05# kill `ps aux | grep 'ssh 192.168.21.254 -o Tunnel=ethernet -w 1:1 -f sleep 1' | grep -v grep | awk '{printf $2}'`
root@W520:/home/hugues/Tutorials/task-05# ifconfig tap0 down
root@W520:/home/hugues/Tutorials/task-05# ifconfig br0 down
root@W520:/home/hugues/Tutorials/task-05# brctl delbr br0
root@W520:/home/hugues/Tutorials/task-05# tunctl -d tap0
Set 'tap0' nonpersistent
\end{verbatim}
}

Remarque : 
Finalement je me suis un peu emballé dans la config réseau, le bridge
local était inutile (je l'avais créé dans l'optique de bridger tap0 et
mon interface wifi), j'aurai pu connecter directement le riot native au
bout du tunnel ethernet ssh (tap1).\\

Aux cours de ces expérimentations j'ai parfois rencontré un problème
après une reprogrammation, un reset, ou même un reboot : seul le voyant
orange du RJ45 clignote, le voyant vert reste éteind, le
\texttt{ifconfig} fonctionne bien mais aucun paquet n'est reçu et il est
impossible d'en émettre. J'ai effectué 10 resets et 10 reboots et 10
flashage d'affilés, dans tous les cas le problème s'est produit 4 fois
sur 10.\\

Le problème est connu chez RIOT pour la nucleo-f767zi :\\
\url{https://github.com/RIOT-OS/RIOT/issues/13490}.\\

On peut trouver le même genre de problème sur la nucleo-f746zg avec Mbed
OS décrit ici :\\
\url{https://os.mbed.com/questions/77138/EthernetInterface-unreliable-on
-NUCLEO-F/}.\\

Et là avec lwIP \enquote{Bare Metal} sauf que ça fonctionne sur la
nucléo et pas sur une autre carte de dev :\\
{\scriptsize\url{https://stackoverflow.com/questions/55260931/stm32h7-
lan8742-lwip-only-works-fine-after-power-up-not-after-reset}}.
